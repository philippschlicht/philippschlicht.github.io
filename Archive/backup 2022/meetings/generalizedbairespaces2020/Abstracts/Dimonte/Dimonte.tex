\documentclass[12pt]{article}

\usepackage{geometry}
\usepackage{amssymb,ifthen,fancyhdr,enumerate,xypic,mathrsfs,hyperref,graphicx}
\usepackage{marginnote}
\usepackage[OT2, T1]{fontenc}
\usepackage{bussproofs}
\usepackage[russian, english]{babel}
\usepackage{mathrsfs}
\usepackage{tabularx}
\usepackage{multirow}

\begin{document}

Title: Baire Property and the Ellentuck-Prikry topology

Author: Vincenzo Dimonte

The use of descriptive-set-theoretical tools for the analysis of rank-into-rank axioms predates the modern, systematic approach to generalized descriptive set theory. Traces of Shoenfield's Absoluteness can be found in Martin's work in the 1908s, I0 has been found to have many similarities with the Axiom of Determinacy and, more recently, a generalized version of the perfect set property proved to be central to understand better I0. That is because rank-into-rank axioms involve strong limit cardinals of cofinality $\omega$, and many researchers found such cardinals intuitively similar to $\omega$.

In recent work with Luca Motto Ros we did justice to such an intuition, proving that the spaces connected to rank-into-rank axioms are indeed generalized Cantor spaces, i.e., isomorphic to ${}^\lambda 2$, albeit with $\lambda$ singular of cofinality $\omega$, and past results can be rephrased or reworked in the generalized descriptive set theory context. For example, a recent result of Scott Cramer can be rephrased as: I0 on $\lambda$ implies that every subset of ${}^\lambda 2$ in $L({}^\lambda 2)$ has the $\lambda$-Perfect Set Property, mirroring an analogue result for AD.

The next result would be to prove the same for the $\lambda$-Baire property, but this result has been elusive. We argue that this is because we need to develop more of the generalized descriptive set theory on a singular cardinal to understand what is the ``right'' definition of $\lambda$-Baire. To this end, under modest large cardinals, we define the Ellentuck-Prikry topology, a topology defined from the Prikry forcing just as the Ellentuck topology is defined from the Mathias forcing, and for this topology we prove that in the generalized Cantor space the classical theorems hold, like Baire Category theorem, Mycielski theorem and Kuratowski-Ulam theorem.  

Joint work with Xianghui Shi.

\end{document}