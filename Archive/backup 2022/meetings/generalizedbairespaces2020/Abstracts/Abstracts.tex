\documentclass[a4paper]{amsart}

\usepackage{amsmath,amsthm,amssymb,hyperref,mathrsfs}
\usepackage{aliascnt}
\usepackage{lmodern}
\usepackage[T1]{fontenc}
\usepackage{dsfont}
\usepackage{relsize}

\usepackage[textsize=footnotesize]{todonotes}
\newcommand{\comment}[1]{\todo[fancyline]{#1}}

\usepackage{xcolor}
\definecolor{dblue}{rgb}{0,0,0.70}
\hypersetup{
	unicode=true,
	colorlinks=true,
	citecolor=dblue,
	linkcolor=dblue,
	anchorcolor=dblue
}

\makeatletter
\expandafter\g@addto@macro\csname th@plain\endcsname{%
		\thm@notefont{\bfseries}
	}%
\expandafter\g@addto@macro\csname th@remark\endcsname{%
		\thm@headfont{\bfseries}
	}%
\makeatother

%%%%%

\newtheorem{theorem}{Theorem}[section]	
\newtheorem*{theorem*}{Theorem}

\newaliascnt{lemma}{theorem}
\newtheorem{lemma}[lemma]{Lemma}
\newtheorem{claim}[theorem]{Claim}
\aliascntresetthe{lemma}
\newtheorem*{lemma*}{Lemma}

\newaliascnt{proposition}{theorem}
\newtheorem{proposition}[proposition]{Proposition}
\aliascntresetthe{proposition}

\newaliascnt{corollary}{theorem}
\newtheorem{corollary}[corollary]{Corollary}
\aliascntresetthe{corollary}

\newtheorem{subclaim}[theorem]{Subclaim}
\newtheorem*{subclaim*}{Subclaim}

\providecommand*{\theoremautorefname}{Theorem}
\providecommand*{\propositionautorefname}{Proposition}
\providecommand*{\lemmaautorefname}{Lemma}
\providecommand*{\corollaryautorefname}{Corollary}
\providecommand*{\remarkautorefname}{Remark}
\providecommand*{\questionautorefname}{Question}

\theoremstyle{remark}

\newaliascnt{remark}{theorem}
\newtheorem{remark}[remark]{Remark}
\aliascntresetthe{remark}
\newaliascnt{question}{theorem}
\newtheorem{question}[question]{Question}
\aliascntresetthe{question}
\newtheorem{notation}[theorem]{Notation}

\providecommand*{\remarkautorefname}{Remark}
\providecommand*{\questionautorefname}{Question}

\newtheorem*{question*}{Question}

\newaliascnt{definition}{theorem}
\newtheorem{definition}[definition]{Definition}
\aliascntresetthe{definition}
\providecommand*{\definitionautorefname}{Definition}

\newtheorem*{axiom}{Axiom}

\newaliascnt{example}{theorem}
\newtheorem{example}[example]{Example}
\aliascntresetthe{example} 
\providecommand*{\exampleautorefname}{Example}

\renewcommand{\restriction}{\mathbin\upharpoonright}

\newcommand{\axiomft}[1]{\mathsf{#1}} 
\newcommand{\ZFC}{\axiomft{ZFC}}
\newcommand{\AC}{\axiomft{AC}}
\newcommand{\CH}{\axiomft{CH}}
\newcommand{\AD}{\axiomft{AD}}
\newcommand{\DC}{\axiomft{DC}}
\newcommand{\ZF}{\axiomft{ZF}}
\newcommand{\BPI}{\axiomft{BPI}}
\newcommand{\Ord}{\mathrm{Ord}}
\newcommand{\HOD}{\mathrm{HOD}}
\newcommand{\GCH}{\axiomft{GCH}}
\newcommand{\PFA}{\axiomft{PFA}}
\newcommand{\NS}{\mathrm{NS}}
\newcommand{\IS}{\axiomft{IS}}
\newcommand{\HS}{\axiomft{HS}}
\newcommand{\KWP}{\axiomft{KWP}}
\newcommand{\SVC}{\axiomft{SVC}}
\newcommand{\MA}{\axiomft{MA}}
\newcommand{\GMA}{\axiomft{GMA}}

\DeclareMathOperator{\cf}{cf}
\DeclareMathOperator{\dom}{dom}
\DeclareMathOperator{\rng}{rng}
\DeclareMathOperator{\supp}{supp}
\DeclareMathOperator{\rank}{rank}
\DeclareMathOperator{\sym}{sym}
\DeclareMathOperator{\stem}{stem}
\DeclareMathOperator{\fix}{fix}
\DeclareMathOperator{\fin}{fin}
%\DeclareMathOperator{\res}{res}
\DeclareMathOperator{\id}{id}
\DeclareMathOperator{\aut}{Aut}
%\DeclareMathOperator{\Col}{Col}
\DeclareMathOperator{\iCol}{Col_{inj}}
\DeclareMathOperator{\Add}{Add}
\DeclareMathOperator{\crit}{crit}
\DeclareMathOperator{\Lim}{Lim}
\DeclareMathOperator{\otp}{otp}
\DeclareMathOperator{\tcl}{tcl}
\DeclareMathOperator{\len}{len}
\DeclareMathOperator{\Ult}{Ult}
\newcommand{\sdiff}{\mathbin{\triangle}}
\newcommand{\proves}{\mathrel{\vdash}}
\newcommand{\forces}{\mathrel{\Vdash}}
\newcommand{\nforces}{\mathrel{\not{\forces}}}
\newcommand{\gaut}[1]{{\textstyle\int_{#1}}}
\newcommand{\PP}{\mathbb{P}}
\newcommand{\power}{\mathcal{P}}
\newcommand{\E}{\mathrel{E}}
\newcommand{\QQ}{\mathbb{Q}}
\newcommand{\RR}{\mathbb{R}}
\newcommand{\GG}{\mathbb G}
\newcommand{\BB}{\mathbb B}
\newcommand{\CC}{\mathbb C}
\newcommand{\cF}{\mathcal F}
\newcommand{\cU}{\mathcal U}
\newcommand{\sF}{\mathscr F}
\newcommand{\cG}{\mathcal G}
\newcommand{\cB}{\mathcal B}
\newcommand{\cD}{\mathcal D}
\newcommand{\sG}{\mathscr G}
\newcommand{\sH}{\mathscr H}
\newcommand{\rG}{\mathrm G}
\newcommand{\bD}{\mathbf D}
\newcommand{\fc}{\mathfrak c}
\newcommand{\cS}{\mathcal S}
\newcommand{\cT}{\mathcal T}
\newcommand{\cK}{\mathcal K}
\newcommand{\cC}{\mathcal C}
\newcommand{\cQ}{\mathcal Q}
%\newcommand{\1}{\mathds{1}}
\newcommand{\wt}[1]{\widetilde{#1}}

\newcommand{\jbd}{\mathcal J_{bd}}
\newcommand{\da}{{\downarrow}}

\newcommand{\tupp}[1]{\left\langle#1\right\rangle}
\newcommand{\tup}[1]{\langle#1\rangle}
\newcommand{\sett}[1]{\left\{#1\right\}}
\newcommand{\set}[1]{\{#1\}}
\newcommand{\middd}{\mathrel{}\middle|\mathrel{}}
\newcommand{\midd}{\mid}
\newcommand{\Loo}{\mathcal{L}_{\omega_1,\omega}}
\newcommand{\res}{\mathrm{res}}
\newcommand{\ran}{\mathrm{ran}}
\newcommand{\PS}{\mathbb{P}_{\mathrm{side}}}

\newcommand{\vsp}{\vspace{20pt}}

\usepackage{enumitem}

\newenvironment{enumerate-(a)}{\begin{enumerate}[label={\upshape (\alph*)}, leftmargin=2pc]}{\end{enumerate}}
\newenvironment{enumerate-(1)}{\begin{enumerate}[label={\upshape (\arabic*)}, leftmargin=2pc]}{\end{enumerate}}

%%%%
%\author{Philipp Schlicht}
%\address[Philipp Schlicht]{School of Mathematics, 
%University of Bristol, 
%Fry Building.
%Woodland Road, 
%Bristol, BS8~1UG, UK}
%\email[Philipp Schlicht]{philipp.schlicht@bristol.ac.uk}

%\subjclass[2010]{Primary ; Secondary }
%\keywords{}
%\thanks{This project has received funding from the European Union's Horizon 2020 research and innovation programme under the Marie Sk\l odowska-Curie grant agreement No 794020 (IMIC) of the first author.}



\title[Workshop on generalised Baire spaces]{Fifth Workshop on Generalised Baire Spaces \\ 
School of Mathematics, University of Bristol \\ 
3rd-4th February 2020}

\begin{document}
%\begin{abstract} 
%\end{abstract}

\maketitle

%%%%%%%%%%%%%%
\section*{Schedule} 

\begin{center} 
Monday 
\end{center} 

\bigskip 

\begin{tabular} 
{ l | l | l}
08.45-09.00  & Welcome & \\ 
09.00-09.50  & Philipp L\" ucke & Generalised descriptive set theory and \\ 
& & combinatorics -- the complexity of club filters \\ 
10.00-10.50 & Sarka Stejskalova & Small ultrafilter number and some \\
& & compactness principles \\ 
10.50-11.10  & \emph{Coffee} & \\ 
11.10-12.00  & David Aspero & The kappa-strong proper forcing axiom \\ 
12.00-13.30  & \emph{Lunch} & \\ 
13.30-14.40  & Isabel M\" uller & Tutorial: Dividing lines in model theory \\ 
14.50-16.00  & Miguel Moreno & Tutorial: Generalised Baire spaces and \\ 
 & & model theory \\ 
16.00-16.30  & \emph{Coffee} &  \\ 
16.30-16.55  & Jan Dobrowolski & Generalised Polish structures \\ 
17.00-17.25  & Claudio Agostini & On a class of Polish-like spaces \\ 
19.00  & \emph{Dinner at Koh Thai} & \\ 
\end{tabular} 


\bigskip 
\bigskip 

\begin{center} 
Tuesday
\end{center} 

\bigskip 

\begin{tabular} 
{ l | l | l}
08.30-09.20  & Luca Motto Ros  \hspace{15pt} & A descriptive main gap theorem \\ 
09.30-10.20  & Dorottya Sziraki & Perfect set games and colorings on generalised \\ 
& & Baire spaces \\ 
10.20-10.40 & \emph{Coffee} \\ 
10.40-11.30  & Lorenzo Galeotti & Computing over generalisations of the reals \\ 
11.35-12.00  & Open problems & \\ 
12.00-13.30  & \emph{Lunch} \\ 
13.30-14.20  & Vincenzo Dimonte & Baire property and the Ellentuck-Prikry topology \\ 
14.30-14.55  & Vera Fischer & Higher independence \\ 
15.00-15.25  & Johannes Sch\"urz & A Cohen real out of nowhere \\ 
15.30-15.55  & Radek Honzik & Definable wellorders with compactness principles \\ 
\end{tabular} 




\newpage 

%%%%%%%%%%%%%%
\section*{Abstracts} 





\bigskip 
\noindent 
$\blacktriangleright$ Claudio Agostini, Luca Motto Ros: \emph{On a class of Polish-like spaces}\footnote{The work was supported in part by 
Universit\`a degli Studi of Torino (Italy), in part by the Gruppo Nazionale per le Strutture Algebriche, Geometriche e le loro Applicazioni (GNSAGA) of the Istituto Nazionale di Alta Matematica (INdAM) of Italy.}

\noindent 
Universit\`a degli Studi di Torino, Dipartimento di Matematica ``G. Peano'', Via Carlo Alberto 10, 10123 Torino, Italy

\noindent 
\emph{Emails:} \texttt{claudio.agostini@unito.it}, \texttt{luca.mottoros@unito.it}

%\noindent 
The Cantor and Baire spaces have natural generalizations to uncountable cardinals $\kappa$, and in the last two decades many results have been proven about them under the hypothesis $\kappa^{<\kappa}=\kappa$. There is more uncertainty about what should be the right generalization of Polish spaces. What we expect is a class of spaces of weight $\kappa$ which includes the two previous mentioned and can support most of generalized descriptive set theory.
Recently S. Coskey and P. Schlicht in \cite{CosSch} showed that one property that characterize the completeness of metric spaces in the classical case, being strong Choquet, can be extended to higher cardinals and gives a class of spaces with promising properties. 


The assumption $\kappa^{<\kappa}=\kappa$ is equivalent to $2^{<\kappa}=\kappa$ for $\kappa$ regular, but the second one is suitable also for singular cardinals. 
In a forthcoming paper, V. Dimonte, L. Motto Ros and X. Shi use this last assumption to study descriptive set theory on singular cardinals of countable cofinality, where there is a natural definition of $\lambda$-Polish as a completely metrizable space of weight $\lambda$. 

In this talk, I will present an ongoing work in collaboration with L. Motto Ros where we try to put together this two approaches and study a class of spaces which is suitable for all cardinals satisfying $2^{<\lambda}=\lambda$, but coincide with S. Coskey and P. Schlicht's definition if $\lambda$ is regular, and with V. Dimonte, L. Motto Ros and X. Shi's one if $\lambda$ has countable cofinality. I will define this class and show how many properties that holds in the classical theory may be proved with different or similar tools in this case as well.  
%\keywords{Generalized descriptive set theory, general topology, topological games, strong Choquet spaces.}

\smallskip 
\renewcommand{\section}[2]{}

\begin{thebibliography}{100}

\bibitem{CosSch} S. Coskey and P. Schlicht, \emph{Generalized Choquet spaces}, Fund. Math., \textbf{232} (2016), 227--248.

\bibitem{Kechris} A. Kechris, \emph{Classical descriptive set theory}, Springer Science \& Business Media, 2012.
 
\bibitem{Engelking} R. Engelking, \emph{General Topology}, Heldermann Verlag, Berlin, second ed., 1989.
 
\end{thebibliography}









\vsp

\noindent 
$\blacktriangleright$ David Aspero: \emph{The $\kappa$-Strong Proper Forcing Axiom.} 

\noindent 
? 

\noindent 
\emph{Email:} \texttt{d.aspero@uea.ac.uk}

%\noindent 
This talk will focus on high analogues of the notion of strong properness and on the corresponding forcing axioms. I will also address the semiproper version of these ideas. This is mostly joint work with Cox, Karagila and Weiss.





\vsp 

\noindent 
$\blacktriangleright$ Vincenzo Dimonte: \emph{Baire Property and the Ellentuck-Prikry topology} 
\noindent 
?

\noindent 
\emph{Email:} \texttt{vincenzo.dimonte@gmail.com}

The use of descriptive-set-theoretical tools for the analysis of rank-into-rank axioms predates the modern, systematic approach to generalized descriptive set theory. Traces of Shoenfield's Absoluteness can be found in Martin's work in the 1908s, I0 has been found to have many similarities with the Axiom of Determinacy and, more recently, a generalized version of the perfect set property proved to be central to understand better I0. That is because rank-into-rank axioms involve strong limit cardinals of cofinality $\omega$, and many researchers found such cardinals intuitively similar to $\omega$.

In recent work with Luca Motto Ros we did justice to such an intuition, proving that the spaces connected to rank-into-rank axioms are indeed generalized Cantor spaces, i.e., isomorphic to ${}^\lambda 2$, albeit with $\lambda$ singular of cofinality $\omega$, and past results can be rephrased or reworked in the generalized descriptive set theory context. For example, a recent result of Scott Cramer can be rephrased as: I0 on $\lambda$ implies that every subset of ${}^\lambda 2$ in $L({}^\lambda 2)$ has the $\lambda$-Perfect Set Property, mirroring an analogue result for AD.

The next result would be to prove the same for the $\lambda$-Baire property, but this result has been elusive. We argue that this is because we need to develop more of the generalized descriptive set theory on a singular cardinal to understand what is the ``right'' definition of $\lambda$-Baire. To this end, under modest large cardinals, we define the Ellentuck-Prikry topology, a topology defined from the Prikry forcing just as the Ellentuck topology is defined from the Mathias forcing, and for this topology we prove that in the generalized Cantor space the classical theorems hold, like Baire Category theorem, Mycielski theorem and Kuratowski-Ulam theorem.  

Joint work with Xianghui Shi.








\vsp 

\noindent 
$\blacktriangleright$ Jan Dobrowolski: \emph{Generalised Polish Structures} 

\noindent 
University of Leeds

\noindent 
\emph{Email:} \texttt{?} 

A Polish structure is a continuous action of a Polish group $G$ on a topological space $X$.
In this purely topological setting, Krupinski has defined a ternary notion of independence on $X$ called \emph{nm-independence} (nm stands for non-meagre), which shares a number of nice properties with model-theoretic independence relations such as forking independence in stable theories. This allowed to prove several interesting theorems about Polish structures using tools from model theory.

A natural question is whether one can apply nm-independence to actions studied in model theory, such as those of automorphism groups of models. In this talk I will explain why this is hard if we restrict ourselves to considering only actions of Polish groups (e.g. $Aut(M)$ for a countable model), and why the obstacles would be removed if a certain property of Polish groups can be generalised to closed subgroups of $Sym(\kappa)$ for some cardinal $\kappa$.








\vsp 

\noindent 
$\blacktriangleright$ Vera Fischer: \emph{Higher Independence} 

\noindent 
? 

\noindent 
\emph{Email:} \texttt{?}

%\noindent 
We will consider higher analogues of the notion of 
independence on $\omega$, introduce the number $\mathfrak{i}(\kappa)$ 
and show that consistently $\mathfrak{i}(\kappa)<2^\kappa$ for certain 
$\kappa$.












\vsp 

\noindent 
$\blacktriangleright$ Lorenzo Galeotti: \emph{Computing Over Generalisations of the Reals} 

\noindent 
? 

\noindent 
\emph{Email:} \texttt{?}

In classical computability theory computations are thought of as \emph{finite} and \emph{discrete} processes carried out by idealised machines on a \emph{finite} amount of data. Although these assumptions are quite natural, since the beginning of the research in this area, researchers have been developing theories in which these assumptions are weakened.

Particularly interesting are those notions of computability in which the finiteness of the process and of the data are relaxed. Prominent in this area are models of computability over the real line. These machines do indeed work on (sometimes representations of) real numbers and are in some cases allowed to run for infinitely many stages. 


An extreme weakening of the finiteness restrictions on computability led to the notion of \emph{transfinite computability}. The idea is that of taking classical models of computability and extend them to the transfinite. Particularly important in this context are the works of Hamkins and Lewis on \emph{Infinite Time Turing Machines} (ITTMs) and of Koepke on \emph{Ordinal Turing Machines} (OTMs).


It is natural to ask whether classical notions of computability over the reals can be generalised to the transfinite. The first problem that has to be solved in order to answer this question is that of finding generalisations of the reals which are suitable for computability. The main focus of this talk will be on the development of theories of computability over generalisations of the real line. 

We will begin with a brief introduction to two classical models of computability over the reals. Then, we will present a generalisation of the reals suitable to do real analysis and computability. After this, we will focus on the use of transfinite models of computability in the context of generalisations of the reals. We will present two different approaches to the problem of defining a notion of computability over generalisations of the reals, one which makes use of OTMs and one which uses a generalisation of Blum-Shub-Smale Machines. 








\vsp 

\noindent 
$\blacktriangleright$ Radek Honzik: \emph{A definable well-ordering with compactness principles} 

\noindent 
Charles University, Department of Logic,
Celetn{\' a} 20, Praha 1, 
116 42, Czech Republic 

\noindent 
\emph{Email:} \texttt{radek.honzik@ff.cuni.cz}, \emph{web page:} logika.ff.cuni.cz/radek  

%\noindent 
We will show from the optimal large cardinal assumption that the tree property at $\omega_2$, and other compactness principles, are compatible with a $\Sigma^1_3$ well-ordering of the reals. We will also discuss the connection of this result to MA, BPFA and PFA and in general to the question of manipulating cardinal invariants at $\omega$ with the tree property at $\omega_2$. At the end we will discuss methods and open problems regarding the generalization of this result to the tree property at $\omega_3$ with a $H(\omega_2)$-definable well-ordering of $H(\omega_2)$, or in general with regard to obtaining a desired pattern of generalized cardinal invariants with the tree property.







\vsp 

\noindent 
$\blacktriangleright$ Philipp L\"ucke: \emph{Generalized descriptive set theory and combinatorics -- the complexity of club filters} 

\noindent 
Mathematisches Institut, Rheinische Friedrich-Wilhelms-Universit\"at Bonn, Endeni-cher Allee 60, 53115 Bonn, Germany 

\noindent 
\emph{Email:} \texttt{pluecke@uni-bonn.de}

The investigation of the structural properties of club filters on uncountable regular cardinals plays a central role in modern set theory.    
In particular, questions about the complexity of these filters motivated much of the development of generalized descriptive set theory. In my talk, I will start with a survey of results connecting questions about the definability of club filters with their  structural properties. Afterwards, I will present some recent results about the complexity of club filters on $\omega_2$ in the presence of forcing axioms. 

This is joint work in progress with Sean Cox (VCU Richmond). 








\vsp 

\noindent 
$\blacktriangleright$ Miguel Moreno: \emph{Connections between generalised Baire spaces and model theory} 

\noindent 
Universit\"at Wien, Institut f\"ur Mathematik, 
Kurt G\"odel Research Center, 
Augasse 2-6, UZA 1--Building 2, 
1090 Wien, Austria 

\noindent 
\emph{Email:} \texttt{morenom71@univie.ac.at}

We give an introduction to connections between generalised Baire spaces and model theory. We focus on Borel reducibility of isomorphism relations by the use of Ehrenfeucht-Fraisse gamesand other games. 










\vsp 

\noindent 
$\blacktriangleright$ Luca Motto Ros: \emph{A descriptive main gap theorem} 

\noindent 
Universit\`a degli Studi di Torino, Dipartimento di Matematica ``G. Peano'', Via Carlo Alberto 10, 10123 Torino, Italy

\noindent 
\emph{Email:} \texttt{luca.mottoros@unito.it}

Answering a question of S. Friedman, Hyttinen and Kulikov, we show that there is a tight connection between the depth of a classifiable shallow theory $T$ and the Borel rank of the isomorphism relation $\cong^\kappa_T$ on its models of size $\kappa$, for $\kappa$ any cardinal satisfying $\kappa^{< \kappa} = \kappa > 2^{\aleph_0}$. This yields a descriptive set-theoretical analogue of Shelah’s Main Gap Theorem. We also discuss some limitations to the possible (Borel) complexities of $\cong^\kappa_T$, and provide a characterization of categoricity of $T$ in terms of the descriptive set-theoretical complexity of $\cong^\kappa_T$. Joint work with F. Mangraviti.








\vsp 

\noindent 
$\blacktriangleright$ Isabel M\"uller: \emph{Dividing lines in model theory} 

\noindent 
Faculty of Natural Sciences, Department of Mathematics, 
Huxley Building, 
South Kensington Campus, 
Imperial College London, 
London 
SW7 2AZ

\noindent 
\emph{Email:} \texttt{isabel.muller@imperial.ac.uk}

I will give an introduction to four important dividing lines for elementary classes admitting a structure theory: $\aleph_1$-categorical, $\omega$-stable, superstable and stable. 








\vsp 

\noindent 
$\blacktriangleright$ Johannes Sch\"urz: \emph{A Cohen real out of nowhere} 

\noindent 
?

\noindent 
\emph{Email:} \texttt{?}

Let $\mathbb{M}_\kappa ^{\mathcal{U}}$ denote the $\kappa$-Miller forcing with respect to some uniform ${<} \kappa$-complete normal ultrafilter $\mathcal{U}$ on $\kappa$, i.e. extending the co-bounded filter. 

Let $\mathbb{P}:= \mathlarger{\mathlarger{\mathlarger{*}}} _{n < \omega} \,\, \mathbb{M}_\kappa ^{\dot{\mathcal{U}}_n}$ be an $\omega$-iteration with full support, where $\dot{\mathcal{U}}_n$ are names for uniform ${<} \kappa$-complete normal ultrafilters. Note that any initial segment $\mathbb{P}_n$ satisfies the Laver property, i.e. 

$\forall f \in \kappa^{\kappa} \cap V^{\mathbb{P}_n}\colon \, \exists g \in \kappa^{\kappa} \cap V \,\,\,  f \leq g \Rightarrow \exists \, \text{slalom} \, (s_i)_{i < \kappa} \in V \,\,\, \forall i < \kappa \,\,\, f \restriction i \in s_i \, .$
 We shall show that $\mathbb{P}$ adds a Cohen real over the ground model $V$. This answers a question of V. Fischer and D. Montoya, whether the Laver property is preserved under $\kappa$-support iterations, in the negative.\\








\vsp 

\noindent 
$\blacktriangleright$ \v{S}\'{a}rka Stejskalov\'{a}: \emph{Small ultrafilter number and some compactness principles} 

\noindent 
Department of Logic, Charles University  / Institute of Mathematics, Czech Academy of Sciences; Prague

\noindent 
\emph{Email:} \texttt{sarka.stejskalova@ff.cuni.cz}, \emph{webpage: logika.ff.cuni.cz/sarka} 

The ultrafilter number $\mathfrak{u}(\kappa)$ is one of the \emph{generalized cardinal invariants} which study the combinatorial properties of the spaces $\kappa^\kappa$ or $2^\kappa$ for topological, purely combinatorial, or forcing-related reasons. Since the tree property and the failure of approachability at $\kappa^{++}$ both imply $2^\kappa>\kappa^+$, they make the structure of the generalized cardinal invariants at $\kappa$ possibly non-trivial. It is natural to ask to what extent the invariants can be manipulated while ensuring some form of compactness, such as the tree property or stationary reflection, at $\kappa^{++}$.

In the talk, we will focus on small $\mathfrak{u}(\kappa)$ for $\kappa$ which is a strong limit singular cardinal. The possibility of having $\kappa$ singular is even more interesting from the point of compactness at $\kappa^{++}$: it combines three intriguing properties -- the necessary failure of $\sf SCH
$ at $\kappa$, compactness at $\kappa^{++}$ and non-trivial cardinal invariants at $\kappa$. 











\vsp 

\noindent 
$\blacktriangleright$ Dorottya Szir\'aki: \emph{Perfect set games and colorings on generalized Baire spaces} 

\noindent 
? 

\noindent 
\emph{Email:} \texttt{?} 

The notion of perfectness can be generalized
for the $\kappa$-Baire space
in a number of different ways (when $\kappa=\kappa^{<\kappa}>\omega$). 
We 
discuss
the connections between 
these different generalizations and between 
the games underlying some of their definitions, 
as well as 
the corresponding generalizations of 
scatteredness,
density in itself 
and the Cantor-Bendixson hierarchy.
For example, we show that V\"a\"an\"anen's generalized Cantor-Bendixson theorem is equivalent to the $\kappa$-perfect set property, and is therefore equiconsistent with the existence of an inaccessible cardinal above~$\kappa$. 
If time permits, we will mention 
analogues of the above results
for variants of these
perfect set games associated to open colorings.
As an application, we present 
a Cantor-Bendixson theorem
for independent subsets of ${}^\kappa\kappa$ with respect to~$\mathbf{\Pi}^0_2(\kappa)$ colorings.
\end{document}







\bibliographystyle{amsplain}
\bibliography{references}
\end{document}
