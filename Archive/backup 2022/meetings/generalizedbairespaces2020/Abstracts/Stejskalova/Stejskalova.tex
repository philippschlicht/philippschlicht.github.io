\documentclass{amsart}

\title{Small ultrafilter number and some compactness principles}

\author{\v{S}\'{a}rka Stejskalov\'{a}}
%\author{Second Author}

\address{Department of Logic, Charles University  / Institute of Mathematics, Czech Academy of Sciences; Prague}
\email{sarka.stejskalova@ff.cuni.cz}

\urladdr{logika.ff.cuni.cz/sarka}

\begin{document}

\maketitle
The ultrafilter number $\mathfrak{u}(\kappa)$ is one of the \emph{generalized cardinal invariants} which study the combinatorial properties of the spaces $\kappa^\kappa$ or $2^\kappa$ for topological, purely combinatorial, or forcing-related reasons. Since the tree property and the failure of approachability at $\kappa^{++}$ both imply $2^\kappa>\kappa^+$, they make the structure of the generalized cardinal invariants at $\kappa$ possibly non-trivial. It is natural to ask to what extent the invariants can be manipulated while ensuring some form of compactness, such as the tree property or stationary reflection, at $\kappa^{++}$.

In the talk, we will focus on small $\mathfrak{u}(\kappa)$ for $\kappa$ which is a strong limit singular cardinal. The possibility of having $\kappa$ singular is even more interesting from the point of compactness at $\kappa^{++}$: it combines three intriguing properties -- the necessary failure of $\sf SCH
$ at $\kappa$, compactness at $\kappa^{++}$ and non-trivial cardinal invariants at $\kappa$. 

\end{document}
