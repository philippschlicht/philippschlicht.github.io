\documentclass[handout, dvipsnames, usenames, 9pt, serif]{beamer}

\fontfamily{cmss}
\usetheme[width=2.2cm]{Boadilla} 
\usecolortheme{dolphin}

\setbeamertemplate{frametitle}[default][center]
\setbeamertemplate{footline}[frame number]
\setbeamertemplate{blocks}[rounded]

%%%%%%%%%%%%%%%%%%%%%%%%%%%%

\usepackage[all,cmtip]{xy}
\usepackage{tikz} 
%\usepackage{txfonts} 


\usetikzlibrary{shapes}
\usetikzlibrary{shadows}
\usetikzlibrary{patterns}
\usetikzlibrary{positioning, fit, calc, matrix}
\usetikzlibrary{shapes.multipart}

\usepackage{xcolor} 
\definecolor{darkgreen}{rgb}{0,0.7,0}
\definecolor{darkergreen}{rgb}{0,0.4,0}
\definecolor{darkblue}{rgb}{0,0,0.7}
\definecolor{darkred}{rgb}{0.7,0,0}

\usepackage{stmaryrd} 

%%%%%%%%%%%%%%%%%%%%%%%%%%%%

\setbeamertemplate{itemize item}[triangle] 

%%%%%%%%%%%%%%%%%%%%%%%%%%%%

\newtheorem{proposition}{Proposition}
\newtheorem{question}{Question}

%%%%%%%%%%%%%%%%%%%%%%%%%%%%

\newcommand{\med}{\medskip}
\newcommand{\PP}{\mathbb{P}}
\newcommand{\QQ}{\mathbb{Q}}
\newcommand{\RR}{\mathbb{R}}
\newcommand{\FF}{\mathcal{F}}
\newcommand{\CC}{\mathbb{C}}
\newcommand{\Def}{\mathrm{Def}}
\newcommand{\IA}{\mathsf{IA}}
\newcommand{\BS}{{}^\omega\omega}
\newcommand{\Coll}{\mathrm{Col}}
\newcommand{\U}{\mathsf{U}}
\newcommand{\R}{\mathsf{R}}
\newcommand{\dom}{\mathsf{dom}}
\newcommand{\cb}{\color{blue}}

%%%%%%%%%%%%%%%%%%%%%%%%%%%%

\title[Internal absoluteness]{{\huge{Internal absoluteness}}}

\author[Philipp Schlicht]{Philipp Schlicht, University of Bristol } 
%\vspace{15pt} 
%Joint work with Sandra M\"uller, Universit\"at Wien} 

%\date{23 September 2019} 

\date{%Luminy \\ 
%\smallskip 
23 September 2019 \\ 
\bigskip 
\bigskip 
\noindent\hspace{15pt}\fcolorbox{white}{white}{%
    \minipage[t]{\dimexpr0.11\linewidth-2\fboxsep-2\fboxrule\relax}
    \includegraphics[width=20pt,height=13pt]{EUflag.jpg} 
    \endminipage}\hfill
    \fcolorbox{white}{white}{%
    \minipage[b]{\dimexpr0.84\linewidth-2\fboxsep-2\fboxrule\relax}
%{\it 
{%\fontfamily{lmtt}
\selectfont 
\begin{center} {\scriptsize 
Partially funded by the EU's Horizon 2020 programme, grant No 794020 
%This project has received funding from the European Union's Horizon 2020 research and innovation programme under grant agreement No 794020 
} 
\end{center} 
} 
%} 
    \endminipage}
} 

%%%%%%%%%%%%%%%%%%%%%%%%%%%%

\begin{document}

\begin{frame}
\titlepage % Print the title page as the first slide
\end{frame}



\begin{frame}
\frametitle{Overview} 

\begin{itemize} 
\item 
We study \emph{internal absoluteness} ($\IA$) between $M[g]$ and $V$, where $M\prec H_\theta$ and $g\in V$ is generic over $M$. 
\pause  
\item 
Here we only consider projective absoluteness. 
\pause  
\item 
The story is that these or similar principles were used in proofs of Steel and Woodin (see Steel 2008) for the tree production lemma 
\pause  
%\item 
%For instance, Steel (2008) uses a connection with universally Baire sets. 
%\pause  
\item 
{\color{blue} Our motivation:} \\ 
We re-discovered such principles for an application (2018) to selectors for ideals. 
%\vspace{2pt} 
%While studying these principles, we independently 
and a connection with universally Baire sets. 
\pause  
\item 
{\color{blue} Our aims:} \\ 
Fix a class of forcings. 
\begin{itemize} 
\item 
(How) can $\IA$ be characterized via descriptive set theory? 
\item 
How is $\IA$ related to other notions of absoluteness? 
\item 
What is the consistency strength of $\IA$? 
\end{itemize} 
\pause 
%relate these principles to other notions of absoluteness and determine their consistency strength for interesting classes of forcings. 
\item 
This is recent work in progress with Sandra M\"uller. 
\end{itemize} 

\end{frame}




\begin{frame}{Forcing $\leftrightsquigarrow$ descriptive set theory} 

Zapletal (2008): TFAE for forcings of the form $\PP=Borel/I$, where $I$ is a $\sigma$-ideal 
\pause 

{\center{ 
\begin{tabular} 
{ l | r }
\emph{Forcing} & \emph{Descriptive set theory} \\ 
\hline 
%\ & \ \\ 
$\PP$ is {\cb proper} \ \ \ \ \ \ \ \ \ \ \ \ \ \ \ \ \ \ \ \ \ \ \ \ \ \ \ \ \ \ \ \ \ \ & For every countable transitive $M\prec H_\theta$  \\ 
\ &  and every $A\in \PP\cap M$, the set of \\ 
\ & $\PP$-generic reals over $M$ in $A$ is $I$-positive \\ 
\ & \\ 
\pause 
$\PP$ is {\cb $\omega^\omega$-bounding} & Every Borel $I$-positive set contains \\ 
\ & a compact $I$-positive subset,  \\ 
\hspace{100pt} 
\ & and continuous reading of names \\ 
\end{tabular} 
} }

\bigskip 
The second equivalence assumes that $\PP$ is proper. Zapletal proved a large number of such characterisations. 

\end{frame} 


%\begin{frame} 
%\frametitle{Forcing versus descriptive set theory} 
%
%A forcing whose conditions are perfect subtrees of $\omega^{<\omega}$ is called a \emph{tree forcing}. 
%
%\begin{theorem}[Ikegami 2010] 
%Suppose that $\PP$ is a proper tree forcing. TFAE: 
%\begin{itemize}
%\item 
%$1$-step ${\bf \Sigma}^1_3$-$\PP$-absoluteness. 
%\item 
%Every ${\bf \Delta}^1_2$ set is $\PP$-measurable. 
%\end{itemize} 
%\end{theorem} 
%
%\begin{theorem}[Ikegami 2010] 
%Suppose that every real has a sharp, but there's no inner model with a Woodin cardinal. 
%
%
%Suppose that $\PP$ is a proper tree forcing. TFAE: 
%\begin{itemize}
%\item 
%$1$-step ${\bf \Sigma}^1_4$-$\PP$-absoluteness. 
%\item 
%Every ${\bf \Delta}^1_3$ set is $\PP$-measurable. 
%\end{itemize} 
%\end{theorem} 
%
%The full equivalence isn't provable in $\mathsf{ZFC}$: projective $1$-step Cohen absoluteness doesn't imply that all projective sets have the Baire property. 
%
%\end{frame} 



\begin{frame}{Forcing $\leftrightsquigarrow$ descriptive set theory} 

%For proper tree forcings $\PP$, Ikegami proved the following equivalences: 
Ikegami (2010): TFAE for proper tree forcings $\PP$ 
%in some cases previously proved by Brendle and others. 

{\center{ 
\begin{tabular} 
{ l | r }
\emph{Forcing} &  \ \ \ \ \ \ \ \ \ \ \ \ \ \ \ \ \ \ \ \ \ \ \  \emph{Descriptive set theory} \\ 
\hline 
$1$-step {\cb ${\bf \Sigma}^1_3$}-$\PP$-absoluteness \ \ \ \ \ \ \ \ \ \ \ \ \ \ \ \ \ & $\PP$-measurability of {\cb ${\bf \Delta}^1_2$} sets \\ 
\ & \\ \pause 
$1$-step {\cb ${\bf \Sigma}^1_4$}-$\PP$-absoluteness & $\PP$-measurability of {\cb ${\bf \Delta}^1_3$} sets \\ 
\end{tabular} 
} }

\bigskip 
Restrictions: 

\begin{enumerate} 
\item[1.] 
The second equivalence assumes that {\cb every real has a sharp}, but there's {\cb no inner model with a Woodin cardinal}. 
\pause 
\item[2.] 
There's {\cb no full equivalence} in $\mathsf{ZFC}$: projective $1$-step Cohen absoluteness doesn't imply that all projective sets have the Baire property. 
\end{enumerate} 

\end{frame} 


%\begin{frame}{Forcing versus descriptive set theory} 
%
%The first result is due to Solovay, the second to Judah and Shelah. 
%
%{\center{ 
%\begin{tabular} 
%{ l | r }
%\emph{Forcing} &  \ \ \ \ \ \ \ \ \ \ \ \ \ \ \ \ \ \ \ \ \ \ \  \emph{Descriptive set theory} \\ 
%\hline 
%For every real $x$, the set of Cohen reals & Baire property of ${\bf \Sigma}^1_2$ sets \\ 
%over $L[x]$ is comeager & \\ 
%\ & \\ 
%For every real $x$, there is a Cohen real & Baire property of ${\bf \Delta}^1_2$ sets \\ 
%over $L[x]$ & \\ 
%\end{tabular} 
%} }
%
%\end{frame} 




\begin{frame}
\frametitle{Forcing classes} 

We consider the classes: 

\begin{enumerate} 
\item[1.] 
All forcings 
\item[2.] 
{\cb Proper} forcings 
\item[3.] 
Simply definable proper forcings on the reals: forcings of the form {\cb $Borel/I$} for $\sigma$-ideals $I$; tree forcings 
\item[4.] 
Simply definable {\cb ccc} forcings on the reals 
\end{enumerate} 

%and specific single forcings such as Cohen forcing. 

\end{frame}



\begin{frame}
\frametitle{$1$-step and $2$-step absoluteness} 

%By a \emph{class of forcings}, we always mean a class given by a first-order definition, e.g. the class of all proper forcings. 

Suppose that $\FF$ is a class of forcings and $\Lambda$ a class of formulas $\varphi(.,x)$ (where $x$ ranges over certain parameters). 
%\newline 

\begin{definition}[$1$-step absoluteness] 
{\cb $1$-step $\FF$-$\Lambda$-absoluteness} means: if $\PP\in\FF$ and $G$ is $\PP$-generic over $V$, then {\cb $V\prec_\Lambda V[G]$} holds. 
\end{definition} 
\pause  

\begin{definition}[$2$-step absoluteness] 
{\cb $2$-step $\FF$-$\Lambda$-absoluteness} means: 
%\newline 
If $\PP\in\FF$ and $G$, $H$ are $\PP$-generic filters over $V$ with $V[G]\subseteq V[H]$, then 
%Let further $V[G]$, $\QQ\in \FF$ and $H$ $\QQ$-generic over $V[G]$. 
%\newline 
{\cb $V[G]\prec_\Lambda V[H]$} holds. 
\end{definition} 
\pause 

\medskip 
If $\FF$ contains the trivial forcing, then $2$-step implies $1$-step absoluteness. 
\end{frame}



\begin{frame}
\frametitle{Why $2$-step absoluteness?} 

Consider the class of all forcings. 

\begin{theorem}[Martin, Solovay, Woodin] 
{\cb $2$-step $\Sigma^1_3$-absoluteness} 
%is equivalent to the statement 
$\Longleftrightarrow$ 
{\cb ``$\forall X\ X^\#$ exists"}, where $X$ denotes sets of ordinals. 
\end{theorem} 
\pause 

\begin{theorem}[Feng, Magidor, Woodin 1992] 
{\cb $1$-step $\Sigma^1_3$-absoluteness} is equiconsistent with the existence of a {\cb regular cardinal $\kappa$ with $H_\kappa\prec_{\Sigma_2} V$} (weaker than a Mahlo cardinal). 
\end{theorem} 
\pause 

%\begin{itemize} 
%\item 
%Martin, Solovay and Woodin:  {\cb $2$-step $\Sigma^1_3$-absoluteness} 
%$\Longleftrightarrow$ 
%{\cb ``$\forall X\ X^\#$ exists"}, where $X$ denotes sets of ordinals. 
%\pause  
%\item 
%Feng, Magidor and Woodin: {\cb $1$-step $\Sigma^1_3$-absoluteness} is equiconsistent with the existence of a {\cb regular cardinal $\kappa$ with $H_\kappa\prec_{\Sigma_2} V$} (weaker than a Mahlo cardinal). 
%\pause
%\end{itemize} 

Here $2$-step-absoluteness is strictly stronger. 
% than $1$-step absoluteness. 
\pause 

\bigskip 
This is not always the case: 

\begin{theorem}[Steel, Woodin] 
The first-order theory {\cb $\mathrm{Th}(L(\RR))$ \cb cannot be changed} by forcing $\Longleftrightarrow$ {\cb $\mathsf{AD}^{L(\RR)}$} holds in all generic extensions, assuming $\mathrm{Ord}$ is measurable in an outer model.  
\end{theorem} 
%\pause 

The proof shows that $1$-step and $2$-step absoluteness {\cb are equivalent} for formulas of the form $\varphi^{L(\RR)}$ with real parameters. 

\end{frame}



\begin{frame} 
\frametitle{Internal absoluteness} 

\begin{definition}[Internal absoluteness] 
Let $\FF$ be a class of forcings, $\Lambda$ a class of formulas and $\theta$ an uncountable regular cardinal. 
\begin{enumerate} 
\item[1.] 
For any countable $M\prec H_\theta$, {\color{blue}$\IA^M_{\FF,\Lambda}$} denotes the statement: \\ 
$${\color{blue} M[g] \prec_{\Lambda} H_\theta}$$ 
holds for all $\mathbb{P} \in \FF \cap \Def(M)$ and all $\mathbb{P}$-generic filters $g \in V$ over $M$. 
\item[2.] 
{\cb $\IA^\theta_{\FF,\Lambda}$} $:\Longleftrightarrow$ $\IA^M_{\FF,\Lambda}$ holds for club many $M\in[H_\theta]^\omega$. 
\end{enumerate} 
\end{definition} 
\pause  

\medskip 
We write  {\color{blue}$\IA$} if $\Lambda$ is the set of projective formulas with real parameters, $\FF$ is the class of all forcings, and $\IA^\theta_{\FF,\Lambda}$ holds for {\cb all regular $\theta$} with $H_\theta$ sufficiently elementary in $V$. 
\pause  

\medskip
{\cb $\IA$ implies} $1$-step and $2$-step absoluteness: 

\xymatrixrowsep{0.1cm}
\xymatrixcolsep{1cm}
\[ \xymatrix{ & M[g]\ar@{->}[dr]^{\prec_{\Lambda}} {}\ar@{->}[dd]^{\subseteq} & \\  
M\ar@{->}[ur]^{\subseteq}   &   & V\ar@{<-}[dl]^{\prec_{\Lambda}} & \\ 
\ & M[h]\ar@{<-}[ul]^{\subseteq} & } 
\] 

\end{frame}



\begin{frame} 
\frametitle{Separating internal from $1$-step and $2$-step absoluteness} 

\begin{fact} 
$\IA_\CC$ implies the {\cb Baire property} for all projective sets. 
\end{fact} 
\pause 
\begin{proof} 
Take a countable $M\prec H_\theta$. Note that the set of Cohen reals $x$ over $M$ is comeager. 

\medskip 
Let $\varphi(x)$ define a projective set. 

\medskip 
The set of Cohen conditions $p$ with {\cb $p\Vdash^M_{\CC}\varphi(\dot{x})$}, where $\dot{x}$ is a name for the Cohen real, defines an open set witnessing the Baire property. 
\end{proof} 

\pause  

\bigskip 
Projective {\cb $1$-step and $2$-step} absoluteness hold in $L^{Add(\omega,\omega_1)}$. 

\medskip 
But not all projective sets have the Baire property in this model, so internal Cohen absoluteness {\cb $\IA_\CC$ fails} there. 

\end{frame}



\iffalse 

\begin{frame} 
\frametitle{Absoluteness versus regularity properties} 

\begin{itemize} 
\item 
Ikegami (2010): 
In many cases, $\PP$-$\Sigma^1_3$-absoluteness holds iff $\Delta^1_2$ sets have the regularity property associated to $\PP$. 
\pause  
\item 
This holds for Cohen forcing ({\color{blue}$\leftrightsquigarrow$} Baire property), random forcing ({\color{blue}$\leftrightsquigarrow$} Lebesgue measurability) and all classical tree forcings. 
\item 
In fact for all forcings on the reals satisfying mild conditions. 
\pause  
\item 
{\color{blue} Problem:} These results do not generalize in the same way to higher projective levels. 
\item 
Example: projective $1$-step Cohen absoluteness does not imply that all projective sets have the Baire property. 
\pause  
\item[{\color{Maroon}$\blacktriangleright$}] 
{\color{Maroon} A possible solution: strengthen both sides to get a natural equivalence. } 
\end{itemize} 

\end{frame}

\fi 



\begin{frame}{Uniformization up to meager} 

\medskip 
We now consider Cohen forcing. 

\medskip 
\begin{proposition}[M\"uller, S.] 
TFAE: 
\begin{enumerate} 
\item[1.]  
$\IA_\CC$ 
\item[2.] 
Every projective relation can be uniformized on a {\cb comeager set}. 
\end{enumerate} 
\end{proposition} 
\pause  

\medskip 
Shelah (1984) proved the consistency of {\cb 2.} relative to $\mathsf{ZFC}$. 
\pause  

\medskip 
\begin{proof}[Proof sketch] 
{\color{blue} 1 $\Rightarrow$ 2: }  
Assume $\IA_\CC$. Let $R=\{(x,y)\mid \varphi(x,y,z)\}$. 
%\pause  

\medskip 
Let $M\prec H_\theta$ with $z\in M$ and $\sigma\in M$ be a name for the Cohen real (all names are nice). 
%\pause  

\medskip 
Let $\tau\in M$ with $1\Vdash \exists y\ \varphi(\sigma,y,z)\Rightarrow \varphi(\sigma,\tau,z)$. 
%\pause  

\medskip 
$\IA_\CC$ implies $M[x]\models \varphi(x,\tau^x,z) \Longleftrightarrow V\models \varphi(x,\tau^x,z)$ for Cohen reals $x\in V$ over $M$. 
%\pause  

\medskip 
The continuous function $x\mapsto \tau^x$ uniformizes $R$ on the set of Cohen reals over $M$. 
\end{proof} 

\end{frame}



\begin{frame}{Uniformization up to meager} 

\begin{proposition}[M\"uller, S.] 
TFAE: 
\begin{enumerate} 
\item[1.]  
$\IA_\CC$ 
\item[2.] 
Every projective relation can be uniformized on a {\cb comeager set}. 
\end{enumerate} 
\end{proposition} 
\pause  

\medskip 
\begin{proof}[Proof sketch] 
{\color{blue} 2 $\Rightarrow$ 1: }  
Take a $\Sigma^1_{n+1}$-formula $\exists y\ \varphi(x,y)$, where $n\geq 1$. 
\pause  

\medskip 
For a nice name $\sigma$ and any $i\in\omega$, let $A_\sigma=\{(x,y) \mid \varphi(\sigma^x,y)\}$. 
%\pause  

\medskip 
Let $M\prec H_\theta$. 
\pause  

\medskip 
{\color{blue} Claim:}  
For any Cohen real $x\in V$ over $M$, $\exists y\ \varphi(\sigma^x,y)$ is downwards absolute from $V$ to $M[x]$.
\pause  

\medskip 
To see this, let $f_\sigma$ be a uniformization of $A_\sigma$ up to meager. Since the existence of $f_\sigma$ is projective, we can take $f_\sigma$ to be projectively defined in $M$. 
%\pause  

\medskip 
Suppose that $\exists y\ \varphi(u,y)$ holds in $V$. 
%\pause  

\medskip 
Recall that $x$ is Cohen generic over $M$ iff $x\in A$ for every comeager Borel set with a code in $M$. So $x\in \dom(f_\sigma)$ and $f_\sigma(x)\in M[x]$. 
\end{proof} 

\end{frame}






\begin{frame}{Forcings of the form $Borel/I$} 

Let $I$ be a $\sigma$-ideal on a Polish space $X$. 
A binary relation $R$ on $X$ is called \emph{total} if $\mathrm{proj}(R)=X$. 
\pause 

\begin{definition}[Uniformization and regularity] 
\begin{enumerate} 
\smallskip 
\item[1.] 
$\U_{I}$ $:\Longleftrightarrow$ 
If $R$ is a projective total relation on $X$, 
then for any Borel set $A\notin I$, $R{\upharpoonright}A$ has a {\cb projective subfunction} with domain $B\notin I$. 
\smallskip \pause 
\item[2.] 
$\R_{I}$ $:\Longleftrightarrow$ 
If $A$ is a projective set and 
$B\notin I$ is a Borel set, then there is some Borel set $C\notin I$ with $C\subseteq B$ and either {\cb $C\cap A=\emptyset$} or {\cb $C\subseteq A$}. 
\end{enumerate} 
\end{definition} 

\medskip 
Clearly $\U_{I}$ $\Longrightarrow$ $\R_{I}$. 

\pause 
\medskip 
Why don't we uniformize everywhere except for a set in $I$? This would be much stronger. 


\end{frame} 




\begin{frame}{Forcings of the form $Borel/I$} 


\begin{theorem}[M\"uller, S.] 
The following statements are equivalent for proper forcings of the form $\PP=Borel/I$: 
\begin{enumerate}
\item[1.]
$\IA_\PP$ 
\item[2.] 
$1$-step $\PP$-absoluteness holds and $\R_I$ holds for all projective sets. 
\item[3.] 
$\U_I$ 
\end{enumerate} 
\end{theorem} 

\pause 
\bigskip 
{\center{ 
\begin{tabular} 
{ l | r }
\emph{Forcing} & \hspace{50pt} \emph{Descriptive set theory} \\ 
\hline 
%\ & \ \\ 
$\IA_\PP$  \hspace{150pt} & $\U_I$ \ for projective sets \\ 
\ & \\ 
$1$-step $\PP$-absoluteness \ \ \ \ \ \ \ \ \ \ \ \ \ \ \ \ \ $ \not \Longrightarrow$  & {\color{red} $\ \Longleftarrow$ $?$} \ \ \ \ \ \ \ \ \ \ \ \  $\R_I$ \ for projective sets \\ 

\end{tabular} 
} }

\pause 
\bigskip 
We have seen {\cb 1.} $\Leftrightarrow$ {\cb 3.} for Cohen forcing. {\cb 1.} $\Rightarrow$ {\cb 2.} is clear. 

\end{frame} 




\begin{frame}{Consequences of internal Cohen absoluteness} 


\begin{fact}[M\"uller, S.] 
$\IA_\CC$ implies: 
\begin{itemize} 
\item[1.] 
$1$-step and $2$-step Cohen {\cb absoluteness} 
\item[2.] 
The {\cb Baire property} for projective sets 
\item[3.] 
For every real $x$, the set of Cohen reals over $L[x]$ is comeager 
\item[4.] 
For every real $x$, there is a real dominating $L[x]$ 
\item[5.] 
Projective {\cb uniformization up to meager} 
\end{itemize} 
\end{fact} 


\end{frame}




\begin{frame}{Obtaining internal absoluteness} 

$\IA_\CC$ is consistent relative to $\mathsf{ZFC}$, by the consistency of projective uniformization up to meager (Shelah 1984). 

\pause 
\bigskip 
In many cases, $\IA_\PP$ follows directly from large cardinals: 

\begin{proposition}[M\"uller, S.] 
$\mathsf{PD}$ implies that any sufficiently absolute Axiom A forcing of the form $\PP=Borel/I$ satisfies {\cb $\IA_\PP$}. 
\end{proposition} 

\medskip 
This is based on work of Fabiana Castiblanco. 

\end{frame}




\begin{frame}{The connection with universally Baire sets} 

Let $A\subseteq \omega^\omega$. 

\begin{definition} [Feng, Magidor, Woodin] 
$A$ is {\cb \emph{universally Baire}} if for every continuous $f\colon X\rightarrow \omega^\omega$, $f^{-1}(A)$ has the Baire property. 

\medskip 
$A$ is {\cb \emph{absolutely complemented}} if there are trees $S$, $T$ with $A=p[S]$ such that  
$$\omega^\omega=p[S]\cup p[T]$$ 
holds in all generic extensions. 
\end{definition} 

\pause 
\smallskip 
Feng, Magidor and Woodin proved the equivalence of these notions: 

\bigskip 
{\center{ 
\begin{tabular} 
{ l | r }
\emph{Forcing} \hspace{120pt} & \hspace{70pt} \emph{Descriptive set theory} \\ 
\hline 
%\ & \ \\ 
$A$ is absolutely complemented & $A$ is universally Baire \\ 
\end{tabular} 
} }


\end{frame} 




\begin{frame}{The connection with universally Baire sets} 

%From now on, consider the class of all forcings. 
%\smallskip 

\begin{definition} 
We call a tree $T$ on $\omega\times \lambda$ {\cb \emph{absolute}} if for all regular $\theta$ such that $H_\theta$ is sufficiently elementary in $V$, there is a club of $M\in[ H_\theta]^\omega$ such that for all generic extensions $M[g]\subseteq V$: 
$$ {\cb  p[T]^{M[g]}=p[T]\cap M[g].} $$ 
\end{definition} 

\pause 
\begin{definition} 
%Suppose that $\lambda$ is an infinite cardinal. 
We call a formula $\varphi(x)$ {\cb \emph{$\PP$-treeable}} if there is an absolute tree $T$ on $\omega\times \lambda$, for some $\lambda$, such that 
$$p[T] = \{ x \in \BS \mid \varphi(x)\}$$ 
holds in every $\PP$-generic extension. 

\medskip 
Moreover, {\cb \emph{treeable}} means $\PP$-treeable for all forcings $\PP$. 
\end{definition} 
\pause  

\begin{fact} 
Every treeable formula is internally absolute. 
\end{fact} 

%This holds by absoluteness of well-foundedness for $$ M\subseteq M[g]\subseteq V  $$

\end{frame}



\begin{frame}{From internally absolute to treeable} 

\begin{proposition}[essentially Steel or Woodin] 
If $\varphi(x)$ is internally absolute, then $\varphi(x)$ is treeable. 
\end{proposition} 
\pause  

\medskip 
\begin{proof}[Proof sketch] 
The tree for $\varphi(x)$ searches for 
\begin{enumerate} 
\item[1.] 
A countable transitive model $M$ of $\mathsf{ZFC}^-$ 
\item[2.] 
An elementary $j\colon M\rightarrow H_\theta$ 
\item[3.] 
A generic filter $g$ over $M$ with $x\in M[g]$ and $M[g]\models \varphi(x)$ 
\end{enumerate} 
\end{proof} 

\medskip 
%Conclusion: 

\xymatrixrowsep{0.4cm}
\xymatrixcolsep{0.2cm}
\[ \xymatrix{ 
%& \IA \ar@{=>}[dr] & \\  
\text{Internal projective absoluteness} \ar@{<=>}@[blue][rr] & & \text{Projective treeability}  
%\ & M[g,h]\ar@{<-}[ul]^{\subseteq} & 
} 
\] 

\end{frame}



\begin{frame}{Internally absolute and universally Baire} 

\medskip 
\begin{block}{} 
The following are equivalent for a formula $\varphi(x)$: 
\begin{enumerate} 
\item[1.] 
$\varphi$ is internally absolute 
\item[2.] 
$\{x\mid \varphi(x)\}$ is universally Baire, with a tree projecting to $\{x\mid \varphi(x)\}$ in all generic extensions. 
\end{enumerate} 
\end{block}{} 

\xymatrixrowsep{0.4cm}
\xymatrixcolsep{0.1cm}
\[ \xymatrix{ & \IA \ar@{=>}@[blue][dr] & \\  
\text{All projective sets are universally Baire } \ar@{<=}@[blue][ur] &   & \text{$1$-step projective absoluteness}  
%\ & M[g,h]\ar@{<-}[ul]^{\subseteq} & 
} 
\] 

%\medskip 
%Recall: 
%Feng, Magidor and Woodin defined a set $A$ of reals to be \emph{universally Baire} if for all forcings (up to a given size), there are trees $S$ and $T$ with 
%\begin{itemize} 
%\item 
%$A=p[S]$ 
%\item 
%$\BS= p[S] \cup p[T]$ in all generic extension for these forcings. 
%\end{itemize} 
%
%\medskip 
%They proved this to be equivalent to: If $f\colon X\rightarrow {}^\omega\omega$ is continuous, where $X$ is any topological space, then $f^{-1}(A)$ has the property of Baire. 

\end{frame}



\begin{frame}{Absoluteness from absorption} 


How does one prove internal absoluteness? 
\pause  

\medskip 
Next is a well-known property that is used in several proofs of absoluteness. 

\begin{definition} 
Let $M$ be an inner model and $\kappa$ a cardinal. $(M,V)$ has the {\cb \emph{${<}\kappa$-absorption}} property for a forcing $\PP$ if: 

\medskip 
For any $\PP$-generic extension $V[H]$ of $V$ and any real $x\in V[H]$, there exist: 
\begin{enumerate} 
\item[1.] 
a forcing $\QQ\in M$ with $|\QQ|^M<\kappa$ and 
\item[2.] 
a $\QQ$-generic filter $I\in V[H]$ over $M$ 
\end{enumerate} 
with $x\in M[I]$. 
\end{definition} 
\pause  

\medskip 
For instance, Schindler (2001) used this property to obtain $L(\RR)$-absoluteness for proper forcings from a remarkable cardinal. 

\end{frame}



\begin{frame}{Internal absoluteness from absorption} 

\begin{lemma}[M\"uller, S.] 
Suppose that $\kappa$ is inaccessible and $G$ is $\Coll(\omega,{<}\kappa)$-generic for $V$. 
Suppose that $(V,V[G])$ has the ${<}\kappa$-absorption property for $\PP$. 

\medskip 
Then every $L(\RR)$-formula with real and ordinal parameters is $\PP$-treeable. 
\end{lemma} 
\pause  

\medskip 
The proof combines previous arguments for absoluteness with the construction of a tree projecting to $\{x\mid \varphi(x)\}$. 

\medskip 
Thus one can, in many cases, prove internal absoluteness in a similar way as $1$-step absoluteness. 

\end{frame}




\begin{frame}{Some open questions} 


\medskip 
For Cohen forcing, our next goals are: 

\begin{question} 
\begin{itemize} 
\item 
Does {\cb $\IA_\CC$ fail} in Shelah's ``first" model of the Baire property for all projective sets? 
%Are any of the following properties equivalent: 
%\begin{enumerate} 
%\item[(a)]
%Internal projective Cohen absoluteness 
%\item[(b)] 
%The property of Baire for all projective sets 
%\item[(c)] 
%Projective Cohen absoluteness and (b) 
%\end{enumerate} 
%\end{question} 
\smallskip 
\item 
%\begin{question} 
For Cohen forcing, does $2$-step imply $1$-step projective absoluteness? 
\end{itemize} 
\end{question} 

\smallskip 
Similar questions appear for forcings of the form $Borel/I$. 
\pause  

\bigskip 
For the class of all forcings, the following are long-standing open questions: 

\begin{question}[Feng, Magidor, Woodin, Wilson] 
\begin{itemize} 
\item 
If every projective set is universally Baire, does {\cb internal} projective absoluteness hold? 
%\end{question} 
\item 
%\begin{question}[Feng, Magidor, Woodin and Wilson] 
If every projective set is universally Baire, does {\cb $1$-step} projective absoluteness hold? 
Does the converse implication hold? 
\end{itemize} 
\end{question} 

\end{frame}




\begin{frame}
\frametitle{Literature} 

\begin{thebibliography}{1}
\bibitem{1} 
Qi Feng, Menachem Magidor, and Hugh Woodin. "Universally Baire sets of reals." Set theory of the continuum. Springer, New York, NY, 1992. 203-242.
\bibitem{2} 
Saharon Shelah. "Can you take Solovay's inaccessible away?." Israel Journal of mathematics 48.1 (1984): 1-47.
\bibitem{3} 
John Steel. The derived model theorem, 2008 
\bibitem{4} 
Jindrich Zapletal. Forcing idealized. Vol. 174. Cambridge: Cambridge University Press, 2008.
\bibitem{5} 
Daisuke Ikegami. "Forcing absoluteness and regularity properties." Annals of Pure and Applied Logic 161.7 (2010): 879-894.
\bibitem{5} 
Sandra M\"uller, Philipp Schlicht. Internal absoluteness, in preparation 
\end{thebibliography} 

\end{frame}



\begin{frame}

\center{ \huge Thank you for listening!}  

\end{frame}


\end{document} 