
\documentclass[10pt]{amsart}
\usepackage{amsmath,amssymb,amscd,amsthm,amsfonts,amstext,amsbsy,mathrsfs,hyperref,upgreek,mathtools,stmaryrd,enumitem,bbm}
%\usepackage{MnSymbol}
%\usepackage[shadow]{todonotes}
% \usepackage{citeref, breakcites}

\usepackage[
%disable, 
textsize=footnotesize,color=yellow!40, bordercolor=white]{todonotes}

\usepackage%[scale=0.682]
[a4paper, margin=1.2in]{geometry}

\usepackage{xcolor}
\definecolor{blue}{rgb}{0,0,1}
\hypersetup{
	unicode=true,
	colorlinks=true,
	citecolor=blue,
	linkcolor=blue,
	anchorcolor=blue
}

\hypersetup{colorlinks=true}

% other fonts
\newcommand\V{\mathsf{V}}
\newcommand{\Minimal}{\MM[\![G]\!]}
\newcommand\bB{\boldsymbol{{\rm B}}}
\newcommand\bN{\boldsymbol{{\rm N}}}
\renewcommand{\L}{\mathcal{L}}
%\newcommand{\C}{\mathcal{C}}
\newcommand{\D}{\mathcal{D}}
\newcommand{\E}{\mathcal{E}}
\newcommand{\M}{\mathcal{M}}
\newcommand{\T}{\mathcal{T}}
\newcommand{\EE}{\mathbb{E}}                                                       
\newcommand{\OO}{\mathbb{O}}
\newcommand{\PP}{\mathbb{P}}
\newcommand{\TT}{\mathbb{T}}
\newcommand{\HH}{\mathbb{H}}
\newcommand{\FF}{\mathbb{F}}
\newcommand{\Ss}{\mathbb{S}}
\newcommand{\HHH}[1]{\mathrm{H}({#1})}
\newcommand{\Hes}{\mathfrak{H}}
\newcommand{\pre}[2]{{}^{#1} #2}
\newcommand{\seq}[2]{\langle #1 \mid #2 \rangle}
\newcommand{\set}[2]{\{ #1 \mid #2 \}}
\newcommand{\Nbhd}{\mathbf{N}}
\newcommand{\Bor}{\mathsf{Bor}}
\newcommand{\pI}{\mathrm{I}}
\newcommand{\pII}{\mathrm{II}}
%\newcommand{\K}[1]{\boldsymbol{K}_{#1}}
\newcommand{\anf}[1]{{\text{``}\hspace{0.3ex}{#1}\hspace{0.01ex}\text{''}}}
\newcommand{\op}{\mathsf{op}}
\newcommand{\rnk}{\mathsf{rnk}}

% operators
\newcommand{\id}{\operatorname{id}}
\newcommand{\cof}{\operatorname{cof}}
\newcommand{\otp}[1]{{{\rm{otp}}\left(#1\right)}}
\newcommand{\supp}[1]{{{\rm{supp}}(#1)}}
\newcommand{\proj}{\operatorname{p}}
\newcommand{\pow}{\mathscr{P}}
\newcommand{\Trcl}{\operatorname{Trcl}}
\newcommand{\p}{\operatorname{p}}
\newcommand{\leng}{\operatorname{lh}}
\newcommand{\range}{\operatorname{range}}
\newcommand{\rank}{\operatorname{rank}}
\newcommand{\Col}{\operatorname{Col}}
\newcommand{\ra}{\rightarrow}
\newcommand{\lr}{\leftrightarrow}
\newcommand{\Llr}{\Longleftrightarrow}
\newcommand{\tc}{\operatorname{tc}}
\newcommand{\depth}{\operatorname{d}}
%\newcommand{\Add}{\operatorname{Add}}
\newcommand{\ul}{\ulcorner}
\newcommand{\ur}{\urcorner}
\newcommand{\lh}{\operatorname{lh}}
\newcommand{\llangle}{\langle\!\langle}
\newcommand{\rrangle}{\rangle\!\rangle}
\newcommand{\vn}{\vec{n}}
\newcommand{\one}{\mathbbm1}
\newcommand{\lkk}{\L_{\kappa,\kappa}}
\newcommand{\gbr}[1]{{\ulcorner{#1}\urcorner}}
\newcommand{\Def}{\mathrm{Def}}
% spaces
\newcommand{\On}{{\mathrm{Ord}}}
%\newcommand{\Card}{{\sf Card}}
\newcommand{\Lim}{{\mathrm{Lim}}}
\newcommand{\Fm}{{\mathrm{Fml}}}
%\newcommand{\Sym}{{\rm Sym}}
\newcommand{\push}{{\mathrm{push}}}

\newcommand{\spl}{\mathrm{split}}


% axioms
\newcommand{\AD}{{\sf AD}}
\newcommand{\ZFC}{{\sf ZFC}}
\newcommand{\ZF}{{\sf ZF}}
\newcommand{\AC}{{\sf AC}}
\newcommand{\DC}{{\sf DC}}
\newcommand{\KM}{{\sf KM}}
\newcommand{\GB}{{\sf GB}}
\newcommand{\GBC}{{\sf GBC}}

\newcommand{\ran}{\mathrm{ran}} 

% others
\newcommand{\Linf}{\L_{\omega_1 \omega}}

%Luecke
\newcommand{\map}[3]{{#1}:{#2}\longrightarrow{#3}}
\newcommand{\Map}[5]{{#1}:{#2}\longrightarrow{#3};~{#4}\longmapsto{#5}}
\newcommand{\pmap}[4]{{#1}:{#2}\xrightarrow{#4}{#3}}
\newcommand{\Set}[2]{\{{#1}~\vert~{#2}\}}
\newcommand{\dom}[1]{{{\rm{dom}}(#1)}}
\newcommand{\length}[1]{{{\rm{lh}}(#1)}}
\newcommand{\VV}{{\rm{V}}}
\newcommand{\WW}{{\rm{W}}}
\newcommand{\LL}{{\rm{L}}}
\newcommand{\Add}[2]{{\rm{Add}}({#1},{#2})}
\newcommand{\goedel}[2]{{\prec}{#1},{#2}{\succ}}
\newcommand{\ot}{ot}




\newcommand{\ZZ}{\mathbb{Z}} 
\newcommand{\QQ}{\mathbb{Q}} 
\newcommand{\RR}{\mathbb{R}} 
\newcommand{\GG}{\mathbb{G}}
\newcommand{\CCC}{\mathbb{C}} 
\newcommand{\NN}{\mathbb{N}} 

\newcommand{\NNN}{\mathcal{N}} 
\newcommand{\MM}{\mathcal{M}}
\newcommand{\KK}{\mathcal{K}} 
\renewcommand{\AA}{\mathcal{A}}
\newcommand{\BB}{\mathcal{B}}
\newcommand{\CC}{\mathcal{C}} 
\newcommand{\DD}{\mathcal{D}} 

%\newcommand{\MM}{\mathbb{M}}
%\newcommand{\NN}{\mathbb{N}} 
%\newcommand{\KK}{\mathbb{K}} 
%\renewcommand{\AA}{\mathbb{A}}
%\newcommand{\BB}{\mathbb{B}}
%\newcommand{\CC}{\mathbb{C}} 
%\newcommand{\DD}{\mathbb{D}} 




\newcommand{\K}{\mathcal{K}} 
\newcommand{\Th}{\mathrm{Th}} 

\newcommand{\DLO}{\mathrm{DLO}} 
\newcommand{\ACF}{\mathrm{ACF}} 

\newcommand{\lex}{\mathrm{lex}} 

\newcommand{\cC}{\mathcal{C}} 
\newcommand{\Mod}{\mathrm{Mod}} 

\newcommand{\acl}{\mathrm{acl}} 
\newcommand{\tp}{\mathrm{tp}} 
\newcommand{\Aut}{\mathrm{Aut}} 
\newcommand{\Sym}{\mathrm{Sym}} 

\newcommand{\Lev}{\mathrm{Lev}} 



%\setcounter{secnumdepth}{2}
% theorems
\newtheorem{theorem}{Theorem}[subsection]
\newtheorem*{theorem*}{Theorem}
\newtheorem{lemma}[theorem]{Lemma}
\newtheorem{corollary}[theorem]{Corollary}
\newtheorem{proposition}[theorem]{Proposition}
\newtheorem{conjecture}[theorem]{Conjecture}
\newtheorem{question}[theorem]{Question}
\newtheorem{problem}[theorem]{Problem}
\newtheorem{observation}[theorem]{Observation}
\newtheorem{claim}[theorem]{Claim}
\newtheorem*{claim*}{Claim}
\newtheorem*{subclaim*}{Subclaim}
\newtheorem*{Vaughtconj}{Vaught's Conjecture}
\theoremstyle{definition}
%\newtheorem{claim}{Claim}[theorem]
\newtheorem{definition}[theorem]{Definition}
\newtheorem*{definition*}{Definition}
\newtheorem*{notation}{Notation}
\newtheorem{fact}[theorem]{Fact}
\newtheorem{example}[theorem]{Example}
\newtheorem*{example*}{Example}
\theoremstyle{remark}
\newtheorem{remark}[theorem]{Remark}



% enumeration

\newenvironment{enumerate-(a)}{\begin{enumerate}[label={\upshape (\alph*)}, leftmargin=2pc]}{\end{enumerate}}

\newenvironment{enumerate-(a)-r}{\begin{enumerate}[label={\upshape (\alph*)}, leftmargin=2pc,resume]}{\end{enumerate}}

\newenvironment{enumerate-(A)}{\begin{enumerate}[label={\upshape (\Alph*)}, leftmargin=2pc]}{\end{enumerate}}

\newenvironment{enumerate-(A)-r}{\begin{enumerate}[label={\upshape (\Alph*)}, leftmargin=2pc,resume]}{\end{enumerate}}

\newenvironment{enumerate-(i)}{\begin{enumerate}[label={\upshape (\roman*)}, leftmargin=2pc]}{\end{enumerate}}

\newenvironment{enumerate-(i)-r}{\begin{enumerate}[label={\upshape (\roman*)}, leftmargin=2pc,resume]}{\end{enumerate}}

\newenvironment{enumerate-(I)}{\begin{enumerate}[label={\upshape (\Roman*)}, leftmargin=2pc]}{\end{enumerate}}

\newenvironment{enumerate-(I)-r}{\begin{enumerate}[label={\upshape (\Roman*)}, leftmargin=2pc,resume]}{\end{enumerate}}

\newenvironment{enumerate-(1)}{\begin{enumerate}[label={\upshape (\arabic*)}, leftmargin=2pc]}{\end{enumerate}}

\newenvironment{enumerate-(1)-r}{\begin{enumerate}[label={\upshape (\arabic*)}, leftmargin=2pc,resume]}{\end{enumerate}}

\newenvironment{itemizenew}{\begin{itemize}[leftmargin=2pc]}{\end{itemize}}




\begin{document}

%\thanks{The first and third author were partially supported by DFG-grant LU2020/1-1.}

%\subjclass[2010]{03E30, 03E40, 03E70} 

%\keywords{Class forcing, Separation, Replacement} 


\author{Philipp Schlicht}
\address{Philipp Schlicht, School of Mathematics, University of Bristol, Fry Building, Woodland Road, Bristol, BS8 1UG, UK}
\email{philipp.schlicht@bristol.ac.uk}
\urladdr{}

%\title{}
\date{\today}


\title[Lecture notes: Countable Borel equivalence relations]{Lecture notes: An introduction to countable Borel equivalence relations and actions of countable groups}

%\begin{abstract} t
%Lecture notes from the lecture Advanced mathematical logic: Model theory, Bonn October-November 2017. 
%\end{abstract} 

\maketitle


\setcounter{tocdepth}{2}
%\tableofcontents 


%\newpage 






%%%%%%%%%%%%
%%%%%%%%%%%%
\section{Overview} 

Borel measurable actions of countable groups have been intensively studied in the last decades from the viewpoints of descriptive set theory, dynamics and measured group theory \cite{MR2827853, MR2711968, MR3612001, MR3359054, MR2583950, KechrisShinko2019, Ioana2019}. 
Still there are many long-standing open questions in the focus of current research. 

We want to study some of the fundamental results in this area, both from the descriptive and measure-theoretic viewpoints. 
For instance, we want to study universal equivalence relations for actions of countable groups, Dye's theorem from 1959 that any two ergodic probability measure preserving actions of $\ZZ$ are orbit equivalent, Harrington, Kechris and Louveau's result from 1990 that there exists a least non-smooth Borel equivalence relation, and connections with logic and model theory related to the complexity of isomorphism problems for countable structures. 
 
The main reference for the first part of the course is the book \emph{Topics in orbit equivalence} by Alexander S. Kechris and Benjamin D. Miller \cite{MR2095154}. 
Most of the background in descriptive set theory can be found in \cite{MR2455198, MR1321597}. 
The book \cite{Kechris2019} gives an overview of the theory of countable Borel equivalence relations. 


%In the following, \emph{least} and \emph{universal} is meant with respect to the quasi-ordering of Borel reducibility. 
%We study the following topics in reducibility: 
%\begin{itemize} 
%\item 
%Universal equivalence relations among those induced by actions of a fixed countable group, 
%universal countable Borel equivalence relations 
%\item 
%Harrington, Kechris, Louveau (1990): there is a least non-smooth Borel equivalence relation 
%\item 
%Isomorphism problems for countable structures 
%\end{itemize} 
%
%Topics in orbit equivalence: 
%
%\begin{itemize} 
%\item 
%Dye (1959): any two ergodic p.m.p. transformations are orbit equivalent. 
%\end{itemize} 
%
%\bigskip 
%\noindent 
%{\bf Literature:} 
%Su Gao: Invariant descriptive set theory; 
%Alexander S. Kechris and Benjamin D. Miller: Topics in orbit equivalence, 
%Alexander S. Kechris: Classical descriptive set theory 







%%%%%%%%%%%%
%%%%%%%%%%%%
\section{Introduction} 

A Borel equivalence relation on a Polish space is called \emph{countable} if all its equivalence classes are countable. 
The standard example is an orbit equivalence relation of an action of a countable group by Borel automorphisms. 
%Moreover, this implication can be reversed by a theorem of Feldman and Moore (see later): any countable Borel equivalence relation is induced by a Borel action of a countable group. 
%By a theorem of Feldman and Moore, the countable Borel equivalence relations are precisely the orbit equivalence relations of Borel actions of countable groups. 

\begin{example*} 
\begin{enumerate-(a)} 
\item 
The rotation $\sigma_\alpha\colon \ZZ\rightarrow S^1$, $\sigma_\alpha(n)=\exp(\pi i n \alpha)$ by $\alpha$ on the unit sphere in $\CCC$. 
%$\ZZ$ $\curvearrowright$ $\RR$ acts by addition. 
%$\QQ$ $\curvearrowright$ $\RR$ acts by addition. 
\item 
The (left) shift action $\Gamma_2$ $\curvearrowright$ $2^{\Gamma_2}$, $(\gamma\cdot x)(\delta)=x(\gamma^{-1}(\delta))$ for $\Gamma=\FF_2$, the free group on 2 generators. 
%The action of $\mathrm{SL}_n(\ZZ)$ on the $n$-dimensional torus $\TT^n$. 
\end{enumerate-(a)} 
\end{example*} 

%\begin{itemize} 
%\item 
%Countable Borel equivalence relations and actions of countable groups, hyperfinite equivalence relations and actions of $\ZZ$ 
%\end{itemize} 

The basic problem is to describe the complexity of equivalence relations in comparison with each other. 
Suppose that $E_\Gamma=E_{\Gamma\curvearrowright X}$ and $E_\Delta=E_{\Delta\curvearrowright Y}$ are the orbit equivalence relations of group actions $\Gamma\curvearrowright X$ and $\Delta\curvearrowright Y$. 
The coarsest way to compare these two relations is the following. 

\begin{definition*} 
$E_\Gamma$ is \emph{Borel reducible} to $E_\Delta$ ($E_\Gamma\leq E_\Delta$) if there is a Borel measurable \emph{reduction} $f\colon X\rightarrow Y$ of $E_\Gamma$ to $E_\Delta$, i.e. for all $x_0,x_1\in X$, we have 
$(x_0,x_1)\in E_\Gamma\Longleftrightarrow (f(x_0),f(x_1))\in E_\Delta$. 
\end{definition*} 

Since all reductions will be assumed to be Borel measurable, we will simply speak of reductions. 
An equivalence relation is called \emph{smooth} if it is reducible to equality. 
Since these relations are arguably trivial, the interesting stuff begins beyond them. 
We will see that $\sigma_\alpha$ in the example above is non-smooth if $\alpha$ is irrational. But this is only the simplest type of non-smooth equivalence relation. 
Just beyond smoothness, it is interesting to study finer notions of equivalence: 

\begin{definition*} 
%Let $E=E_{\Gamma\curvearrowright X}$ and $F=E_{\Delta\curvearrowright Y}$. 
\begin{enumerate-(a)} 
\item 
$\Gamma\curvearrowright X$, $\Delta\curvearrowright Y$ are \emph{conjugate} if there is a group isomorphism $\pi\colon \Gamma\rightarrow \Delta$ and a Borel isomorphism $f\colon X\rightarrow Y$ with $f(\gamma x)= \pi(\gamma) f(x)$ for all $x\in X$ and $\gamma\in \Gamma$. 
\item 
$E_\Gamma$, $E_\Delta$ are \emph{orbit equivalent}\footnote{or \emph{Borel isomorphic}} (OE) if there is a Borel isomorphism $f\colon X\rightarrow Y$ with $f(\Gamma x)=\Delta f(x)$ for all $x\in X$. 
\item 
$E_\Gamma$, $E_\Delta$ are \emph{Borel bi-reducible} ($E_\Gamma\equiv E_\Delta$) if $E_\Gamma\leq E_\Delta$ and $E_\Delta\leq E_\Gamma$. 
\end{enumerate-(a)} 
\end{definition*} 

We have the implications (a) $\Rightarrow$ (b) $\Rightarrow$ (c). 
%$$(\Gamma\curvearrowright X \equiv^{\mathrm{Conj}} \Delta\curvearrowright Y) \Longrightarrow 
%(\Gamma\curvearrowright X \equiv^{\mathrm{OE}} \Delta\curvearrowright Y) \Longrightarrow 
%(E_\Gamma \equiv E_\Delta) $$ 
\emph{Rigidity} is the phenomenon that these implications can be reversed in some cases. 
The converse phenomenon is \emph{elasticity}: for instance, $\ZZ$ and $\oplus_{n\in\NN} \ZZ/2\ZZ$ have actions with the same orbit equivalence relation, so the implication (b) $\Rightarrow$ (c) cannot be reversed. 
This shows that orbit equivalence is much finer than conjugacy. 
This phenomenon may depend on the group: for instance, $\ZZ$ has many non-equivalent actions, but for some other groups there is only one ergodic measure preserving free action up to orbit equivalence. 
It is therefore important to study connections between groups and orbit equivalence relations of their actions. 






%%%%%%%%%%%%
%%%%%%%%%%%%
\section{Some background in descriptive set theory} 

In section, we review some material on standard Borel spaces, Borel sets, analytic sets and Borel measurable functions, in particular the fact that any two uncountable standard Borel spaces are isomorphic. 
%We give proofs from scratch. 
We give very direct and concise proofs from scratch, in contrast to most proofs in the literature, which refer to other theorems. 

%%%%%%%%%%%%
\subsection{Standard Borel spaces} 

\begin{definition} 
\begin{enumerate-(a)} 
\item 
A metric space $(X,d)$ is called \emph{Polish} if it is countably based and complete. 
\item 
A topological space $(X,\tau)$ is called \emph{Polish} if there's a metric $d$ on $X$ that induces $\tau$. 
\item 
A pair $(X,\sigma)$ is called a \emph{standard Borel space} if $\sigma$ is the Borel $\sigma$-algebra generated by a Polish topology on $X$.  
\end{enumerate-(a)} 
\end{definition} 

It is an easy exercise to show that Polish metric spaces are closed under countable products by taking a weighted sum over the coordinates. 

We always write $(X,d)$, $(Y,d)$, $(Z,d)$ for Polish metric spaces and often omit the metric $d$. 
We further write $d(A)$ for the \emph{diameter} $\sup_{x,y\in A} d(x,y)$ of a subset $A$ of $X$. 
Moreover, $B_\epsilon(x)$ denotes the open ball of radius $\epsilon$ around $x$. 

\begin{example} 
$\RR^n$, Baire space $\NN^\NN$,  Cantor space $2^\NN$, compact metric spaces. 
\end{example} 

Here $\NN^\NN$ caries the product topology. 
The standard metric on $\NN^\NN$ is defined by $d(x,y)=\frac{1}{2^n}$ for the least $n$ with $x(n)\neq y(n)$, where $x\neq y$. 
The basic open subsets of $\NN^\NN$ (clopen balls) are denoted $N_t=\{x\in \NN^\NN\mid t\subseteq x\}$ for $t\in \NN^{<\NN}$. 

A \emph{tree} on $\NN$ is a subset of $\NN^{<\NN}$ that is closed under initial segments. 
It is easy to see that the closed subsets of $\NN^\NN$ are exactly the sets of the form $[T]=\{x\in \NN^\NN\mid \forall n\ x{\upharpoonright}n \in T\}$ for trees $T$ on $\NN$. 
Let $|t|$ denote the length of a sequence $t\in \NN^{<\NN}$. 
Let $x{\upharpoonright}n$ denote the restriction of a sequence $x\in \NN^\NN$ to $\{0,\dots,n-1\}$. 
A tree is called \emph{pruned} if it has no end nodes. 
Let $\Lev_n(T)$ denote the $n$-th level of a tree $T$. 
%A node $t\in T$ \emph{splits} in $T$ if it has at least $2$ direct successors in $T$. Let $\spl(T)$ denote the set of splitting nodes of $T$. A tree $T$ is called \emph{perfect} if it is nonempty, pruned and every node in $T$ has some node in $\spl(T)$ above it. 

\bigskip 
We next show that any two uncountable Polish spaces are \emph{Borel isomorphic} in the following sense. 

\begin{definition} 
A \emph{Borel isomorphism} $f\colon A\rightarrow B$ between Borel subsets $A$ and $B$ of standard Borel spaces is a bijection that preserves Borel sets in both directions. 
\end{definition} 

\begin{proposition} \label{Borel isomorphism of Polish spaces} 
Any two uncountable Polish spaces are Borel isomorphic. 
\end{proposition} 

This will follow from the next four lemmas. The first three lemmas show that any uncountable Polish space $X$ is Borel isomorphic to a Borel subset of $\NN^\NN$ and vice versa. The last lemma show that this is sufficient to obtain Proposition \ref{Borel isomorphism of Polish spaces}. 

%\begin{definition} 
%\begin{enumerate-(a)} 
%\item 
An \emph{$\epsilon$-cover} of a subset $A$ of $X$ is a sequence $\vec{A}=\langle A_i\mid i<N\rangle$ with $N\in \NN\cup\{\NN\}$ whose union equals $A$ with $d(A_i)<\epsilon$ and $\overline{A_i}\subseteq A$ for all $i\in\NN$. 
%\item 
%Let $T$ be a pruned subtree of $\NN^{<\NN}$ and $\vec{\epsilon}=\langle \epsilon_i \mid i\in\NN\rangle$ be any sequence in $\RR^+$ converging to $0$. 
%A \emph{$T$-scheme} for $X$ is a family $\vec{A}=\langle A_t\mid t\in T\rangle$ of nonempty subsets of $X$ with $A_0=X$ that satisfies the following properties for all $t\in T$: \\ 
% (i) $A_t=\bigcup_{ti \in T} A_{ti}$ (disjoint), \ \ (ii) $\overline{A_{ti}} \subseteq A_t$ if $ti\in T$, and \ \ (iii) $d(A_t)<\epsilon_n$ if $|t|=n$. 
%\end{enumerate-(a)} 
%\end{definition} 
An \emph{$F_\sigma$-set} is a countable union of closed sets. 

\begin{lemma} \label{partition epsilon-covers} 
Let $\epsilon>0$. 
\begin{enumerate-(a)} 
\item \label{partition epsilon-covers - open sets} 
Any open set has an $\epsilon$-cover consisting of open sets. 
\item 
Any $F_\sigma$-set has a disjoint $\epsilon$-cover consisting of $F_\sigma$-sets. 
\end{enumerate-(a)} 
\end{lemma} 
\begin{proof} 
(a) 
Suppose that $A$ is an open subset of $X$. 
Let $\vec{A}=\langle A_i\mid i\in N\rangle$ list all open balls $B_q(x)$ with $x\in Q$, $q\in (0,\frac{\epsilon}{2})\cap \QQ$ and $\overline{B_q(x)}\subseteq A$, where $Q$ is a countable dense subset of $X$. 

We claim that $A=\bigcup_{i\in N} A_i$. To see this, take any $x\in A$. 
Since $A$ is open, there is some $q\in \QQ$ with $B_{2q}(x)\subseteq A$. 
Fix any $y\in B_q(x)\cap Q$. 
Then $x\in B_q(y)$ and $\overline{B_q(y)}\subseteq A$. 
By the latter, $B_q(y)$ appears in $\vec{A}$. 

(b) 
Suppose that $A$ is an $F_\sigma$ subset of $X$. 
Take a sequence $\vec{A}=\langle A_i\mid i\in N\rangle$ whose union equals $A$ with $A_i$ closed. 
Then $A$ is the disjoint union of the sets $B_i=A_i\setminus (A_0\cup\dots\cup A_{i-1})$. 
Since $B_i$ is of the form $C\cap D$ with $C$ open and $D$ closed, it suffices to prove the claim for sets of this form. 
By (a), there is an $\epsilon$-cover $\vec{C}=\langle C_i\mid i\in K\rangle$ of $C$ consisting of open sets. 
Let $\vec{D}=\langle D_i\mid i\in K\rangle$ with $D_i=(C_i\setminus (C_0\cup\dots\cup C_{i-1}))\cap D$. 
Then $\vec{D}$ is a disjoint $\epsilon$-cover of $C\cap D$ consisting of $F_\sigma$-sets. 
\end{proof} 

\begin{lemma} \label{Polish space as continuous image} 
Any Polish space $X$ is Borel isomorphic to a Borel subset of $\NN^\NN$. 
In fact, there is a closed subset $A$ of $\NN^\NN$ and a continuous Borel isomorphism $f\colon A\rightarrow X$.\footnote{In fact, any continuous bijection is a Borel isomorphism, but we don't prove this here. } 
\end{lemma} 
\begin{proof} 
Let $\vec{\epsilon}=\langle \epsilon_i \mid i\in\NN\rangle$ be a sequence in $\RR^+$ converging to $0$. 
By iterative applications of Lemma \ref{partition epsilon-covers}, we obtain a tree $T\subseteq \NN^{<\NN}$ without end nodes and a family $\vec{A}=\langle A_t\mid t\in T\rangle$ of nonempty $F_\sigma$-subsets of $X$ with $A_0=X$ that satisfies the following properties for all $t\in T$: \\ 
\indent 
(a) $A_t=\bigcup_{ti \in T} A_{ti}$ (disjoint), \ \ (b) $\overline{A_{ti}} \subseteq A_t$ if $ti\in T$, and \ \ (c) $d(A_t)<\epsilon_n$ if $|t|=n$. \\ 
%\begin{enumerate-(a)} 
%\item 
%$A_t=\bigcup_{ti \in T} A_{ti}$ (disjoint), 
%\item 
%$\overline{A_{ti}} \subseteq A_t$ and 
%\item 
%$d(A_t)<\epsilon_n$ if $|t|=n$. 
%\end{enumerate-(a)} 
%Note that $A:=[T]$ is a closed subset of $\NN^{<\NN}$. 
We then define $f\colon [T]\rightarrow X$ by choosing $f(x)$ as the unique element of $\bigcap_{n\in\NN} A_{x\upharpoonright n}$. 
The latter is nonempty by (b), (c) and completeness of $X$, and has a unique element by (c). 

\begin{claim*} 
$f$ is bijective. 
\end{claim*} 
\begin{proof} 
It follows from (a) that $f$ is injective. 
To see that $f$ is surjective, take any $x\in X$. 
By (a), there is for each $n\in\NN$ a unique $t_n\in T$ of length $n$ with $x\in A_{t_n}$. 
By (b), $t_i\subseteq t_j$ for $i\leq j$. 
Hence $f(y)=x$ for $y=\bigcup_{i\in\NN} t_i$. 
\end{proof} 

\begin{claim*} 
$f$ is continuous. 
\end{claim*} 
\begin{proof} 
If $U$ is open and $f(x)\in U$, then $A_{x{\upharpoonright}n}\subseteq U$ for sufficiently large $n$ by (c). 
\end{proof} 

\begin{claim*} 
$f$ is a Borel isomorphism. 
\end{claim*} 
\begin{proof} 
It suffices that the images of Borel sets are Borel. 
This holds since $f$ maps each basic open set $N_t$ onto $A_t$ and $f$ is bijective. 
\end{proof} 
Thus $f$ is as required. 
\end{proof} 

Using the previous lemma, one can see that any nonempty Polish space $X$ is the image of $\NN^\NN$ under a continuous function. 
By the lemma, it suffices to show this in the case that $X=[T]$ is a closed subset of $\NN^\NN$. In this case, map each $x\in \NN^\NN$ to the leftmost $y\in [T]$ with $x{\upharpoonright}n \subseteq y$, where $n$ is maximal with $x{\upharpoonright}n\in T$. 

\begin{lemma} 
If $X$ is an uncountable Polish space, then $\NN^\NN$ is Borel isomorphic to a Borel subset of $X$. 
\end{lemma} 
\begin{proof} 
Fix a sequence $\vec{\epsilon}=\langle \epsilon_i \mid i\in\NN\rangle$ in $\RR^+$ converging to $0$. 
It is easy to construct a family $\vec{U}=\langle U_t\mid t\in \NN^{<\NN}\rangle$ of uncountable open subsets of $X$ such that: 
(a) $\overline{U_t}\subseteq U_s$ for all $s\subseteq t$, \\ 
(b) $U_s\cap U_t=\emptyset$ for all $s\neq t$ with $|s|=|t|$, and \ \  (c) $d(U_t)<\epsilon_n$ if $|t|=n$. 
%$\lim_{i\rightarrow \infty} d(U_{ti})=0$ for all $t\in \NN^{<\NN}$. 
% and $\lim_{i\rightarrow \infty} x_i\in Q$ for any sequence $\vec{x}=\langle x_i\mid i\in\NN\rangle$ with $x_i\in U_{ti}$ for all $i\in\NN$. 

Define $f\colon \NN^\NN\rightarrow X$ by choosing $f(x)$ as the unique element of $\bigcap_{n\in \NN} U_{x{\upharpoonright}n}$. 
It is then easy to see that $f$ is a homeomorphism onto its image. 
To see that $f(\NN^\NN)$ is Borel, let $U_n=\bigcup_{|t|=n} U_t$. We then have $f(\NN^\NN)=\bigcap_{n\in\NN} U_n$. 
\end{proof} 

\begin{lemma} 
Suppose that $X$ is Borel isomorphism to a Borel subset of $Y$ and vice versa. 
Then $X$ and $Y$ are Borel isomorphic. 
\end{lemma} 
\begin{proof} 
Suppose that $f\colon X\rightarrow Y$ and $g\colon Y\rightarrow X$ are Borel isomorphisms onto Borel sets. 
We call a pair $(A,B)$ of Borel subsets of $X$ and $Y$ \emph{nice} if $f\colon A\rightarrow B$ and $g\colon B\rightarrow A$ are well-defined.\footnote{This is shorthand for $\ran(f{\upharpoonright}A)\subseteq B$ and $\ran(g{\upharpoonright}B)\subseteq A$. } 

\begin{claim*} 
For any nice pair $(A,B)$, there is a nice pair $(A_*,B_*)$ such that $A_*\subseteq A\cap g(B)$, $B_*\subseteq B\cap f(A)$ and the Borel sets $A\setminus A_*$, $B\setminus B_*$ are Borel isomorphic. 
\end{claim*} 
\begin{proof} 
To see this, let $A^*=A\setminus g(B)$, $B^*=B\setminus f(A)$. 
Since $A^*\cap g(B^*)=B^*\cap f(A^*)=\emptyset$, the maps $f\colon A^*\rightarrow f(A^*)$ and $g\colon B^*\rightarrow g(B^*)$ witness that 
$A^*\cup g(B^*)$ and $B^*\cup f(A^*)$ are Borel isomorphic. 
Thus $A_*=A\setminus (A^*\cup g(B^*))=g(B) \setminus g(B^*)$ and $B_*=B\setminus (B^*\cup f(A^*))=f(A)\setminus f(A^*)$ are as required. 
\end{proof} 

Let $A_0=X$ and $B_0=Y$. By iterative applications of the previous claim, we obtain decreasing sequences $\vec{A}=\langle A_n\mid n\in\NN\rangle$ and $\vec{B}=\langle B_n\mid n\in\NN\rangle$ of Borel sets such that $(A_n, B_n)$ is nice for all $n\in\NN$. Let $A=\bigcap_{n\in\NN} A_n$ and $B=\bigcap_{n\in\NN}B_n$. 

It suffices to show that $A$ and $B$ are Borel isomorphic via $f$. 
To see this, note that $f\colon A\rightarrow B$ is well-defined, since $f\colon A_n\rightarrow B_n$ is well-defined for all $n\in\NN$. 
Moreover, $f\colon A\rightarrow B$ is surjective, since $B_{n+1}\subseteq f(A_n)$ for all $n\in \NN$. 
\end{proof} 





%%%%%%%%%%%%
\subsection{Analytic sets} 

Let $p\colon X\times Y\rightarrow X$ denote the first projection. 

\begin{definition} 
A subset $A$ of $X$ is called \emph{analytic} if $A=p(B)$ for some Borel subset $B$ of $X\times Y$. 
\end{definition} 

By Proposition \ref{Borel isomorphism of Polish spaces}, these are precisely the images of analytic subsets of $\NN^\NN$ under Borel isomorphisms. 
Hence we can work in $\NN^\NN$ to study their properties. 

To do this, we will go through a slight detour. 
We call a subset $A$ of $\NN^\NN$ \emph{*analytic} if $A=p[C]$ for some closed subset of $\NN^\NN\times \NN^\NN$, equivalently $A=p[T]=\{x\mid \exists y\ (x,y)\in [T]\}$ for some subtree $T$ of $(\NN\times\NN)^{<\NN}$. 

\begin{problem} \label{characterization of analytic sets} 
\begin{enumerate-(a)} 
\item \label{characterization of analytic sets - closure} 
The class of *analytic sets is closed under countable unions and intersections. It follows that every Borel subset of $\NN^\NN$ is *analytic. 
\emph{(Hint: first show that the intersection of two *analytic sets is again *analytic by merging trees projecting to the sets.) } 
\item 
Every analytic subsets of $\NN^\NN$ is *analytic. 
\emph{(Hint: this is an easy consequence of (a).) } 
\end{enumerate-(a)} 
\end{problem} 

The previous result also holds for finite products of $\NN^\NN$, by observing that the proof still works or noting that they are homeomorphic to $\NN^\NN$. 

\begin{lemma} \label{analytic sets as continuous images of N^N} 
Every nonempty analytic subset of $X$ equals $f(\NN^\NN)$ for some continuous $f\colon \NN^\NN\rightarrow X$. 
\end{lemma} 
\begin{proof} 
Since every nonempty Polish space is a continuous image of $\NN^\NN$ by the remark after Lemma \ref{Polish space as continuous image}, it suffices to find some continuous $f\colon Y\rightarrow X$ with $f(Y)=X$. 

Take an analytic subset $A$ of $X$. By Lemma \ref{Polish space as continuous image}, there is a closed subset $Z$ of $\NN^\NN$ and a continuous Borel isomorphism $g\colon Z\rightarrow A$. 
Since $g$ is continuous, it suffices to find a continuous $f\colon Y\rightarrow \NN^\NN$ with $f(Y)=B:=g^{-1}(A)$. 

Since analytic sets are preserved under Borel isomorphisms, $B$ is an analytic subset of $Z$ and thus of $\NN^\NN$. 
Let $C$ be a Borel subset of $\NN^\NN\times \NN^\NN$ with $B=p(C)$. 
Since Borel subsets of $\NN^\NN\times \NN^\NN$ are *analytic by Problem \ref{characterization of analytic sets} \ref{characterization of analytic sets - closure}, there is a closed subset $D$ of $(\NN^\NN)^3$ with $B=p(D)$. 
Then $p\colon D\rightarrow \NN^\NN$ is as required. 
\end{proof} 

Subsets $A$, $B$ of $X$ are called \emph{Borel separated} if there is a Borel set $C$ with $A\subseteq C$ and $B\subseteq X\setminus C$. 

\begin{proposition} \label{separation of analytic sets} 
Any two disjoint analytic subsets $A$, $B$ of $X$ are Borel separated. 
\end{proposition} 
\begin{proof} 
It suffices to show this for $X=\NN^\NN$, and by Problem \ref{characterization of analytic sets} $A$ and $B$ are then *analytic. 
Let $f\colon [S]\rightarrow A$ and $g\colon [T]\rightarrow B$ be continuous surjections, where $S$ and $T$ are pruned subtrees of $\NN^{<\NN}$.\footnote{One can assume that $[S]=[T]=\NN^\NN$ to simplify the notation at the expense of using Lemma \ref{analytic sets as continuous images of N^N}. } 

Suppose that $A$, $B$ are not Borel separated. 
We then construct increasing sequences $\vec{s}=\langle s_i\mid i\in\NN\rangle$ and $\vec{t}=\langle t_i\mid i\in\NN\rangle$ in $\NN^{<\NN}$ such that for all $i\in\NN$, $f(N_{s_i})$ and $g(N_{t_i})$ are not Borel separated as follows. 

Let $s_0=t_0=\emptyset$. 
Given $s_i$ and $t_i$, there are $k,m\in \NN$ with $s_i k\in S$ and $t_i m\in T$ such that $f(N_{s_i k})$ and $g(N_{t_i m})$ are not Borel separated. 
If each of these pairs were Borel separated by some $B_{k,m}$, then $C_k=\bigcap_{t_i m\in T} B_{k,m}$ separates $f(N_{s_i k})$ and $g(N_{t_i})$ for each $k\in\NN$ with $s_i k\in S$, and hence $C=\bigcup_{s_i k\in S} C_k$ separates $f(N_{s_i})$ and $g(N_{t_i})$, contradicting our assumption. 
Let $s_{i+1}=s_i k$ and $t_{i+1}= t_i m$. 

Let $x=\bigcup_{i\in\NN} s_i\in [S]$ and $y=\bigcup_{i\in\NN} t_i\in [T]$. 
Since $A$ and $B$ are disjoint, $f(x)\neq g(y)$. 
Let $U$, $V$ be disjoint open subsets of $X$ with $f(x)\in U$ and $g(y)\in V$. 
By continuity of $f$ and $g$, $f(N_{x{\upharpoonright}n})\subseteq U$ and $g(N_{y{\upharpoonright}n})\subseteq V$ for sufficiently large $n$, so $f(N_{x{\upharpoonright}n})$ and $g(N_{y{\upharpoonright}n})$ are Borel separated. 
But this contradicts the construction. 
\end{proof} 




%%%%%%%%%%%%
\subsection{Borel measurable functions} 

A function $f\colon X\rightarrow Y$ between standard Borel spaces is called \emph{Borel measurable} if $f$-preimages of Borel sets are Borel. 
%A function $f\colon X\rightarrow Y$ is called \emph{Borel measurable} if for every Borel subset $C$ of $Y$, $f^{-1}(C)$ is a Borel subset of $X$. 

\begin{lemma} 
The following are equivalent for a function $f\colon X\rightarrow Y$: 
\begin{enumerate-(a)} 
\item 
$f$ is Borel measurable. 
\item 
The graph $G(f)$ of $f$ is a Borel subset of $X\times Y$. 
\end{enumerate-(a)} 
\end{lemma} 
\begin{proof} 
(a) $\Rightarrow$ (b): $f(x)=y$ $\Longleftrightarrow$ $x\in \bigcap_{k\in \NN} f^{-1}(N_{y{\upharpoonright}k})$ $\Longleftrightarrow$ $(x,y)\in \bigcap_{k\in\NN} \bigcup_{t\in \NN^k} f^{-1}(N_t) \times N_t$. 

(b) $\Rightarrow$ (a): 
It suffices to show that $f^{-1}(U)$ is Borel for any open subset of $Y$. 
By Proposition \ref{separation of analytic sets}, it is sufficient to show that both $f^{-1}(U)$ and its complement in $X$ are analytic. 
These claims follows from the equivalences $x\in f^{-1}(U) \Longleftrightarrow \exists y\in Y\ (x,y)\in G(f) \Longleftrightarrow \forall y\in Y\ ((x,y)\in G(f)\Rightarrow y\in U)$. 
\end{proof} 






\bibliographystyle{alpha}
\bibliography{references}
  
\end{document}
