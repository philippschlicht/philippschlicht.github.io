
\documentclass[10pt]{amsart}
\usepackage{amsmath,amssymb,amscd,amsthm,amsfonts,amstext,amsbsy,mathrsfs,hyperref,upgreek,mathtools,stmaryrd,enumitem,bbm}
%\usepackage{MnSymbol}
%\usepackage[shadow]{todonotes}
% \usepackage{citeref, breakcites}

\usepackage[
disable, 
textsize=footnotesize,color=blue!40, bordercolor=white]{todonotes}

\usepackage%[scale=0.682]
[a4paper, margin=1.2in]{geometry}

\usepackage{xcolor}
\definecolor{blue}{rgb}{0,0,1}
\hypersetup{
	unicode=true,
	colorlinks=true,
	citecolor=blue,
	linkcolor=blue,
	anchorcolor=blue
}

\hypersetup{colorlinks=true}

% other fonts
\newcommand\V{\mathsf{V}}
\newcommand{\Minimal}{\MM[\![G]\!]}
\newcommand\bB{\boldsymbol{{\rm B}}}
\newcommand\bN{\boldsymbol{{\rm N}}}
\renewcommand{\L}{\mathcal{L}}
%\newcommand{\C}{\mathcal{C}}
\newcommand{\D}{\mathcal{D}}
\newcommand{\E}{\mathcal{E}}
\newcommand{\M}{\mathcal{M}}
\newcommand{\T}{\mathcal{T}}
\newcommand{\EE}{\mathbb{E}}                                                       
\newcommand{\OO}{\mathbb{O}}
\newcommand{\PP}{\mathbb{P}}
\newcommand{\TT}{\mathbb{T}}
\newcommand{\HH}{\mathbb{H}}
\newcommand{\FF}{\mathbb{F}}
\newcommand{\Ss}{\mathbb{S}}
\newcommand{\HHH}[1]{\mathrm{H}({#1})}
\newcommand{\Hes}{\mathfrak{H}}
\newcommand{\pre}[2]{{}^{#1} #2}
\newcommand{\seq}[2]{\langle #1 \mid #2 \rangle}
\newcommand{\set}[2]{\{ #1 \mid #2 \}}
\newcommand{\Nbhd}{\mathbf{N}}
\newcommand{\Bor}{\mathsf{Bor}}
\newcommand{\pI}{\mathrm{I}}
\newcommand{\pII}{\mathrm{II}}
%\newcommand{\K}[1]{\boldsymbol{K}_{#1}}
\newcommand{\anf}[1]{{\text{``}\hspace{0.3ex}{#1}\hspace{0.01ex}\text{''}}}
\newcommand{\op}{\mathsf{op}}
\newcommand{\rnk}{\mathsf{rnk}}

% operators
\newcommand{\id}{\operatorname{id}}
\newcommand{\cof}{\operatorname{cof}}
\newcommand{\otp}[1]{{{\rm{otp}}\left(#1\right)}}
\newcommand{\supp}[1]{{{\rm{supp}}(#1)}}
\newcommand{\proj}{\operatorname{p}}
\newcommand{\pow}{\mathscr{P}}
\newcommand{\Trcl}{\operatorname{Trcl}}
\newcommand{\p}{\operatorname{p}}
\newcommand{\leng}{\operatorname{lh}}
\newcommand{\range}{\operatorname{range}}
\newcommand{\rank}{\operatorname{rank}}
\newcommand{\Col}{\operatorname{Col}}
\newcommand{\ra}{\rightarrow}
\newcommand{\lr}{\leftrightarrow}
\newcommand{\Llr}{\Longleftrightarrow}
\newcommand{\tc}{\operatorname{tc}}
\newcommand{\depth}{\operatorname{d}}
%\newcommand{\Add}{\operatorname{Add}}
\newcommand{\ul}{\ulcorner}
\newcommand{\ur}{\urcorner}
\newcommand{\lh}{\operatorname{lh}}
\newcommand{\llangle}{\langle\!\langle}
\newcommand{\rrangle}{\rangle\!\rangle}
\newcommand{\vn}{\vec{n}}
\newcommand{\one}{\mathbbm1}
\newcommand{\lkk}{\L_{\kappa,\kappa}}
\newcommand{\gbr}[1]{{\ulcorner{#1}\urcorner}}
\newcommand{\Def}{\mathrm{Def}}
% spaces
\newcommand{\On}{{\mathrm{Ord}}}
%\newcommand{\Card}{{\sf Card}}
\newcommand{\Lim}{{\mathrm{Lim}}}
\newcommand{\Fm}{{\mathrm{Fml}}}
%\newcommand{\Sym}{{\rm Sym}}
\newcommand{\push}{{\mathrm{push}}}


% axioms
\newcommand{\AD}{{\sf AD}}
\newcommand{\ZFC}{{\sf ZFC}}
\newcommand{\ZF}{{\sf ZF}}
\newcommand{\AC}{{\sf AC}}
\newcommand{\DC}{{\sf DC}}
\newcommand{\KM}{{\sf KM}}
\newcommand{\GB}{{\sf GB}}
\newcommand{\GBC}{{\sf GBC}}



% others
\newcommand{\Linf}{\L_{\omega_1 \omega}}

%Luecke
\newcommand{\map}[3]{{#1}:{#2}\longrightarrow{#3}}
\newcommand{\Map}[5]{{#1}:{#2}\longrightarrow{#3};~{#4}\longmapsto{#5}}
\newcommand{\pmap}[4]{{#1}:{#2}\xrightarrow{#4}{#3}}
\newcommand{\Set}[2]{\{{#1}~\vert~{#2}\}}
\newcommand{\ran}[1]{{{\rm{ran}}(#1)}}
\newcommand{\dom}[1]{{{\rm{dom}}(#1)}}
\newcommand{\length}[1]{{{\rm{lh}}(#1)}}
\newcommand{\VV}{{\rm{V}}}
\newcommand{\WW}{{\rm{W}}}
\newcommand{\LL}{{\rm{L}}}
\newcommand{\Add}[2]{{\rm{Add}}({#1},{#2})}
\newcommand{\goedel}[2]{{\prec}{#1},{#2}{\succ}}
\newcommand{\ot}{ot}




\newcommand{\ZZ}{\mathbb{Z}} 
\newcommand{\QQ}{\mathbb{Q}} 
\newcommand{\RR}{\mathbb{R}} 
\newcommand{\GG}{\mathbb{G}}


\newcommand{\CCC}{\mathbb{C}} 
\newcommand{\NNN}{\mathbb{N}} 

\newcommand{\MM}{\mathcal{M}}
\newcommand{\NN}{\mathcal{N}} 
\newcommand{\KK}{\mathcal{K}} 
\renewcommand{\AA}{\mathcal{A}}
\newcommand{\BB}{\mathcal{B}}
\newcommand{\CC}{\mathcal{C}} 
\newcommand{\DD}{\mathcal{D}} 

%\newcommand{\MM}{\mathbb{M}}
%\newcommand{\NN}{\mathbb{N}} 
%\newcommand{\KK}{\mathbb{K}} 
%\renewcommand{\AA}{\mathbb{A}}
%\newcommand{\BB}{\mathbb{B}}
%\newcommand{\CC}{\mathbb{C}} 
%\newcommand{\DD}{\mathbb{D}} 




\newcommand{\K}{\mathcal{K}} 
\newcommand{\Th}{\mathrm{Th}} 

\newcommand{\DLO}{\mathrm{DLO}} 
\newcommand{\ACF}{\mathrm{ACF}} 

\newcommand{\lex}{\mathrm{lex}} 

\newcommand{\cC}{\mathcal{C}} 
\newcommand{\Mod}{\mathrm{Mod}} 

\newcommand{\acl}{\mathrm{acl}} 
\newcommand{\tp}{\mathrm{tp}} 
\newcommand{\Aut}{\mathrm{Aut}} 
\newcommand{\Sym}{\mathrm{Sym}} 




%\setcounter{secnumdepth}{2}
% theorems
\newtheorem{theorem}{Theorem}[subsection]
\newtheorem*{theorem*}{Theorem}
\newtheorem{lemma}[theorem]{Lemma}
\newtheorem{corollary}[theorem]{Corollary}
\newtheorem{proposition}[theorem]{Proposition}
\newtheorem{conjecture}[theorem]{Conjecture}
\newtheorem{question}[theorem]{Question}
\newtheorem{problem}[theorem]{Problem}
\newtheorem{observation}[theorem]{Observation}
\newtheorem{claim}[theorem]{Claim}
\newtheorem*{claim*}{Claim}
\newtheorem*{subclaim*}{Subclaim}
\newtheorem*{Vaughtconj}{Vaught's Conjecture}
\theoremstyle{definition}
%\newtheorem{claim}{Claim}[theorem]
\newtheorem{definition}[theorem]{Definition}
\newtheorem*{notation}{Notation}
\newtheorem{fact}[theorem]{Fact}
\newtheorem{example}[theorem]{Example}
\newtheorem*{example*}{Example}
\theoremstyle{remark}
\newtheorem{remark}[theorem]{Remark}



% enumeration

\newenvironment{enumerate-(a)}{\begin{enumerate}[label={\upshape (\alph*)}, leftmargin=2pc]}{\end{enumerate}}

\newenvironment{enumerate-(a)-r}{\begin{enumerate}[label={\upshape (\alph*)}, leftmargin=2pc,resume]}{\end{enumerate}}

\newenvironment{enumerate-(A)}{\begin{enumerate}[label={\upshape (\Alph*)}, leftmargin=2pc]}{\end{enumerate}}

\newenvironment{enumerate-(A)-r}{\begin{enumerate}[label={\upshape (\Alph*)}, leftmargin=2pc,resume]}{\end{enumerate}}

\newenvironment{enumerate-(i)}{\begin{enumerate}[label={\upshape (\roman*)}, leftmargin=2pc]}{\end{enumerate}}

\newenvironment{enumerate-(i)-r}{\begin{enumerate}[label={\upshape (\roman*)}, leftmargin=2pc,resume]}{\end{enumerate}}

\newenvironment{enumerate-(I)}{\begin{enumerate}[label={\upshape (\Roman*)}, leftmargin=2pc]}{\end{enumerate}}

\newenvironment{enumerate-(I)-r}{\begin{enumerate}[label={\upshape (\Roman*)}, leftmargin=2pc,resume]}{\end{enumerate}}

\newenvironment{enumerate-(1)}{\begin{enumerate}[label={\upshape (\arabic*)}, leftmargin=2pc]}{\end{enumerate}}

\newenvironment{enumerate-(1)-r}{\begin{enumerate}[label={\upshape (\arabic*)}, leftmargin=2pc,resume]}{\end{enumerate}}

\newenvironment{itemizenew}{\begin{itemize}[leftmargin=2pc]}{\end{itemize}}




\begin{document}

%\thanks{The first and third author were partially supported by DFG-grant LU2020/1-1.}

%\subjclass[2010]{03E30, 03E40, 03E70} 

%\keywords{Class forcing, Separation, Replacement} 


\author{Philipp Schlicht}
\address{Philipp Schlicht, Math. Institut, Universit\"at Bonn, 
Endenicher Allee 60, 53115 Bonn, Germany}
\email{schlicht@math.uni-bonn.de}
\urladdr{}

%\title{}
\date{\today}


\title{Lecture notes: An introduction to model theory}

%\begin{abstract} t
%Lecture notes from the lecture Advanced mathematical logic: Model theory, Bonn October-November 2017. 
%\end{abstract} 

\maketitle


\setcounter{tocdepth}{2}
\tableofcontents 


Model theory studies classes of structures and their abstract properties, in particular the relationship between the properties of theories and properties of the classes of their models. A theory is a set of sentences in a language and all languages are assumed to be first-order. \footnote{These are lecture notes from the course 'Advanced mathematical logic: model theory' at the University of Bonn in October and November 2017. } 
%A (first-order) \emph{theory} is simply a set of sentences in a (first-order) language. 

\begin{example*} 
The language $\L_{LO}$ is that of (strict) linear orders and $LO=\{\forall x\ x\not< x, \forall x,y,z\ (x<y\wedge y<z\rightarrow x<z), \forall x,y\ (x<y\vee x=y\vee y<x)\}$ is the theory of (strict) linear orders. 
\end{example*} 

If $\kappa$ is an infinite cardinal, a theory is called \emph{$\kappa$-categorical} if it has exactly one model of size $\kappa$ up to isomorphism. 

\begin{example*} 
\begin{enumerate-(a)} 
\item 
The theory $DLO$ of dense linear orders without end points is $\aleph_0$-categorical, but not $\kappa$-categorical for any uncountable cardinal $\kappa$. 
\item 
The theory of vector spaces over a fixed finite field is $\kappa$-categorical for all infinite cardinals $\kappa$. 
\item 
Then theory $ACF_p$ of algebraically closed fields of characteristic $p$ for any prime $p$ or $p=0$ is not $\aleph_0$-categeorical, but $\kappa$-categorical for every uncountable cardinal $\kappa$. 
\end{enumerate-(a)} 
\end{example*} 

We will study the following problems, among others. 

\begin{itemize} 
\item 
How can we prove that a theory is $\kappa$-categorical? 
\item 
For which infinite cardinals can a theory be $\kappa$-categorical? 
\end{itemize} 

We will see several techniques for the first problem. 
The second problem is solved by the following important theorem. 
% which we will prove towards the end of the course. 

\begin{theorem*} [Morley] 
If a theory in a countable language is $\kappa$-categorical for some uncountable cardinal, then it is $\kappa$-categorical for every uncountable cardinal. 
\end{theorem*} 

We will further study various related problems, including the following.  
\begin{itemize} 
\item 
How can we prove that a theory is decidable? 
\item 
What can we say about the definable subsets of a model of a given theory? 
\end{itemize} 

A theory is called \emph{decidable} if there is an algorithm that decides for every formula $\varphi$ whether or not it is provable in the theory. A variant of the second problem is for which theories every definable subset is already definable by a quantifier-free formula. We will also see various applications to algebra, for instance Hilbert's Nullstellensatz. 

I mostly follow Chapters 1-4 in the book 'A course in model theory' by Tent and Ziegler \cite{MR2908005}, but in a slightly different order. I also very much recommend the book 'Model theory: an introduction' by Marker \cite{MR1924282} for further reading. For more information, I recommend the books 'Model theory' and 'A shorter model theory' by Hodges \cite{MR1221741, MR1462612}. For further reading in algebra, see for example the book 'Fields and Galois theory' by Milne.\footnote{See http://www.jmilne.org/math/CourseNotes/FT.pdf} 

I would like to thank Andreas Lietz for proofreading these notes and the participants of the lecture for asking interesting questions. 




%%%%%%%%%%%%
%%%%%%%%%%%%
\section{Basic notions of model theory} 
%Structures, languages, elementary substructures and compactness}

We begin by introducing various notions from mathematical logic such as structures, languages, theories etc. 

%%%%%%%%%%%%
\subsection{Structures, languages and theories} 

\begin{definition} 
A \emph{structure} is a pair $(M,\vec{R})$, where $\vec{R}$ is a sequence of elements of $M$, subsets of $M^n$ and functions from $M^n$ to $M$ for $n\in\NNN$. 
\end{definition} 

\begin{example} 
\begin{enumerate-(a)} 
\item 
A ring $(R,0,1,+,\cdot)$ 
\item 
A group $(G,\cdot,{}^{-1})$ 
\item 
The natural numbers $(\NNN,0,S,+,\cdot,<)$ 
\item 
The field of rationals $(\QQ,0,1,+,-,\cdot)$ 
\end{enumerate-(a)} 
\end{example} 

A \emph{language} (also known as \emph{alphabet} or \emph{signature}) is a set of constant, relation and function symbols. Each relation and function symbol has a fixed arity in $\NNN$. 

\begin{example} 
The languages 
\begin{enumerate-(a)} 
\item 
%The empty language 
$L_{\emptyset}=\emptyset$ 
\item 
%The language 
$\L_{AG}=\{0,+,-\}$ of abelian groups 
\item 
%The language 
$\L_R=\{0,1,+,-,\cdot\}$ of rings and fields 
\item 
%The language 
$\L_G=\{1,\cdot,{}^{-1}\}$ of groups 
\item 
%The language 
$\L_{LO}=\{<\}$ of (strict) linear orders 
\item 
%The language 
$\L_{OF}=\L_R\cup \L_{LO}$ of ordered fields 
\item 
%The language 
$\L_{\NNN}=\{0,S,+,\cdot,<\}$ of the natural numbers 
\item 
%The language 
$\L_{\in}=\{\in\}$ of set theory 
\end{enumerate-(a)} 
\end{example} 

If $R$ is a binary relation symbol, we also write $xRy$ for $(x,y)\in R$. 

\begin{definition} 
If $\L$ is a language, an \emph{$\L$-structure} is a structure $(M,\langle R^M\mid R\in \L \rangle)$, where 
\begin{enumerate-(a)} 
\item 
$c^M\in M$ if $c\in\L$ is a constant symbol 
\item 
$R^M\subseteq M^n$ if $R\in\L$ is a relation symbol with arity $n$ 
\item 
$f^M\colon M^n\rightarrow M$ if $f\in\L$ is a function symbol with arity $n$ 
\end{enumerate-(a)} 
\end{definition} 

\begin{definition} 
Suppose that $\MM=(M,\langle R^M\mid R\in\L\rangle)$ and $\NN=(N,\langle R^N\mid R\in\L\rangle)$ are $\L$-structures and $h\colon M\rightarrow N$. 
\begin{enumerate-(a)} 
\item 
$h$ is a \emph{homomorphism} if for all $n$ and all $a_0,\dots,a_{n-1}\in M$ 
\begin{enumerate-(i)} 
\item 
$h(c^M)=c^N$ for all constant symbols $c\in\LL$ 
\item 
If $R^M(a_0,\dots,a_{n-1})$, then $R^N(h(a_0),\dots,h(a_{n-1}))$ for all relation symbols $R\in\L$ of arity $n$ 
\item 
$h(f^M(a_0,\dots,a_{n-1}))=f^N(h(a_0),\dots,h(a_{n-1}))$ for all function symbols $f\in\L$ of arity $n$ 
\end{enumerate-(i)} 
\item 
$h$ is an \emph{embedding} if it is an injective homomorphism and $R^M(a_0,\dots,a_{n-1})$ if and only if $R^N(h(a_0),\dots,h(a_{n-1}))$ for all relation symbols $R\in\L$ of arity $n$. 
\item 
$h$ is an \emph{isomorphism} if it is a surjective embedding. 
\item 
$h$ is an \emph{automorphism} if it is an isomorphism and $\MM=\NN$. 
\end{enumerate-(a)} 
\end{definition} 

If there is an isomorphism between $\L$-structures $\MM$ and $\NN$, we say that $\MM$ and $\NN$ are isomorphic ($\MM\cong\NN$). 

\begin{definition} 
Suppose that $\MM=(M, \langle R^M\mid R\in\LL\rangle)$ and $\NN=(N, \langle R^N\mid R\in\LL\rangle)$ are $\L$-structures. 
\begin{enumerate-(a)} 
\item 
$\MM$ is a \emph{substructure} of $\NN$ if $M\subseteq N$ and $\id\colon M\rightarrow N$ is an embedding from $\MM$ to $\NN$. 
\item 
$\NN$ is an \emph{extension} of $\MM$ if $\MM$ is a substructure of $\NN$. 
\end{enumerate-(a)} 
\end{definition} 

\begin{definition} 
Suppose that $\K \subseteq \L$ are languages and $\MM=(M, \langle R^M\mid R\in\LL\rangle)$ is an $\L$-structure. 
\begin{enumerate-(a)} 
\item 
$M{\upharpoonright}\K=(M,\langle R^M\mid R\in\K\rangle)$ is called the \emph{reduct} of $M$ to $\K$. 
\item 
$M$ is called an \emph{expansion} of $M{\upharpoonright}\K$. 
\end{enumerate-(a)} 
\end{definition} 
\begin{example} 
\begin{enumerate-(a)} 
\item 
$\MM=(\RR,0,1,+,\cdot,<)$ is an $\LL_{OF}$-structure and $\MM{\upharpoonright}\L_R=(\RR,0,1,+,\cdot)$ is an $\L_R$-structure. 
\item 
Suppose that $\MM=(M,\vec{R})$ is an $\L$-structure and $A\subseteq M$. Then $\MM_A:=(M,\vec{R}\cup \langle a\mid a\in\LL\rangle)$ is an $\L_A$-structure, where $\L_A=L\cup A$. 
\end{enumerate-(a)} 
\end{example} 

For any $\L$-structure $\MM$, $\L$-formula $\varphi(x_0,\dots,x_{n-1})$, $a_0,\dots, a_{n-1}\in M$ and any assignment $A$ of values in $M$ to free variables that occur in $\varphi$, the validity $\MM\models \varphi(a_0,\dots,a_{n-1})[A]$ is defined by induction in the usual way. 

We fix a sequence of variables $\langle v_i\mid i\in\NNN\rangle$. The logical symbols are the equality symbol  $=$, the negation symbol $\neg$, the conjunction symbols $\wedge$, the disjunction symbol $\vee$, the existential quantifier $\exists$, the universal quantifier $\forall$, brackets $($, $)$ and the true and false statements $\top$ and $\bot$. We define $\L$-terms and $\L$-formulas in the usual way by induction from the variables, logical symbols and symbols of the language. 
An \emph{$\L$-sentence} is an $\L$-formula without free variables. 
Let $\mathrm{Form}_\L$ denote the set of $\L$-formulas and $\mathrm{Sent}_\L$ the set of $\L$-sentences. 

A formula is \emph{basic} if it is atomic or the negation of an atomic formula. A formula is in \emph{negation normal form} if it is built up from basic formulas by using $\wedge$, $\vee$, $\exists$ and $\forall$. Two formulas are \emph{logically equivalent} if they are equivalent in all models and for all assignments of free variables. A formula is called \emph{universal} (\emph{existential}) if it is logically equivalent to a formula in negation normal form that doesn't contain existential (universal) quantifiers. 

\begin{definition} 
The equality axioms are 
\begin{enumerate-(a)} 
\item 
(reflexivity) $\forall x\ x=x$ 
\item 
(symmetry) 
$\forall x,y\ (x=y\rightarrow y=x)$ 
\item 
(transitivity) 
$\forall x,y,z\ (x=y\wedge y=z\rightarrow x=z)$ 
\item 
(congruence for relations) 
$\forall x_0,\dots,x_{n-1},y_0,\dots,y_{n-1}\ ((x_0=y_0\wedge\dots\wedge x_{n-1}=y_{n-1})\wedge R(x_0,\dots x_{n-1})\rightarrow R(y_0,\dots, y_{n-1}))$ 
\item 
(congruence for functions) 
$\forall x_0,\dots,x_{n-1},y_0,\dots,y_{n-1}\ ((x_0=y_0\wedge\dots\wedge x_{n-1}=y_{n-1})\rightarrow f(x_0,\dots, x_{n-1})=f(y_0,\dots, y_{n-1}))$ 
\end{enumerate-(a)} 
\end{definition} 

If $T$ is a set of $\L$-sentences and $\varphi$ is an $\L$-formula, we say the \emph{$T$ syntactially implies $\varphi$} ($T\vdash \varphi$) if there is a formal derivation of $\varphi$ from $T$ together with the equality axioms in one of the standard proof calculi.\footnote{See any textbook on mathematical logic, for instance Martin Ziegler's book 'Mathematische Logik' \cite{MR2683672}. } Moreover, we say that $T$ \emph{implies} $\varphi$ ($T\models \varphi$) if every model $\MM$ of $T$ with an assignment of free variables is a model of $\varphi$. 

\begin{definition} 
Suppose that $\MM=(M,\vec{R})$ and $\NN=(N,\vec{R})$ are $\L$-structures and $h\colon M\rightarrow N$. 
\begin{enumerate-(a)} 
\item 
$h$ is an \emph{elementary embedding} if for all $a_0,\dots,a_{n-1}\in M$ and all $\L$-formulas $\varphi(x_0,\dots,x_{n-1})$, we have $\MM\vDash \varphi(a_0,\dots,a_{n-1})$ if and only if $\NN\vDash \varphi(h(a_0),\dots,h(a_{n-1}))$. 
\item 
$\MM$ is an \emph{elementary substructure} of $\NN$ and $\NN$ is an \emph{elementary extension} of $\MM$ ($\MM\prec \NN$) if $M\subseteq N$ and $\id\colon M\rightarrow N$ is an elementary embedding from $\MM$ to $\NN$. 
\item 
$\MM$ and $\NN$ are \emph{elementarily equivalent} ($\MM\equiv\NN$) if for all $\L$-sentences $\varphi$, we have $\MM\models \varphi$ if and only if $\NN\models \varphi$. 
\end{enumerate-(a)} 
\end{definition} 

\begin{example} 
\begin{enumerate-(a)} 
\item 
$(\ZZ,<)$ is a substructure of $(\QQ,<)$, but not an elementary substructure. 
\item 
$(\QQ,<)$ is an elementary substructure of $(\RR,<)$. 
\end{enumerate-(a)} 
\end{example} 

An \emph{$\L$-theory} is a set of $\L$-sentences. 

\begin{example} 
The theory of groups is the set of axioms 
%\begin{enumerate-(a)} 
%\item 
$\forall x,y,z\ (x\cdot y)\cdot z=x\cdot (y\cdot z)$, 
%\item 
$\forall x (x\cdot 1= 1\cdot x= x)$ and 
%\item 
$\forall x (x\cdot x^{-1}= x^{-1}\cdot x = 1)$
%\end{enumerate-(a)} 
\end{example} 

\begin{definition} 
Suppose that $T$ is a set of $\L$-sentences. 
\begin{enumerate-(a)} 
\item 
$T$ is \emph{syntactically consistent} if there is no formal proof of a contradiction from $T$. 
\item 
$T$ is \emph{deductively closed} if $\psi\in T$ for all $\varphi_0,\dots,\varphi_{n-1}\in T$ and $\psi\in\mathrm{Form}_\L$ with $\{\varphi_0,\dots,\varphi_{n-1}\}\vdash \psi$. 
\item 
$T$ is \emph{satisfiable} or \emph{consistent} if there is a model $\MM$ of $T$. 
\item 
$T$ is \emph{finitely satisfiable} if every finite subset of $T$ is satisfiable. 
\item 
If $T$ is a theory, it is \emph{complete} if it is consistent and for every $\L$-sentence $\varphi$, either $T\models \varphi$ or $T\models \neg\varphi$. 
\end{enumerate-(a)} 
\end{definition} 

For example, if $\MM$ is an $\L$-structure then the theory $\Th(\MM)=\{\varphi\in\mathrm{Sent}_\L\mid \MM\models \varphi\}$ of $\MM$ is complete. 

\begin{lemma} 
If $T$ is a consistent deductively closed theory, the following conditions are equivalent. 
\begin{enumerate-(a)} 
\item 
$T$ is complete. 
\item 
There is a structure $\MM$ with $\Th(\MM)=T$. 
\item 
All models of $T$ are elementarily equivalent. 
\end{enumerate-(a)} 
\end{lemma} 
\begin{proof} 
Since $T$ is consistent, there is a model $\MM$ of $T$. Since $T$ is complete, we have $\Th(\MM)=T$. Assuming that there is a structure $\MM$ with theory $T$, we have that any model of $T$ satisfies the same sentences as $\MM$. Finally, assume that all models of $T$ are elementarily equivalent. To show that $T$ is complete, we assume towards a contradiction that $T\not\models\varphi$ and $T\not\models\neg\varphi$. Thus both theories $T\cup\{\varphi\}$ and $T\cup\{\neg\varphi\}$ have model, contradicting the assumption. 
\end{proof} 

\begin{lemma} 
Suppose that $\MM=(M,\dots)$ is an $\L$-structure and $\MM^*$ is the $\L_M$-structure induced by $\MM$. Then the elementary extensions of $\MM$ correspond to the models of $\Th(\MM^*)$ as follows. 
\begin{enumerate-(a)} 
\item 
If $\NN=(N,\dots)$ is an $\L$-structure that is an elementary extension of $\MM$, then it is also a model of $\Th(\MM^*)$ with the canonical interpretation of the new constants. 
\item 
If $\NN=(N,\dots)$ is a model of $\Th(\MM^*)$, then its reduct to $\L$ has an elementary submodel that is isomorphic to $\MM$. 
\end{enumerate-(a)} 
\end{lemma} 
\begin{proof} 
To prove the first claim, suppose that $\varphi(x_0,\dots,x_{n-1})$ is an $\L$-formula and $a_0,\dots,a_{n-1}\in M$ with $\MM\models \varphi(a_0,\dots,a_{n-1})$. Since $a_0,\dots,a_{n-1}$ are constants in $\L_M$, $\varphi(a_0,\dots,a_{n-1})\in \Th(\MM^*)$ and each $a_i$ takes the value $a_i$ in $\NN$, we have that $\NN\models \varphi(a_0,\dots,a_{n-1})$. 

For the second claim, the function $h\colon M\rightarrow N$, $h(a)=a^{\NN}$ is an elementary embedding with respect to $\L$-formulas and thus the claim follows. 
\end{proof} 


\begin{definition} 
Suppose that $\MM=(M,\dots)$ is an $\L$-structure and $A\subseteq M$. We say that $X\subseteq M^n$ is \emph{$A$-definable} if there is an $\L$-formula $\varphi(x_0,\dots,x_n,y_0,\dots,y_m)$ and $a_0,\dots,a_m\in A$ with $X=\{(x_0,\dots,x_n)\mid \varphi(x_0,\dots,x_n,a_0,\dots,a_m)\}$. Moreover $X$ is called \emph{definable} if it is $\emptyset$-definable. 
\end{definition} 

\begin{example} 
\begin{enumerate-(a)} 
\item 
Let $(R,+,-,\cdot,0,1)$ be a ring and $f(X)\in R[X]$ is an element of the polynomial ring $R[X]$ with coefficients that are definable in $R$. Then $\{x\in R\mid f(x)=0\}$ is definable over $(R,+,-,\cdot,0,1)$. 
\item 
Let $(\RR,+,-,\cdot,0,1)$ be the ring of real numbers. Then the standard order of $\RR$ is defined by the formula $\varphi(x,y)=\exists z\ (z\neq0\wedge x+z^2=y)$. 
\item 
Let $(\ZZ,+,-,\cdot,0,1)$ be the ring of integers. By Lagrange's four-square theorem,\footnote{You can find a proof sketch of this theorem in the article \href{https://en.wikipedia.org/wiki/Lagrange\%27s_four-square_theorem}{'Lagrange's four-square theorem'} on Wikipedia. } 
every natural number is the sum of $4$ squares of integers. Hence the standard order of the integers is defined by the formula $\varphi(x,y)=\exists z_0,\dots,z_3\ (x+z_0^2+\dots+z_3^2=y)$. 
\end{enumerate-(a)} 
\end{example} 

\begin{example} 
The computably enumerable sets of natural numbers are the same as the subsets of the structure $(\NNN,0,+,\cdot,<)$ that are definable by a formula of the form $\exists y\varphi(x,y)$, where $\varphi$ is a formula with only bounded quantifiers of the form $\forall m< n$ and $\exists m< n$. This follows from standard results in mathematical logic. 
\end{example} 

\begin{remark} 
The theory of $(\NNN,+,\cdot)$ is undecidable. To see this, assume that there is an algorithm that decides for every sentence $\varphi$ whether $\varphi$ of $\neg\varphi$ holds. 
%Moreover, we will write $\NN$ for this structure. 
By standard results in mathematical logic, there is a formula $\varphi(x,y)$ that defines the halting problem in $(\NNN,+,\cdot)$, i.e. we have $(\NNN,+,\cdot)\models \varphi(m,n)$ if and only if the Turing machine coded by $m$ halts when given the input $n$. However, it is easy to see that the halting problem is not decidable and moreover it is definable in this structure. 
\end{remark} 

Given any set $S$, we now define $S$-sorted languages $\L$, $S$-sorted structures and $S$-sorted $\L$-structures as follows. We will call the elements of $S$ \emph{sorts}. 

\begin{definition} 
\begin{enumerate-(a)} 
\item 
An \emph{$S$-sorted language} is a language $\L$ where each constant symbol has a sort $s\in S$, each $n$-ary relation symbol has a type $(s_0,\dots,s_{n-1})\in S^n$ and each $n$-ary function symbol has a type $(s_0,\dots,s_{n-1},s)\in S^{n+1}$. 
\item 
An \emph{$S$-sorted structure} is a pair $(\vec{M},\vec{R})$, where $\vec{M}=\langle M_s\mid s\in S\rangle$ and each element of $\vec{R}$ is either an element of $M_s$, a subset of $M_{s_0}\times \dots \times M_{s_{n-1}}$ or a function from $M_{s_0}\times \dots \times M_{s_{n-1}}$ to $M_s$ for some $s_0,\dots,s_{n-1},s\in S$. 
\item
Suppose that $\L$ is an $S$-sorted language. An \emph{$\L$-structure} is an $S$-sorted structure $(\vec{M},\langle R^{\vec{M}}\mid R\in\L\rangle)$ such that $c^{\vec{M}}\in M_s$ for each constant symbol $c\in\L$ of sort $s$, $R^{\vec{M}}\subseteq M_{s_0}\times \dots \times M_{s_{n-1}}$ for each relation symbol $R\in\L$ of type $(s_0,\dots,s_{n-1})$ and $f^{\vec{M}}\colon M_{s_0}\times \dots \times M_{s_{n-1}}\rightarrow M_s$ for each function symbol $f\in\L$ of type $(s_0,\dots,s_{n-1},s)$. 
\end{enumerate-(a)} 
\end{definition} 

If $|S|=n$, we also talk about $n$-sorted languages and structures. 

\begin{example} 
Vector spaces over arbitrary fields can be easily formalized as $2$-sorted structures, where one sort corresponds to fields and the other one to vector spaces. 
\end{example} 




%%%%%%%%%%%%
\subsection{Elementary substructures and extensions} 

The proof of the following theorem can be found in most textbooks in mathematical logic, and is essentially the same as that of \cite[Theorem 2.2.1]{MR2908005}. 

\begin{theorem} (G\"odel) \label{syntactically consistent theories have models} 
Every syntactically consistent set of $\L$-formulas has a model with an assignment of free variables. 
\end{theorem} 

%Modulo the fact that all formal implications are correct, 
This easily implies the completeness theorem. 

\begin{theorem} (Completeness) 
Suppose that $\L$ is a language, $T$ is a set of $\L$-formulas and $\varphi$ is an $\L$-formula. Then $T\vdash \varphi$ if and only if $T\models \varphi$. 
\end{theorem} 
\begin{proof} 
%To prove the forward implication, we can assume that both $T$ and $\varphi$ don't contain free variables by replacing them with new constants, since the statements $T\models \varphi$ and $T\vdash \varphi$ both remain equivalent to the version with free variables by this replacement. In the first case this is clear and in the second case this can be checked in any formal proof calculus. 
The first implication holds since all rules of formal proof calculi are correct. To prove the reverse implication, we can assume that $T$ is syntactically consistent, since otherwise both $T\models \varphi$ and $T\vdash\varphi$ hold. We now assume that $T\models \varphi$ but $T\not\vdash \varphi$. In any proof calculus, $T\not\vdash \varphi$ implies that $T\cup\{\neg\varphi\}$ is syntactically consistent. 
%(here it is used that $\varphi$ does not have free variables). 
Hence this theory has a model by Theorem \ref{syntactically consistent theories have models}, but this contradicts the assumption. 
%If $T\cup\{\varphi\}$ and $T\cup\{\neg\varphi\}$ are both inconsistent then $T\cup \{\varphi\vee \neg \varphi\}$ is as well, but this would contradict the assumption that $T$ is consistent. Hence $T\cup\{\neg\varphi\}$ is consistent, contradicting the assumption that $T\models\varphi$. The reverse implication follows from the fact that the formal proof calculus is correct. 
\end{proof} 

We immediately obtain the following result from the completeness theorem. 

\begin{theorem} (Compactness) \label{compactness theorem} 
Every finitely satisfiable theory is satisfiable. 
\end{theorem} 

The following is one of the basic applications of compactness. 

\begin{lemma} 
If a theory $T$ has arbitrarily large finite models, then $T$ has an infinite model. 
\end{lemma} 
\begin{proof} 
Let $\varphi_n$ be the sentence $\exists x_0,\dots,x_n \bigwedge_{i,j\leq n,\ i\neq j} x_i\neq x_j$ and let $T^*=T\cup\{\varphi_n\mid n\in\NNN\}$. Then $T^*$ is finitely satisfiable and hence satisfiable by the compactness theorem. Moreover any model of $T^*$ is infinite. 
\end{proof} 

The following criterion is useful to prove that for a structure $\MM=(M,\dots)$, a subset of $M$ is the domain of an elementary substructure. 

\begin{lemma} \label{Tarskis test} (Tarski's test) 
If $\NN=(N,...)$ is an $\L$-structure and $M\subseteq N$, 
%substructure of an $\L$-structure $\NN=(N,...)$. 
the following conditions are equivalent. 
\begin{enumerate-(a)} 
\item 
$M$ is the domain of a (unique) elementary substructure of $\NN$. 
%$\MM\prec \NN$ 
\item 
The following holds for all $\L$-formulas $\varphi(x,x_0,\dots,x_{n-1})$ and $a_0,\dots, a_{n-1}\in M$. If there is some $b\in N$ with $\NN\models \varphi(b,a_0,\dots,a_{n-1})$, then there is some $a\in M$ with $\NN\models \varphi(a,a_0,\dots,a_{n-1})$. 
\end{enumerate-(a)} 
\end{lemma} 
\begin{proof} 
The first implication is clear. Assuming that the second statement holds, it follows that $M$ is the domain of a (unique) subtructure $\MM$ of $\NN$ by applying the statement to the formula $\varphi(x,x_0,\dots,x_{n-1})=(f(x_0,\dots,x_{n-1})=x)$. We now prove $\MM\prec \NN$ by induction on formulas. Since the cases for $\wedge$, $\vee$ and $\neg$ are clear, we only prove the existential case. To this end, we first assume that $a_0,\dots, a_{n-1}\in M$ and $\MM\models \exists x \varphi(x,a_0,\dots,a_{n-1})$. We choose some $a\in M$ with $\MM\models \varphi(a,a_0,\dots,a_{n-1})$. By the induction hypothesis, we have $\NN\models \varphi(a_0,\dots,a_{n-1})$. We now assume that $a_0,\dots, a_{n-1}\in M$ and $\NN\models \exists x \varphi(x,a_0,\dots,a_{n-1})$. By the assumption, there is some $a\in M$ with $\NN\models \exists x \varphi(x,a_0,\dots,a_{n-1})$. Thus $\MM\models \varphi(x,a_0,\dots,a_{n-1})$ by the induction hypothesis. 
\end{proof} 

To obtain precise sizes of models in the L\"owenheim-Skolem theorem below, we will use the following fact. It is an easy exercise in set theory that is based on the fact that $\kappa\cdot\lambda=|\kappa\times\lambda|= \max\{\kappa,\lambda\}$ for all infinite cardinals $\kappa$, $\lambda$. 

\begin{problem} \label{cardinality of the set of finite subsets} 
Suppose that $\kappa$ is an infinite cardinal and $X$ is a set of size $\kappa$. 
\begin{enumerate-(a)} 
\item 
If $\langle X_i\mid i<\alpha\rangle$ is a sequence of sets of size at most $\kappa$ and $\alpha\leq\kappa$, then $|\bigcup_{i\leq\kappa} X_i|\leq\kappa$. 
\item 
The set $X^{<\omega}$ of finite sequences in $X$ has size $\kappa$. 
\item 
The set $[X]^{<\omega}$ of finite subsets of $X$ has size $\kappa$. 
\end{enumerate-(a)} 
\end{problem} 

Since any formula is a finite sequence of symbols, it follows that $|\mathrm{Form}_\L|=\max\{|\L|,\omega\}$ for any language $\L$. 
%is countably infinite for any countable language $\L$ and $|\mathrm{Form}_\L|=|\L|$ for any infinite language $\L$. 

\begin{definition} 
Suppose that $\MM=(M,\dots)$ is an $\L$-structure. 
\begin{enumerate-(a)} 
\item 
A \emph{Skolem function} for an $\L$-formula $\varphi(x_0,\dots,x_{n-1})$ with respect to $\MM$ is a function $f_{\varphi}\colon M^n\rightarrow M$ such that for all $a_0,\dots,a_{n-1}\in M$ with $\MM\models \exists x \varphi(x,a_0,\dots,a_{n-1})$, we have $\MM\models \varphi(f_\varphi(a_0,\dots,a_{n-1}),a_0,\dots,a_{n-1})$. 
\item 
If $f_\varphi$ is a Skolem function with respect to $\MM$ for each $\varphi\in \mathrm{Form}_\L$, then the set $\mathcal{F}=\{f_\varphi\mid \varphi\in\mathrm{Form}_\L\}$ is called a \emph{set of Skolem functions} for $\MM$. 
%\item 
%A \emph{sequence of Skolem functions} for $\MM$ is a sequence $\vec{f}=\langle f_\varphi\mid \varphi\in\mathrm{Form}_\L\rangle$, where each $f_\varphi$ is a Skolem function for $\varphi$ with respect to $\MM$. 
\item 
Suppose that $A\subseteq M$ and $\mathcal{F}$ is a set of Skolem functions for $\MM$. The \emph{Skolem hull} of $A$ in $\MM$ with respect to $\mathcal{F}$ is defined as $h_{\mathcal{F}}(A)=\bigcup_{n\in\NN} h_n(A)$, where $h_0(A)=A$ and for all $n\in\NN$, $h_{n+1}(A)=\{f_\varphi(a_0,\dots,a_{m-1})\mid \varphi(x_0,\dots,x_{m-1})\in\mathrm{Form}_\L,\ a_0,\dots,a_{m-1}\in h_n(A)\}$. 
\end{enumerate-(a)} 
\end{definition} 

\begin{theorem} (L\"owenheim-Skolem) 
Let $\MM=(M,\dots)$ be an $\L$-structure, $A\subseteq M$ and $\kappa$ an infinite cardinal. 
\begin{enumerate-(1)} 
\item 
If $\max\{|A|,|\L|\}\leq\kappa\leq |M|$, then $\MM$ has an elementary substructure of size $\kappa$ that contains $A$. 
\item 
If $M$ is infinite and $\max\{|M|,|\L|\}\leq\kappa$, then $\MM$ has an elementary extension of size $\kappa$. 
\end{enumerate-(1)} 
\end{theorem} 
\begin{proof} 
To prove the first claim, let $\mathcal{F}$ be a set of Skolem functions for $\MM$ and $B$ a set of size $\kappa$ with $A\subseteq B\subseteq M$. By applying Tarski's test, it follows immediately that $h_{\mathcal{F}}(B)$ is the domain of an elementary substructure of $\MM$. Moreover $|h_n(B)|=\kappa$ for every $n\in\NN$ and $|h_{\mathcal{F}}(B)|=\kappa$ by Problem \ref{cardinality of the set of finite subsets}. 

The second claim is proved by applying the compactness theorem. We first construct an elementary extension of $\MM$ of size at least $\kappa$. To this end, let $\L'$ be the language $\L_M\cup\{c_i\mid i\in I\}$, where $I$ is a set of size $\kappa$ and $T$ the $\L'$-theory $\Th(\MM)\cup\{c_i\neq c_j\mid i,j\in I,\ i\neq j\}$. Since $M$ is infinite, $T$ is finitely satisfiable and hence there is a model $\NN=(N,\dots)$ of $T$ by the compactness theorem. Since $|N|\geq\kappa$, we can apply the first claim to obtain an elementary submodel of $\NN$ of size $\kappa$ whose domain contains $M$. 
\end{proof} 

If $T$ is an $\L$-theory, let $\Mod(T)$ denote the class of $\L$-structures that are models of $T$. We further call a class $\cC$ of $\L$-structures \emph{axiomatizable} or \emph{elementary} if $\cC=\Mod(T)$ for some $\L$-theory $T$. Moreover $\cC$ is called \emph{finitely axiomatizable} if $\cC=\Mod(T)$ for some finite theory $T$. 

\begin{lemma} 
A class $\cC$ of $\L$-structures is finitely axiomatizable if and only if both $\cC$ and its complement are axiomatizable. 
\end{lemma} 
\begin{proof} 
If $\cC=\Mod(T)$ for some finite theory $T=\{\varphi_0,\dots,\varphi_{n-1}\}$, then in fact $\cC=\Mod(\varphi)$ for $\varphi=\varphi_0\wedge\dots\wedge \varphi_{n-1}$ and hence the complement of $\cC$ is axiomatized by $\neg\varphi$.  

Now assume that $\cC$ is axiomatized by $T$ and its complement by $T'$. Since $T\cup T'$ is not satisfiable, it is not finitely satisfiable by the compactness theorem. Thus we can find finite subset $S$ and $S'$ of $T$ and $T'$ such that $S\cup S'$ is not satisfiable. We now claim that $\cC=\Mod(S)$. Otherwise there is some $M\in\Mod(S)\setminus \Mod(T)$ and hence $M\in\Mod(S\cup T')$. 
\end{proof} 

\begin{example} 
The class of connected graphs is not axiomatizable. To see this, we assume towards a contradiction that it is axiomatized by an $\L$-theory $T$, where $\L$ contains a binary relation symbol $R$ that is interpreted as the edge relation for graphs. We now expand the language to $\L'=\L\cup\{c,d\}$, where $c$ and $d$ are new constants, and further let $T'=T\cup\{\varphi_n\mid n\in\NNN\}$, where $\varphi_n=\neg(\exists x_0,\dots,x_n (\bigwedge_{i<n} R(x_i,x_{i+1}) \wedge x_0=c\wedge x_n=d))$ for each $n\in\NNN$, i.e. $\varphi_n$ states that there is no path of length $n$ from $c$ to $d$. It is easy to see that $T'$ is finitely satisfiable and hence has a model $\MM$ by the compactness theorem. Since $\varphi_n$ holds in $\MM$ for all $n\in\NNN$, $c^\MM$ and $d^\MM$ are not connected in $\MM$ by a finite path, contradicting the assumption. 
\end{example} 

We will also use Los' theorem about ultraproducts (see \cite[pp. 149-152]{MR1940513}). Joshua Chen asked whether stronger languages can axiomatize more classes of models. In particular, we can ask whether $\cC$ can become axiomatizable by adding a single relation symbol, or if the following holds. 

\begin{question} 
Is there a countable language $\L$ and a class $\cC$ of $\L$-structures that is not axiomatizable in $\L$, but is equal to the set of reducts of the models of a theory in some countable language that extends $\L$? 
\end{question} 





%%%%%%%%%%%%
%%%%%%%%%%%%
\section{Proving categoricity} 

In this section, we consider several natural theories such as those of dense linear orders, vector spaces over $\QQ$ and algebraically closed fields. We will determine for which infinite cardinals $\kappa$ these theories are $\kappa$-categorical. 



%%%%%%%%%%%%
\subsection{Categoricity of various theories} 

We will see various examples of $\aleph_0$-categorical and uncountably categorical theories. The following criterion shows that all such theories are already complete. 

\begin{lemma} \label{Vaught's test} (Vaught's test) 
Suppose that $\kappa$ is an infinite cardinal, $\L$ is a language with $|\L|\leq\kappa$ and $T$ is a consistent theory with no finite models. If $T$ is $\kappa$-categorical, then it is complete. 
\end{lemma} 
\begin{proof} 
We show that any two models $\MM$ and $\NN$ of $T$ are elementarily equivalent. since $\MM$ and $\NN$ are infinite and $|\L|\leq\kappa$, $\Th(\MM)$ and $\Th(\NN)$ have infinite models $\MM'$ and $\NN'$ of size $\kappa$ by the L\"owenheim-Skolem Theorem. By our assumption, $\MM\equiv \MM'\cong\NN'\equiv \NN$. 
\end{proof} 

\begin{example} 
Our first example is the language $\L = \{E\}$, where $E$ is a binary relation symbol, and $T$ is the theory of an equivalence relation with exactly two classes, both of which are infinite; the axioms are those of equivalence relations together with the axioms $\exists x, y \neg E(x,y)$, $\forall x,y,z (E(x,y)\vee E(x,z)\vee E(y,z)$ and $\varphi_n=\forall x_0,\dots,x_n(\bigwedge_{i\leq n} E(x_0,x_i)\rightarrow \exists x \bigwedge_{i\leq n} (x\neq x_i\wedge E(x,x_i)))$ for all $n\in\NNN$. It is easy to see that any two countable models of $T$ are isomorphic. However $T$ is not $\kappa$-categorical for any uncountable cardinal $\kappa$. To see this, let $\MM$ be a model of $T$ of size $\kappa$ where both equivalence classes have size $\kappa$ and let $\NN$ be a model of size $\kappa$ where one equivalence class has size $\kappa$ and the other one has $\aleph_0$. 
\end{example} 

The theory $\DLO$ of dense linear orders without end points is defined as the theory of (strict) linear orders without end points in the language $\L_{LO}=\{<\}$ with the additional axioms $\forall x,z(x<z\rightarrow \exists y (x<y<z))$, $\forall y \exists x (x<y)$ and $\forall x \exists y (x<y)$. For example $(\QQ,<)$ and $(\RR,<)$ are models of $\DLO$. 

\begin{theorem} \label{DLO is aleph0-categorical} 
$\DLO$ is $\aleph_0$-categorical. 
\end{theorem} 
\begin{proof} 
This is a typical example of a \emph{back-and-forth construction}. Suppose that $\AA=(A,<_A)$ and $\BB=(B,<_B)$ are countably infinite models of $\DLO$ and let $\langle a_n\mid n\in\NNN\rangle$ and $\langle b_n\mid n\in\NNN\rangle$ enumerate them without repetitions. We construct a sequence of finite sets $A_n$, $B_n$ of $A$, $B$ and isomorphisms $f_n\colon A_n\rightarrow B_n$, starting with $A_0=B_0=f_0=\emptyset$. 

For the induction step, assume that $A_n$, $B_n$ and $f_n$ are already defined. If $a_n\in A_n$, let $A_n'=A_n$, $B_n'=B_n$ and  $f_n'=f_n$. If on the other hand $a_n\notin A_n$, then we let $A_n'=A_n\cup\{a_n\}$ and choose some $b_n'\in B$ such that the extension $f_n'$ of $f_n$ with $f_n'(a_n)=b_n'$ is an isomorphism of $A_n'$ to $B_n'=B_n\cup\{b_n'\}$. If $b_n\in B_n'$, let $A_{n+1}=A_n'$, $B_{n+1}=B_n'$ and  $f_{n+1}=f_n'$. If however $b_n\notin B_n'$, we let $B_{n+1}=B_{n'}\cup\{b_n\}$ and choose some $a_n'\in A$ such that the extension $f_{n+1}$ of $f_n'$ with $f_{n+1}(a_n')=b_n$ from $A_{n+1}=A_n'\cup\{a_n'\}$ to $B_{n+1}$ is an isomorphism. 

We finally define $f$ as the union of the functions $f_n$ for all $n\in\NNN$. Then $f\colon A\rightarrow B$ is bijective and it is an isomorphism, since each $f_n$ is an isomorphism. 
\end{proof} 

If $\AA=(A,<_A)$ and $\BB=(B,<_B)$ are linear orders, the \emph{lexicographical order} $<_\lex$ of these two linear orders on $A\times B$ is defined by letting $(a,b)<_\lex(a',b')$ if $a<a'$ or $(a=a' \wedge b<b')$. It is easy to check that this is always a linear order. 

\begin{theorem} 
$\DLO$ is not $\kappa$-categorical for any uncountable cardinal $\kappa$. 
\end{theorem} 
\begin{proof} 
We consider the usual linear order $<_\QQ$ on $\QQ$. We first claim that for any linear order $(\AA,<_\AA)$, the lexicographical order $<_\lex$ on $\AA\times \QQ$ given by $<_\AA$ and $<_\QQ$ is dense. To show this, assume that $(a,q)<_\lex (b,r)$. If $a<_\AA b$, then we can pick any $q'\in \QQ$ with $q<_\QQ q'$ and have that $(a,q)<_\lex(a,q')<_\lex(b,r)$. If otherwise $a= b$ and $q<_\QQ r$, then we choose some $q'\in\QQ$ with $q<_\QQ q'<_\QQ r$ and have that $(a,q)<_\lex(a,q')<_\lex(b,r)$. Moreover $<_\lex$ has no end points, since there are no end points in $<_\QQ$. 

Two prove the claim, we identify $\kappa$ with the set (of size $\kappa$) of all ordinals $\alpha<\kappa$ and let $(\kappa,<)$ denote the usual linear order. It is in fact a well-order, i.e. a linear order with no infinite decreasing sequences. We further let $<^*$ denote the reverse linear order on $\kappa$ that is defined by $\alpha<^*\beta\Leftrightarrow \beta<\alpha$. Now let $<_\lex$ and $<^*_\lex$ denote the lexicographical orders on $\kappa\times \QQ$ that are given by $<$, $<_\QQ$ and  $<^*$, $<_\QQ$. Since $(\kappa\times \QQ, <_\lex)$ and $(\kappa\times \QQ, <_\lex^*)$ are dense linear orders without end points, it suffices to show that they are not isomorphic. 

The first one contains strictly increasing sequences of length $\kappa$, for instance $\langle(\alpha,0)\mid \alpha<\kappa\rangle$. Assuming that they are isomorphic, there is such a sequence $\langle (\alpha_\beta,q_\beta)\mid \beta<\kappa\rangle$ in $(\kappa\times \QQ, <_\lex^*)$ as well. Then the sequence $\langle \alpha_\beta\mid \beta<\kappa\rangle$ is non-increasing by the definition of $<_\lex^*$. Since $(\kappa,<)$ is a well-order, it contains no infinite decreasing sequences and therefore, there is some $\beta_0<\kappa$ such that $\alpha_\beta=\alpha_\gamma$ for all $\beta$ with $\beta_0\leq \beta<\kappa$. By the definition of $<_\lex^*$, it follows that the sequence $\langle q_\beta\mid \beta_0\leq \beta<\kappa\rangle$ is an uncountable strictly decreasing sequence in $\QQ$. This contradicts the fact that $\QQ$ is countable. 
\end{proof} 

Suppose that $\KK=(K,0,+,-,\cdot)$ is a field, $\L=\L_{AG}\cup\{m_a\mid a\in K\}$, where $m_a$ is interpreted as scalar multiplication with $a$, and $T$ is the $\L$-theory of $\KK$-vector spaces. The reason for adding the function symbols $m_a$ is that we want the models to be vector spaces over some fixed field, instead of having different fields for different models. 

\begin{example} 
The $\L$-theory $T$ of $K$-vector spaces is $\kappa$-categorical for all cardinals $\kappa>|K|$, since any two $K$-vector spaces are isomorphic if and only if their dimension is equal and for any $K$-vector space $V$ of size $|V|>|K|$, we have $|V|=\mathrm{dim}(V)$. Thus for finite fields, $T$ is $\kappa$-categorical for all infinite cardinals $\kappa$ and for all countably infinite fields, it is not $\aleph_0$-categorical but $\kappa$-categorical for all uncountable cardinals $\kappa$. 
\end{example} 

The next example is the theory of torsion-free divisible abelian groups. To define these notions, suppose that $(G,0,+,-)$ is an abelian group. We write $n\cdot x= x+\dots+x$ ($n$ times) for all $x\in G$ and $n\in\NN$. The group is called \emph{torsion-free} if it satisfies the axioms $\forall x\neq 0\ n \cdot x\neq 0$ for all $n\in\NN$ and \emph{divisible} if it satisfies the axioms $\forall x\exists y\ x=n\cdot y$ for all $n\in\NN$. 

\begin{lemma} 
The theory of torsion-free abelian groups is not $\aleph_0$-categorical, but $\kappa$-categorical for all uncountable cardinals $\kappa$. 
\end{lemma} 
\begin{proof} 
We show that every torsion-free abelian group is a $\QQ$-vector space and conversely. Assume first that $(G,0,+,-)$ is a torsion-free divisible abelian group. Then for any $g\in G$ and $n>0$, there is some $h\in G$ with $n\cdot h=g$. If $h'\in G$ and $n\cdot h'=g$, then $n(h-h')=0$ and since the group is torsion-free, we have $h=h'$. We can thus define $\frac{m}{n}\cdot g= m\cdot h$. It is easy to check that this gives $G$ the structure of a $\QQ$-vector space. If conversely $(V,0,+,-,\langle m_q\mid q\in\QQ\rangle)$ is a $\QQ$-vector space, then it is clear that $(V,0,+,-)$ is a torsion-free divisible abelian group. 
\end{proof} 

We now turn to algebraically closed fields. Recall that a field is called \emph{algebraically closed} if every polynomial $f(X)\in K[X]$ has a root in $K$. If moreover $K\subseteq L$ are fields, then an element $x\in L$ is called \emph{algebraic} over $K$ if it is the solution  to some polynomial in $K[X]$ and \emph{transcendent} otherwise. Moreover $L$ is called \emph{algebraic} over $K$ if each of its elements is algebraic over $K$. The \emph{characteristic} of a field $\KK=(K,0,1,+,-,\cdot)$ is the least $n>0$ with $n\cdot 1=0$ if this exists and $0$ otherwise. Since for all $a,b\neq 0$ we have $a\cdot b\neq 0$,  the characteristic $n$ is necessarily a prime number; if $n=m\cdot k$ with $m,k>1$, then $(m\cdot 1)\cdot (k\cdot 1)=0$ in $K$ and hence one of $m\cdot 1$ and $k\cdot 1$ is equal to $0$, contradicting the minimality of $n$. 

We first recall some facts about fields and polynomials over fields. In the following, we will use without proof that polynomial division is possible in $K[X]$.\footnote{A proof of this can be found in any textbook in algebra.} Note that no finite field $K$ is algebraically closed, since for  $K=\{a_0,\dots,a_n\}$ the polynomial $f(X)=(X-a_0)\cdot \dots \cdot(X-a_n)+1$ has no root in $K$. We further call a polynomial $f$ in the polynomial ring $K[X]$ \emph{irreducible} if there are no polynomials $g,h\in K[X]$ of degrees $\deg(g),\deg(h)>0$ with $f= g\cdot h$; for example the polynomial $X^2+1$ is irreducible over $\RR$. Moreover, a \emph{greatest common divisor} of two polynomials $f,g\in K[X]$ is a polynomial $h\in K[X]$ with leading coefficient $1$ that is a divisor of both $f$ and $g$ and a polynomial multiple of every other common divisor of $f$ and $g$. 

\begin{fact} 
If $f$ and $g$ are non-zero polynomials in $K[X]$, then there is a unique greatest common divisor $h\in K[X]$ of $f$ and $g$ and there are polynomials $f',g'\in K[X]$ with $f\cdot f'+ g\cdot g'= h$. 
\end{fact} 

\todo{write out?} This is easy to show by taking the unique non-zero polynomial $h\in \{f\cdot f'+ g\cdot g'\mid f',g'\in K[X]\}$ of minimal degree with leading coefficient $1$ and using polynomial division. 

An \emph{ideal} in a commutative ring $(R,0,1,+,-,\cdot)$ is a subgroup $I$ of $R$ with $R\cdot I\subseteq I$ and the least ideal containing some $x\in R$ is denoted by $(x)=R\cdot x$. If $I$ is an ideal in $R$, then is it easy to check that $R/I=\{x+I\mid x\in R\}$ forms a ring with addition $(x+I)+(y+I)=x+y+I$ and multiplication $(x+I)\cdot(y+I)=(x\cdot y)+I$. 

\begin{lemma} 
If $f$ is an irreducible polynomial in $K[X]$, then the ring $L=K[X]/(f)$ is a field that contains a root of $f$. Moreover $\dim_K(L)\leq n$ if $\deg(f)=n$. 
% and hence $L$ is an algebraic extension of $K$. 
\end{lemma} 
\begin{proof} 
Let $I=(f)$. To see that $L$ is a field, it is sufficient to show that  every nonzero element $g+I$ of $L$ has a multiplicative inverse. Since $g\notin I$, $f$ does not divide $g$ and since $f$ is irreducible, the greatest common divisor $h\in K[X]$ of $f$ and $g$ is equal to $1$. Hence there are polynomials $f',g'\in K[X]$ with $f\cdot f'+ g\cdot g'=1$. Since $f\cdot f'\in I$, we have $g\cdot g'+I= 1+ I$ and hence $(g + I)(g' + I) = 1+I$, so $g'+I$ is a multiplicative inverse to $g+I$. 

Moreover, it is clear that $X+I$ is a root of $f$ in $L$. 
To finally see that $\dim_K(L)\leq n$, suppose that $f(X)=a_0+a_1 X+\dots + a_n X^n$, where $a_n\neq 0$. Then $X^n=\frac{a_0}{a_n}+\dots +\frac{a_{n-1}}{a_0}X^{n-1}$ and hence every element $g+I$ in $L$ is equal to $h+I$ for some polynomial $h$ of degree $\deg(h)<n$. Therefore $X,X^2,\dots, X^{n-1}$ generates $L$ as a $K$-vector space. 
%Moreover, it is easy to see that every finite-dimensional extension of $K$ is algebraic. 
\end{proof} 

Suppose that $K$ is a subfield of $L$ and $x\in L\setminus K$ is algebraic over $K$. The \emph{minimal polynomial} $f\in K[X]$ of $x$ over $K$ is defined as the unique polynomial of minimal degree with leading coefficient $1$ and $f(x)=0$. It can be checked that this divides all polynomials $g\in K[X]$ with $g(x)=0$. This implies that the map $h\colon K[X]/(f)\rightarrow K(x)$ is an isomorphim, where $K(x)$ denotes the least subfield of $L$ that contains $K$ and $x$. 

This implies the following fact: if $K\subseteq L$ are fields and $x\in L$, then $x$ is algebraic over $K$ if and only if $\dim_K(K(x))$ is finite. 

If $K$ is a field, an \emph{algebraic closure} of $K$ is a field $\bar{K}\supseteq K$ that is both algebraically closed and algebraic over $K$. 

\begin{theorem} 
Every field $K$ has an algebraic closure $\bar{K}$ that is unique up to isomorphism. 
\end{theorem} 
\begin{proof} 
Suppose that $|K|=\kappa$ and $\lambda=\max\{\kappa,\omega\}^+$. Let $\cC$ be the set of fields $L\supseteq K$ that are algebraic over $K$ and whose domain is a subset of $\lambda$. Note that every algebraic field extension of $K$ has size strictly less than $\lambda$, since every polynomial has only finitely many roots. Hence for every field that is algebraic over $K$, there is an isomorphic field in $\cC$. We order $\cC$ by extension of fields. Since $\cC$ is closed under increasing unions, it has a maximal element $\bar{K}$ by Zorn's lemma. Towards a contradiction, suppose that $\bar{K}$ is not algebraically closed. Let $f(X)\in \bar{K}[X]$ with no roots in $\bar{K}$. We can write $f$ as a product $f=f_0\cdot \dots \cdot f_n$ of irreducible factors $f_i\in \bar{K}[X]$. By the previous lemma, there is a proper algebraic extension $L$ of $\bar{K}$ in which $f_0$ has a root. Now suppose that an arbitrary element of $L$ is given. It is the root $x$ of some polynomial $g(X)=a_0+a_1 X+\dots+a_n X^n\in \bar{K}[X]$. Since $a_i$ is algebraic over $K$ for all $i\leq n$, these elements generate a an extension $K(a_0,\dots,a_n)$ of $K$ that is finite-dimensional as a vector space over $K$. Since $x$ is algebraic over $K(a_0,\dots,a_n)$, $K(a_0,\dots,a_n,x)$ is finite dimensional over $K$ as a $K$-vector space and hence it is an algebraic extension of $K$. Hence $L$ is algebraic over $K$, contradicting the maximality of $\bar{K}$. 

To prove uniqueness, suppose that $L$ and $M$ are two algebraic closures of $K$. By Zorn's lemma, there is a maximal isomorphism $\pi\colon L'\rightarrow M'$, where $L'$ and $M'$ are subfields of $L$ and $M$. Towards a contradiction, we assume that $L'\neq L$ or $M\neq M'$ and we can hence assume that $L'\neq L$. We can further assume that $L'=M'$ and $\pi=\id_{L'}$. Now let $x\in L\setminus L'$ and $f\in L'[X]$ irreducible with $f(x)=0$. Then $\sigma\colon L'[X]/(f)\rightarrow L'(x)$, $\sigma(g+(f))=g(x)$ is a field isomorphism, since $\mathrm{ker}(\sigma)=\{g+(f)\mid g(x)=0\}=0$. Since $M$ is algebraically closed, there is some $y\in M\setminus M'$ with $f(y)=0$. Then $\tau\colon L'[X]/(f)\rightarrow L'(y)$, $\tau(g+(f))=g(y)$ and $\tau \sigma^{-1}\colon L'(x)\rightarrow L'(y)$ are isomorphisms, but this contradicts the fact that $\pi$ is maximal. 
\end{proof} 

The algebraic closure of $\QQ$ is simply the set of algebraic numbers in $\CCC$, since $\CCC$ is algebraically closed. We now aim to describe the algebraic closure of the fields $\FF_p=\ZZ/p\ZZ$, where $p$ is a prime. To see that $\FF_p$ is a field, note that for any $x,y\in \FF_p$, we have $(x+y)^p=\sum_{i\leq p}\binom{p}{i}x^i y^{p-i}=x^p+y^p$, since the remaining binomial coefficients cancel out. We now claim that  $x^p=x$ and hence $x\cdot x^{p-2}=1$ for all $x\in \FF_p$. This holds for $0$ and $1$ and the previous equation shows that it holds for all elements of $\FF_p$ by induction. 

Note that every field $K$ of characteristic $p$ for a prime $p$ contains an isomorphic copy of $\FF_p$, namely the subset $\{n\cdot 1\mid n<p\}$ of $K$. We now fix an algebraic closure $\bar{K}$ of $K=\FF_p$. Note that any finite field extension of $\FF_p$ with $\FF_p$-dimension $n$ has size $p^n$. 

\begin{lemma} 
For every $n>0$, there is a field $\FF_{p^n}$ of characteristic $p$ and size $p^n$ that is unique up to isomorphism. 
\end{lemma} 
\begin{proof} 
Letting $q=p^n$, we first prove that there is a subfield of $\bar{K}$ with these properties. The polynomial $f=X^q-X\in \FF_p[X]$ has no repeated factors, since its formal derivative is $f'=(X^q-X)'=qX^{q-1}-1=-1$ and therefore $f$ and $f'$ have no common factors. Let $x_0,\dots,x_{q-1}$ be the distinct roots of $f$ in $\bar{K}$ and $\FF_q=\FF_p(x_0,\dots,x_{q-1})$ The field extension generated by $x_0,\dots,x_{q-1}$. 

We now claim that $\FF_q=\{x_0,\dots,x_{q-1}\}$. To see this, let $F\colon \bar{K}\rightarrow \bar{K}$, $F(x)=x^q$. We have $(x+y)^q=\sum_{i\leq q}\binom{q}{i}x^i y^{q-i}=x^q+y^q$ and clearly $(x\cdot y)^q=x^q \cdot y^q$ for all $x,y\in \bar{K}$. Since $F(x)=x$ holds for all $x\in \FF_p$ and $x\in\{x_0,\dots,x_{q-1}\}$, the same holds for all elements of $\FF_p(x_0,\dots,x_{q-1})$ by these equations. But $\{x_0,\dots,x_{q-1}\}$ is the set of all solutions of the equation $F(x)=x$ in $\bar{K}$. 

It is sufficient to prove uniqueness inside $\bar{K}$, since every algebraic extension of $\FF_p$ embeds into $\bar{K}$ by the uniqueness of algebraic closures. To this end, assume that $L$ is such a subfield of $\bar{K}$ and $L^\times$ is the set of non-zero elements of $L$. Since $|L^\times|=p^n-1$, we have $x^{p^n-1}=1$ and hence $x^q=x$ for all $x\in L$. Hence $L\subseteq \FF_q$ and thus $L=\FF_q$. 
\end{proof} 

\begin{remark} 
The function $F$ is called the \emph{Frobenius automorphism} (depending on $q$). The previous proof shows that it is a field \todo{automorphism?} homomorphism of $\bar{K}$. Since $F^m$ fixes $\FF_{q^m}$ for every $m\in\NNN$, $F{\upharpoonright}\FF_{q^m}$ is an automorphism of $\FF_{q^m}$ which fixes exactly the elements of $\FF_q$. 
\end{remark} 

It follows from the previous lemma that $\bar{K}=\bigcup_{n\in\NNN} \FF_{p^n}$. Moreover we can write $\bar{K}$ as an increasing union of fields $\FF_{p^{n!}}$, since this is the set of roots of $X^{p^{n!}}-X$ and we have $\FF_{p^m}\subseteq \FF_{p^n}$ for all $m,n\in\NNN$ with $m|n$. 

%\begin{lemma} 
%If $f(X)$ is a polynomial in $\FF_q[X]$ and $x$ is a root in $\bar{K}$, then $F^n(x)$ is also a root for every $n>0$. 
%\end{lemma} 
%\begin{proof} 
%Suppose that $f(X)=a_0+a_1 X+\dots a_k X^k$. By applying $F^n$ to the equation $f(x)=0$, we obtain $a_0+a_1F^n(x)+\dots a_k F^n(x)^k=F^n(a_0)+F^n(a_1)F^n(x)+\dots F^n(a_k)F^n(x)^k=0$, since $F^n$ does not change coefficients in $\FF_q$. Hence $f(F^n(x))=0$ and thus $F^n(x)$ is a root of $f$ . 
%\end{proof} 
%
%\begin{lemma} 
%Suppose that $A=\{x_0,\dots,x_k\}$ is closed under $F$ and $x_i\neq x_j$ for all $i<j\leq k$. Then the polynomial $(X-x_0)\cdot \dots \cdot (X-x_k)$ has coefficients in $\FF_q$. 
%\end{lemma} 
%\begin{proof} 
%If $f(X)=a_0+a_1X+\cdot +a_nX^n$ is a polynomial in $\bar{K}[X]$, we define a new polynomial $F(f)$ by applying $F$ to the coefficients. By applying $F$ to the product above, we obtain $a_0+a_1X+\cdot +a_kX^k=(X-x_0)\cdot \dots \cdot (X-x_k)=(X-F(x_0))\cdot \dots \cdot (X-F(x_k))=F(a_0)+F(a_1)X+\cdot +F(a_k)X^k$. Hence the coefficients are equal and we have $F(a_i)=a_i$ for all $i\leq k$. By the definition of $\FF_q$, the coefficients are in $\FF_q$. 
%\end{proof} 
%
%The {minimal polynomial} of $x\in \bar{K}$ over $\FF_q$ is the unique polynomial $f\in\FF_q[X]$ with leading coefficient $1$ such that any other polynomial in $\FF_q[X]$ of which $x$ is a root is a polynomial multiple of $f$. Its existence follows by polynomial division. 
%
%\begin{lemma} 
%If $x\in \bar{K}$ generates an extension $\FF_q(x)$ with $\FF_q$-dimension $n>0$, then the minimal polynomial of $x$ is $f(X)=(X-x)(X-F(x))\cdot \dots \cdot (X,F^{n-1}(x))$. 
%\end{lemma} 
%\begin{proof} 
%It follows from the previous lemma that the $\FF_q$-dimension of $\FF_q(x)$ is equal to $n$. 
%
%Since $x$ generates a degree n extension of Fq, from above Fn? = ?. Thus, the set ?, F?, F2?, ..., Fn?1? is F-stable, and the right-hand side product (when multiplied out) has coefficients in Fq. Thus, it is a polynomial multiple of M. Since it is monic and has degree n (as does M), it must be M itself.
%\end{proof} 

We aim to define the transcendence degree of field extensions. To this end, we will assume that $K$, $L$ and $M$ are algebraically closed fields with $K\subseteq L, M$. A subset $A$ of $L$ is \emph{algebraically independent} over $K$ if for all $a_0,\dots,a_n\in A$ and $f\in K[X_0,\dots,X_n]$ with $f\neq 0$, we have $f(a_0,\dots,a_n)\neq 0$. Moreover, a \emph{trancendence base} of $L$ over $K$ is a maximal algebraically independent subset of $L$ over $K$. If $A$ is such a base, it follows that $L$ is an algebraic extension of $K(A)$ and in fact for every $x\in L$, there is a polynomial with coefficients in $K[A]$ with $f(x)=0$. 

We can use the next lemma to show that the size of transcendence bases is unique. 

\begin{lemma} (Exchange property) 
Suppose that $A$ and $B$ are transcendence bases of $L$ over $K$ and $b\in B$, then there is some $a\in A$ such that $A'=(A\setminus \{a\})\cup \{b\}$ is a transcendence base of $L$ over $K$. 
\end{lemma} 
\begin{proof} 
Suppose that $f\in K[X_0,\dots,X_{n+1}]$ is non-zero with $f(a_0,\dots,a_n,b)=0$, where $a_0,\dots,a_n\in A$ and $n$ is minimal. Let $a=a_n$ (we could also let $a=a_i$ for any other $i\leq n$). We now write $f(X_0,\dots,X_n,X)=\sum_{j\leq m} f_j(X_0,\dots,
%X_{i-1},X_{i+1},\dots,
X_{n-1},X) X_n^j$ for some $m\in\NNN$ with $f_m(X_0,\dots,X_{n-1},X)\neq0$. Since $n$ is minimal, we have $f_m(a_0,\dots,a_{n-1},b)\neq 0$ and hence $a$ is algebraic over $K(a_0,\dots,a_{n-1},b)$ as witnessed by $f$. 

We first claim that $A'$ is algebraically independent. Otherwise $b$ is algebraic over $K(a_0,\dots,a_{n-1})$. Since $a$ is algebraic over $K(a_0,\dots,a_{n-1},b)$, this would imply that $a$ is algebraic over $K(a_0,\dots,a_{n-1})$, but this contradicts the assumption. 

It remains to show that $A'$ is a maximal algebraically independent set. Since $a$ is algebraic over $A'$, $L$ is an algebraic extension of $L(A')$ and hence $A'$ is maximal. 
\end{proof} 


By successively replacing elements of $A$ with elements of $B$ in a transfinite induction, we obtain that $|A|=|B|$. We can thus define the \emph{transcendence degree} of $L$ over $K$ are unique size of a transcendence base. If $K$ is the algebraic closure of $\QQ$ or $\FF_p$ (depending on the characteristic), then this is simply called the \emph{transcendence degree} of $L$. 

\begin{lemma} 
Any two algebraically closed fields of the same characteristic and transcendence degree are isomorphic. 
\end{lemma} 
\begin{proof} 
Suppose that 
%$A=\{a_i\mid i\in I\}$ and $B=\{b_i\mid i\in I\}$ 
$A$ and $B$ are transcendence base of algebraically closed fields $L$ and $M$ over $K$ and $F\colon A\rightarrow B$ is a bijection. Then $F$ extends uniquely to a ring isomorphism $F\colon K[A]\rightarrow K[B]$ and hence to an isomorphism between the fields of fractions $K(A)$ and $K(B)$. Then $L$ is an algebraic closure of $K(A)$, $M$ is an algebraic closure of $K(B)$ and hence they are isomorphic by the uniqueness of algebraic closures. 
\end{proof} 

Let $ACF_p$ denote the theory of algebraically closed fields of characteristic $p$, where $p=0$ or $p$ is a prime. 

\begin{example} 
$ACF_p$ is not $\aleph_0$-categorical, but $\kappa$-categorical for all uncountable cardinals $\kappa$. This follows from the fact that two algebraically closed fields with characteristic $p$ are isomorphic if and only if their transcendence degree is equal. 
\end{example} 

By Vaught's test (Lemma \ref{Vaught's test}), $ACF_p$ is complete. We now derive some consequences of this fact. 

\begin{lemma} \label{Lefschetz principle} (Lefschetz principle) 
The following conditions are equivalent for any sentence $\varphi$ in the language $\L_R$ of rings. 
\begin{enumerate-(a)} 
\item 
$\varphi$ holds in every algebraically closed field of characteristic $0$. 
\item 
$\varphi$ holds in the complex numbers. 
\item 
$\varphi$ holds in some algebraically closed field of characteristic $0$. 
\item 
There is some $n\in\NNN$ such that for all primes $p>n$, $\varphi$ holds in all algebraically closed fields of characteristic $p$. 
\item 
There are arbitrarily large primes $p$ such that $\varphi$ holds in some algebraically closed field of characteristic $p$. 
\end{enumerate-(a)} 
\end{lemma} 
\begin{proof} 
The implications from (a) to (b) and from (b) to (c) are clear. Assuming (c) we have that $\varphi$ holds in $\CC$ and thus $ACF_0\models \varphi$, since $ACF_0$ is complete. Therefore there is a finite set $\Delta\subseteq ACF_0$ with $\Delta \models \varphi$ and hence $ACF_p\models \varphi$ if $p$ is sufficiently large. 
The implication from (d) to (e) is again clear. 
If (e) holds, we assume towards a contradiction that $ACF_0\not\models \varphi$. Since $ACF_0$ is complete, we have $ACF_0\models \neg\varphi$. By the implication from (a) to (d) for $\neg\varphi$, we have that $\neg\varphi$ holds in all algebraically closed fields of sufficiently large characteristic, contradicting the assumption. 
\end{proof} 


\begin{theorem} 
%\begin{problem} 
Every injective polynomial map from $\CCC^n$ to $\CCC^n$ is surjective.
%\end{problem} 
\end{theorem} 
\begin{proof} 
We first show this for the algebraic closure $\bar{\FF}_p$ of $\FF_p$ for all primes $p$. Suppose that $f\colon (\bar{\FF}_p)^n\rightarrow (\bar{\FF}_p)^n$ is given by polynomials $p_0,\dots,p_k$ 
%$p(X_0,\dots,X_{n-1})$ 
with coefficients $a_0,\dots,a_l\in \bar{\FF}_p$ and it is injective, but some $b\in \bar{\FF}_p$ is not in its range. The subfield $K\subseteq \bar{\FF}_p$ generated by $a_0,\dots,a_k,b$ is finite and the polynomials $p_0,\dots,p_k$ define an injective map $f'\colon K\rightarrow K$. Since $K$ is finite $f'$ is surjective, contradicting the assumption that $b\notin \ran{f'}$. 

Suppose that there is a counterexample that is given by polynomials of degrees at most $d$. Let $\Phi_{n,d}$ be the first-order statement that every injective polynomial map with $n$ inputs and outputs that is given by polynomials of degrees at most $d$ is surjective. Since $AFC_p\models \Phi_{n,d}$, this holds in $\CCC$ as well by Lemma \ref{Lefschetz principle}. 
\end{proof} 






\subsection{Amalgamation classes} 

An $\L$-structure $\MM$ of size $\kappa$ is called \emph{$\kappa$-categorical} if its theory $\Th_\L(\MM)$ is $\kappa$-categorical. The following amalgamation technique (\emph{Fraisse construction}) leads to the construction of $\aleph_0$-categorical structures. 

\begin{definition} 
Suppose that $\L$ is a language and $\MM=(M,\dots)$ is an $\L$-structure. 
\begin{enumerate-(a)} 
\item 
%If $\MM=(M,\dots)$ is an $\L$-structure and 
If $A$ is a subset of $M$, let $\langle A\rangle^{\MM}$ denote the least substructure of $\MM$ whose domain contains $A$. 
\item 
A substructure of $\MM$ is called \emph{finitely generated} if it is equal to $\langle A\rangle^{\MM}$ for some finite subset $A$ of $\MM$. 
%there is a subset $A$ of $M$ such that $\MM$ is equal to the least substructure $\langle A\rangle^{\NN}$ of $\NN$ that contains $A$. 
%A substructure $\MM=(M,\dots)$ of an $\L$-structure $(\NN,\dots)$ is called \emph{finitely generated} if there is a subset $A$ of $M$ such that $\MM$ is equal to the least substructure $\langle A\rangle^{\NN}$ of $\NN$ that contains $A$. 
\item 
The \emph{skeleton} (or \emph{age}) of 
%an $\L$-structure 
$\MM$ is the class of all finitely generated $\L$-structures that are isomorphic to a substructure of $\MM$. 
\item 
$\MM$ is \emph{(ultra-)homogeneous} if any isomorphism between finitely generated substructures can be extended to an automorphism. 
%If $\MM$ is an $\L$-structure and 
%If $\K$ is a class of $\L$-structures, we say that $\MM$ is \emph{$\K$-saturated} if its skeleton is $\K$ and the following condition holds. For all $\AA, \BB\in\K$ and all embeddings $f\colon \AA\rightarrow \MM$ and $g\colon\AA\rightarrow \BB$, there is an embedding $h\colon \BB\rightarrow \MM$ with $f=h\circ g$. 
\end{enumerate-(a)} 
\end{definition} 

Suppose that $C$ is an $\L$-structure and for every isomorphism $f\colon A\rightarrow B$ between finitely generated substructures of $C$ and any $a\in C$, there is an isomorphism $f'\colon A'\rightarrow B'$ that extends $f$, where $A'$ is the substructure of $C$ generated by $A\cup\{a\}$ and $B'$ is a substructure of $C$. We can then extend $f$  to an automorphism of $C$ by inductively applying this condition to the isomorphism and its inverse. Hence this condition is equivalent to homogeneity. Moreover it is a first-order statement if $\L$ is a finite relational language. 

\iffalse 
\begin{lemma} 
If $\L$ is a countable language, then any two countable $\K$-saturated $\L$-structures are isomorphic. 
\end{lemma} 
\begin{proof} 
Suppose that $\MM=(M,\dots)$ and $\NN=(N,\dots)$ are countable $\K$-saturated $\L$-structures and let $\langle a_n\mid n\in\NN\rangle$ and $\langle b_n\mid n\in\NN\rangle$ enumerate $M$ and $N$ without repetitions. We construct a sequence of subsets $A_n$ and $B_n$ of $M$ and $N$ that are domains of finitely generated substructures of $\MM$ and $\NN$ and isomorphisms $f_n\colon A_n\rightarrow B_n$. Let $A_0$ be the domain of any finitely generated substructure of $\MM$. Since $\NN$ has the same skeleton $\K$, there is a finitely generated substructure of $\NN$ with domain $B_0$ and an isomorphism $f_0\colon A_0\rightarrow B_0$. 

Now assume that $A_n$, $B_n$ and $f_n$ are already defined. If $a_n\in A_n$, let $A_n'=A_n$, $B_n'=B_n$ and  $f_n'=f_n$. If $a_n\notin A_n$, we let $A_n'=\langle A_n\cup\{a_n\}\rangle^{\MM}$. Since this appears in the skeleton of $\NN$, $f_n^{-1}\colon B_n\rightarrow A_n'$ is an embedding and $\NN$ is $\K$-saturated, there is an embedding $f_n'\colon A_n'\rightarrow N$ with $f_n'\circ f_n^{-1}=\id_{B_n}$. Let $B_n'$ denote the pointwise image $f_n'(A_n')$ of $A_n'$. 
%and choose some $b_n'\in B$ such that the extension $f_n'$ of $f_n$ with $f_n'(a_n)=b_n'$ is an isomorphism of $A_n'$ to $B_n'=\langle B_n\cup\{b_n'\}\rangle^{\NN}$. 

We now proceed similarly to extend $f_n'\colon A_n'\rightarrow B_n'$ to an isomorphism $f_{n+1}\colon A_{n+1}\rightarrow B_{n+1}$ with $b_n\in B_{n+1}$. 
%If $b_n\in B_n'$, let $A_{n+1}=A_n'$, $B_{n+1}=B_n'$ and  $f_{n+1}=f_n'$. If however $b_n\notin B_n'$, we let $B_{n+1}=B_{n'}\cup\{b_n\}$ and choose some $a_n'\in A$ such that the extension $f_{n+1}$ of $f_n'$ with $f_{n+1}(a_n')=b_n$ from $A_{n+1}=A_n'\cup\{a_n'\}$ to $B_{n+1}$ is an isomorphism. 
Finally, the union $f$ of the functions $f_n$ for all $n\in\NN$ is an isomorphism from $\MM$ to $\NN$. 
\end{proof} 
\fi 

\begin{lemma} \label{uniqueness homogeneous structures with the same skeleton} 
%If $\K$ is the skeleton on $M$, then $M$ is $\K$-homogeneous if and only if it is homogeneous. 
Any two countable homogeneous structures with the same skeleton are isomorphic. 
\end{lemma} 
\begin{proof} 
Assume that $M$ and $N$ are homogeneous $\L$-structures with the same skeleton. Moreover, assume that $f\colon A\rightarrow B$ is an isomorphism between finitely generated substructures of $M$ and $N$ and $a\in M\setminus A$. It is sufficient to show that $f$ can be extended to an isomorphism whose domain contains $a$. We can then obtain an isomorphism by a back-and-forth construction as in Theorem \ref{DLO is aleph0-categorical}. 

Let $A'$ be the substructure of $M$ generated by $A\cup\{a\}$.
% and $f'=h{\upharpoonright}A'$. 
Since $M$ and $N$ have the same skeleton, there is a finitely generated substructure $C'$ of $M$ and an isomorphism $g\colon A'\rightarrow C'$. Let $C$ denote the image of $A$ under $g$. Since $N$ is homogeneous, there is an automorphism $h$ of $N$ that extends $f\circ (g^{-1}{\upharpoonright}C)\colon C\rightarrow B$. Let $B'$ denote the image of $C'$ under $h$. Since $h\colon C'\rightarrow B'$ is an isomorphism extending $f\circ (g^{-1}{\upharpoonright}C)\colon C\rightarrow B$, $h\circ g\colon A'\rightarrow B'$ is an isomorphism extending $f\colon A\rightarrow B$ as required. 
\end{proof} 

\begin{definition} 
A class $\K$ of finitely generated $\L$-structures is called an \emph{amalgamation class} (or \emph{Fraisse class}) if it satifies the following properties. 
\begin{enumerate-(a)} 
\item 
(Heredity) If $\AA\in \K$, then any structure that is isomorphic to a finitely generated substructure of $\AA$ is an element of $\K$. 
\item 
(Joint embedding) 
If $\AA,\BB\in\K$, then there is some $\CC\in \K$ and embeddings $f_0\colon \AA\rightarrow \CC$ and $f_1\colon \BB\rightarrow\CC$. 
\item 
(Amalgamation) 
If $\AA, \BB, \CC\in\K$ and $f_0\colon \AA\rightarrow \BB$, $f_1\colon \AA\rightarrow \CC$ are embeddings, then there is some $\DD\in\K$ and embeddings $g_0\colon \BB\rightarrow \DD$, $g_1\colon \CC\rightarrow \DD$ with $g_0\circ f_0=g_1\circ f_1$. 
\end{enumerate-(a)} 
\end{definition} 

Note that the amalgamation property does not imply the  joint embedding property; the former holds but the latter fails for the class of finite fields. 

\begin{theorem} (Fraisse) \label{existence of Fraisse limits} 
A countable class of $\L$-structures is an amalgamation class if and only if it is the skeleton of a countable homogeneous $\L$-structure. 
\end{theorem} 
\begin{proof} 
Let $\K$ be a countable class of $\L$-structures. We first assume that $\K$ is the skeleton of a homogeneous structure $M$. Heredity and the joint embedding property are obvious. To show amalgamation, suppose that $f_0\colon A\rightarrow B$ and $f_1\colon A\rightarrow C$ are as above. We can assume that $A$, $B$ and $C$ are subsets of $M$ and $f_0=\id_A$. Since $M$ is homogeneous, there is an automorphism $h$ of $M$ that extends $f_1$. Then $D=\langle B\cup h(C)\rangle^M$ amalgamates $B$ and $C$ by the embeddings $\id_B\colon B\rightarrow D$ and $h^{-1}{\upharpoonright}C\colon C\rightarrow D$. 

We now assume that $\K$ is an amalgamation class. We will construct a structure as required whose domain is a subset of $\NN$. Let $\vec{K}=\langle K_l\mid l\in\NNN\rangle$ enumerate $\K$ and let 
%$\vec{f}=
$\langle a_l,b_l,f_l\mid l\in\NNN\rangle$ enumerate all triples of the form $(\vec{a},\vec{b},f)$ such that each triple appears infinitely often, where $\vec{a}$ and $\vec{b}$ are tuples in $\NN$ and $f\colon \NN\rightarrow \NN$ is a partial function with domain $\vec{a}$ and range $\vec{b}$. 
%\footnote{In the previous version of these notes, I made the mistake of working with embeddings $f_l\colon A_l\rightarrow B_l$ instead, but this does not work since there are uncountably many such embeddings. } 
%$f_n\colon A_n\rightarrow B_n$ with $A_n,B_n\in \K$ such that the domains of $A_n$ and $B_n$ are subsets of $\NN$ and each embedding appears infinitely often. 
%Moreover suppose that $A_n$ is generated by $a_{n,0},\dots,a_{n,k_n}$ and this tuple only depend on the structures, but not on the indices. 

We will construct the required structure as a union of an increasing sequence $\vec{C}=\langle C_n\mid n\in\NNN\rangle$ of elements of $\K$, starting with $C_0=K_0$. We now assume that $C_n$ is already constructed. If $n=2l$ is even, we apply the joint embedding property to $C_n$ and $K_l$ and obtain an element $C_{n+1}$ of $\K$. We now assume that $n=2l+1$ is odd. If at least one of $\vec{a}_l$ or $\vec{b}_l$ is not a subset of $C_n$, let $C_{n+1}=C_n$. If on the other hand both $\vec{a}_l$ and $\vec{b}_l$ are subsets of $C_n$, then we apply the amalgamation property to $\id_{A_l}\colon A_l\rightarrow C_n$ and the embedding $f\colon A_l\rightarrow B_l$ that is induced by $f_l$ from the substructure $A_l$ of $C_n$ generated by $\vec{a}_l$ to the stubstructure $B_l$ of $C_n$ generated by $\vec{b}_l$, and obtain an element $C_{n+1}$ of $\K$. Moreover we can choose $C_{n+1}$ such that the embedding from $C_n$ to $C_{n+1}$ is equal to $\id_{C_n}$ and $C_n$ is a substructure of $C_{n+1}$. 

Let $C=\bigcup\vec{C}$. The even steps of the construction ensure that $\K$ is the skeleton of $C$. To prove that $C$ is homogeneous, it is sufficient to show that for any isomorphism $f\colon A\rightarrow B$ between finitely generated substructures of $C$ and any $b\in C\setminus B$, there is a partial isomorphism $f'\colon A'\rightarrow B'$ between finitely generated substructures of $C$ that extends $f$ and whose range contains $b$. So suppose that $n=2l+1$ is odd and $f_l$ 
%\colon A_l\rightarrow B_l$ is equal to
induces the embedding $f\colon A\rightarrow \langle B\cup\{b\}\rangle^C$; such a number $l$ exists because each function appears infinitely often in the enumeration above. In step $n+1$ of the construction the structures $B_l$ and $C_n$ are amalgamated over $A_l$ to $C_{n+1}$, i.e. amalgamation is applied to $f_l\colon A_l\rightarrow B_l$ and $\id_{A_l}\colon A_l\rightarrow C_n$ to obtain an embedding $g\colon B_l\rightarrow C_{n+1}$ with $g\circ f_l=\id_{A_l}$ (since $C_{n+1}$ is chosen such that we obtain the embedding $\id_{C_n}\colon C_n\rightarrow C_{n+1}$). Let $a=g(b)$. Then $(g{\upharpoonright}B_l)^{-1}\colon \langle A\cup\{a\}\rangle^C\rightarrow B_l$ is an isomorphism as required. 
\end{proof} 

In the situation of Lemma \ref{existence of Fraisse limits}, the countable homogeneous structure is unique by Lemma \ref{uniqueness homogeneous structures with the same skeleton} and is called the \emph{Fraisse limit} of $\K$. 
The next result shows that homogeneity is equivalent to amalgamation. 

\begin{problem} 
Show that a class $\K$ of finitely generated $\L$-structures is hereditary and has the joint embedding property if and only if it is the skeleton of a countable $\L$-structure. 
\end{problem} 

\begin{theorem} 
If $\L$ is a finite relational language and $\K$ is a Fraisse class of finite $\L$-structures, then its Fraisse limit is $\aleph_0$-categorical. 
\end{theorem} 
\begin{proof} 
Let $C$ denote the Fraisse limit of $\K$. Let $T_\L$ be a theory that characterizes homogeneity of a structure (as above). Let $\varphi_A$ be the statement that there is a substructure of $C$ that is isomorphic to $A$ for any $A\in\K$ and $\Phi_\K$ the set of all such sentences. Moreover let $\psi_A$ be the statement that no finite substructure of $C$ is isomorphic to $A$ for any finite $\L$-structure that is not an element of $\K$ and $\Psi_\K$ the set of all such sentences. Then $C$ is a model of the theory $T_\K=T_\L\cup \Phi_\K\cup\Psi_\K$. Moreover any model of $T$ has $\K$ as its skeleton and is homogeneous; by uniqueness of Fraisse limits it is isomorphic to $C$. 
\end{proof} 

It is easily checked that the classes of structures in the next two examples are amalgamation classes. 

\begin{example} 
It is easy to see that the following classes of structures are amalgamation classes. 
\begin{enumerate-(a)} 
\item 
The class of finite-dimensional vector spaces over a fixed countable field. 
\item 
The class of finite (undirected) graphs. 
\end{enumerate-(a)} 
\end{example} 

The Fraisse limit of the class of finite (undirected) graphs is called the \emph{random graph}. To understand the structure of this graph, we fix a binary relation symbol $R$ and let $\L=\{R\}$ be the language of (undirected) graphs. Then the random graph $G$ is the unique model of the theory $T_\K$ above up to isomorphism. 

We can also characterize the random graph by the following \emph{extension property}. Let $\L=\{R\}$ and $T_{RG}$ the theory that consists of the axiom $\psi_{\mathrm{sym}}=\forall x,y(xRy\rightarrow yRx)$ and the scheme of axioms $\sigma_{m,n}$ defined as 
% (\emph{extension property}) 
$$\forall x_0,\dots,x_{m-1},y_0,\dots,y_{n-1}(\bigwedge_{i<m\wedge j<n} x_i\neq y_j \rightarrow \exists z (\bigwedge_{i<m} z R x_i\wedge z\neq x_i)\wedge (\bigwedge_{j<n}\neg zR y_j\wedge z\neq y_j)).$$ 

\begin{lemma} 
The theories $T_{RG}$ and $T_\K$ have the same models. 
\end{lemma} 
\begin{proof} 
It is easy to see that the extension property implies that the skeleton is equal to $\K$. Thus it is sufficient to show that for structures with skeleton $\K$, the extension property is equivalent to homogeneity. 

To show that any model $C$ of $T_{RG}$ is homogeneous, suppose that $f\colon A\rightarrow B$ is an isomorphism between finite substructures of $C$ and $a\in C\setminus A$. The extension property allows us to find some $b\in C$ such that the extension $g\colon A\cup\{a\}\rightarrow B\cup\{b\}$ of $f$ with $g(a)=b$ is an isomorphism. 

Conversely suppose that $H$ is a model of $T_\K$ and $A=\{a_0,\dots,a_{m-1},a'_0,\dots,a'_{n-1}\}$ is a finite subset of $H$. Since the skeleton of $H$ is $\K$, there is a finite subset $B=\{b,b_0,\dots,b_{m-1},b'_0,\dots, b'_{n-1}\}$ of $H$ such that the function $f\colon A\rightarrow B$ defined by $f(a_i)=b_i$ and $f(a'_i)=b'_i$ is an embedding and we have $bRb_i$ for all $i<m$ and $\neg bRb'_j$ for all $j<n$. Since $C$ is homogeneous, there is some $a\in A$ and an isomorphism $g\colon B\rightarrow A\cup\{a\}$ that extends $f^{-1}$. Hence $a$ witnesses this instance of the extension property. 
\end{proof} 

To introduce the following random graphs, we fix the following notation. Let $N\in\NNN$ or $N=\NNN$ with the convention that $N\leq \NNN$ and $N$ is identified with $\{0,\dots,N-1\}$ for all $N\in\NNN$. We further say that a property holds for all sufficiently large $N$ if there is some $N_0\in\NNN$ such that this property holds for all $N\geq N_0$. Moreover, recall that a sequence $\vec{\zeta}=\langle\zeta_n\mid n\in I\rangle$ of random variables is mutually independent if $P((\zeta_{m_0},\dots,\zeta_{m_k})\in A\mid (\zeta_{n_0},\dots,\zeta_{n_l})\in B)=P((\zeta_{m_0},\dots,\zeta_{m_k})\in A)$ for all distinct $m_0,\dots,m_k,n_0,\dots,n_l\in I$ and all measurable sets $A$ and $B$. 

The random graph was introduced by Erd\"os and Renyi by the following probabilistic definition. Suppose that $\vec{\xi}=\langle \xi_{m,n}\mid m,n\in\NNN,\ m\neq n\rangle$ is a sequence of mutually independent random variables with Bernoulli distribution with values $0$ and $1$ and $p=0.5$. 
Let $G_N$ denote the random graph with vertex set $N$ such that there is an edge between $i$ and $j$ if and only if $\xi_{i,j}=1$. If $\varphi$ is an $\L$-sentence, we let $p_N(\varphi)=P(G_N\models \varphi)$ denote the probability that $\varphi$ holds in $G_N$. 

\begin{lemma} \label{inequality for random graphs} 
For all $m,n\in\NNN$ and $\epsilon>0$, we have 
%either $p_N(\sigma_{m,n})<\epsilon$ for all sufficiently large $N$ or 
$1-p_N(\sigma_{m,n})<\epsilon$ for all sufficiently large $N$. 
\end{lemma} 
\begin{proof} 
Suppose that $x,x_0,\dots,x_{m-1},y_0,\dots,y_{n-1}$ are distinct elements of $G_N$. Since the random variables in $\vec{\xi}$ are mutually independent, the statement that the extension property holds for these values is $2^{-(m+n)}$. By considering all possible values for $x$, it follows that $\sigma_{m,n}$ is false with probability $1-p_N(\sigma_{m,n})=(1-2^{-(m+n)})^{N-(m+n)}$. But this value converges to $0$ as $N$ increases. 
\end{proof} 

\begin{theorem} ($0$-$1$-law for graphs) 
For any $\L$-sentence $\varphi$ either $\lim_{N\rightarrow\infty} p_N(\varphi)=1$ or $\lim_{N\rightarrow\infty} p_N(\varphi)=0$; the limit is $1$ if and only if $T_{RG}\models\varphi$. 
%Moreover $\lim_{N\rightarrow \infty}p_N(\varphi)=1 \Longleftrightarrow T_{RG}\models \varphi$. 
\end{theorem} 
\begin{proof} 
Since $T_{RG}$ is complete by Vaught's test \ref{Vaught's test}, we have $T_{RG}\models \varphi$ or $T_{RG}\models \neg\varphi$. If $T_{RG}\models \varphi$, then there are $m,n$ with $\models \psi_{\mathrm{sym}}\wedge \sigma_{m,n} \rightarrow \varphi$. Hence $p_N(\varphi)\geq p_N(\sigma_{m,n})$ and thus $\lim_{N\rightarrow\infty} p_N(\varphi)=1$ by Lemma \ref{inequality for random graphs}. Otherwise we have $T_{RG}\models \neg\varphi$ and then $\lim_{N\rightarrow\infty} p_N(\varphi)=\lim_{N\rightarrow\infty} (1-p_N(\neg\varphi))=0$. 
\end{proof} 

Turning again to fields: the proof of the uniqueness of algebraic closures shows that the algebraic closures of $\QQ$ and $\FF_p$ for primes $p$ are homogeneous structures. Hence for each of these fields, the class of its algebraic extensions is an amalgamation class. 

\begin{problem} 
Show directly that the class of finite fields with a fixed prime characteristic $p$ is an amalgamation class by checking the conditions on amalgamation classes. 
\end{problem} 
\iffalse 
\begin{proof} 
It is clear that the class is hereditary. Let $K$ denote a fixed algebraic closure of $\FF_p$ and assume that $\FF_{p^n}$ is a subset of $K$ for all $n\geq 1$. Since every finite field with characteristic $p$ is isomorphic to $\FF_{p^n}$ for some $n\geq 1$ and hence embeds into $K$, the joint embedding property holds. 

To prove amalgamation, we first show that $\FF_{p^m}\subseteq \FF_{p^n}$ if and only if $m|n$ for all $m,n\geq 1$. If $\FF_{p^m}$ is a subfield of $\FF_{p^n}$, then $\FF_{p^n}$ is an $\FF_{p^m}$-vector space of some dimension $d\geq1$. Hence $p^n=(p^m)^d=p^{md}$. Now assume conversely that $n=md$ for some $d\geq1$. Recall that $x\in \FF_{p^l}\Longleftrightarrow x^{p^l}=x$. If $x^{p^m}=x$, then $x^{p^n}=x^{p^{md}}=x^{(p^m)^d}=((x^{p^m}){}^{p^m \dots})^{p^m}=x$. Thus $\FF_{p^m}$ is a subfield of $\FF_{p^n}$. 

SHOW AMALGAMATION 
\end{proof} 
\fi 


Note that all theories that we have considered in this section have computable axiomatizations, i.e. there is an algorithm deciding which sentences are axioms. In fact the theories $DLO$, $AFC_p$ for primes $p$ and the theory of vector spaces over a fixed finite field even have finite axiomatizations. To see that completeness of these theories implies that they are decidable, consider an algorithm that enumerates all provable sentences $\varphi$; then for any sentence $\varphi$, either $\varphi$ or $\neg\varphi$ appears at some stage of the enumeration by completeness. 



 
%%%%%%%%%%%%
%%%%%%%%%%%%
\section{Quantifier elimination and applications} 

In general, we would like to understand the properties of arbitrary definable subsets of a structure. The subsets that are definable without quantifiers are usually easier to study: for a field they are Boolean combinations of sets of zeros of polynomials; for a linear order they are Boolean combinations of (possibly unbounded) intervals. However, in some structures every definable set is of this form: A theory has \emph{quantifier elimination} if every formula is equivalent to some fixed quantifier-free formula in all models of the theory. This yields that every definable sets is in fact definable by a quantifier-free formula. Several such theories are mentioned in Section \ref{subsection: some theories with quantifier elimination}. For instance: 

\begin{itemize} 
\item
DLO has quantifier elimination. 
\item 
$ACF_p$ has quantifier elimination for any characteristic $p$. 
\item 
The theory $T_{RG}$ of the random graph has quantifier elimination. 
\end{itemize} 

%%%%%%%%%%%%
\subsection{A criterion for quantifier elimination} 
We first give the definition of quantifier elimination and then describe a quite useful criterion for proving it. 
%Together with the fact that it is sufficient to check this for primitive existential formulas (see Lemma \ref{quantifier elimination using primitive existential formulas}), quantifier elimination can be shown with this criterion in most cases. 
%We will use the following terminology: Two $\L$-formulas $\varphi$ and $\psi$ are called \emph{equivalent modulo an $\L$-theory $T$} if $T\models \varphi \leftrightarrow \psi$. 

\begin{definition} 
A theory $T$ in a language $\L$ has quantifier elimination if and only if any $\L$-formula $\varphi(x_0,\dots,x_{n-1})$ is equivalent modulo $T$ to a quantifier-free formula $\psi(x_0,\dots,x_{n-1})$, i.e. $$T\models \varphi(x_0,\dots,x_{n-1})\leftrightarrow \psi(x_0,\dots,x_{n-1}).$$ 
\end{definition} 

%We first aim for useful characterizations of quantifier elimination. 
The \emph{atomic diagram} $\mathrm{Diag}(\MM)$ of an $\L$-structure $\MM=(M,\dots)$ is defined as the set of basic $\L_M$-formulas that are true in $\MM$. In the following proof, we will further write $\vec{x}=(x_0,\dots,x_n)$, $\vec{a}=(a_0,\dots,a_n)$. 

The following is a useful test for quantifier elimination; Lemma \ref{quantifier elimination using primitive existential formulas} below shows that it is sufficient to check condition \ref{condition for quantifier elimination} only for primitive existential formulas (defined below). 

\begin{lemma} \label{criterion for quantifier elimination} 
If $\MM=(M,\dots)$ is an $\L$-structure, $T$ is an $\L$-theory and $\varphi(x_0,\dots,x_n)$ is an $\L$-formula, then the following statements are equivalent. 
\begin{enumerate-(a)} 
\item 
There is a quantifier-free $\L$-formula $\psi(\vec{x})$ with $T\models \forall \vec{x}(\varphi(\vec{x})\leftrightarrow\psi(\vec{x}))$. 
\item \label{condition for quantifier elimination} 
If $\MM=(M,\dots)$ and $\NN=(N,\dots)$ are models of $T$ and $\AA=(A,\dots)$ is a substructure of both $\MM$ and $\NN$, then $\MM\models\varphi(a_0,\dots,a_n)\Leftrightarrow\NN\models \varphi(a_0,\dots,a_n)$ for all $a_0,\dots,a_n\in A$. 
\end{enumerate-(a)} 
\end{lemma} 
\begin{proof} 
The first implication follows from the fact that quantifier-free statements about elements of $\AA$ are absolute between $\AA$, $\MM$ and $\NN$. 
We now assume that (b) holds. If $T\models \forall \vec{x} \varphi(\vec{x})$, then let $\psi=\top$ (the true statement) and if $T\models \forall \vec{x} \varphi(\vec{x})$, then let $\psi=\bot$ (the false statement). We can thus assume that both $T\cup\{\varphi(\vec{x})\}$ and $T\cup\{\neg\varphi(\vec{x})\}$ are consistent. We choose new constants  and $(d_0,\dots,d_n)$ and write $\vec{d}=(d_0,\dots,d_n)$. 
Let 
$$\Gamma(\vec{x})=\{\psi(\vec{x})\mid\psi(\vec{x})\in\mathrm{Form}_{\L}\text{ is quantifier-free and }T\cup\{\varphi(\vec{x})\}\models\psi(\vec{x})\}$$ denote the set of quantifier-free consequences of $T\cup\{\varphi(\vec{x})\}$. 

\begin{claim} 
$T\cup \Gamma(\vec{d})\models\varphi(\vec{d})$. 
\end{claim} 
\begin{proof} 
Assuming otherwise, there is a model $\MM$ of $T\cup \Gamma(\vec{d})\cup\{\neg\varphi(\vec{d})\}$. Let $\AA$ be the substructure of $\MM$ that is generated by $\vec{d}^\MM=(d_0^\MM,\dots,d_n^\MM)$. 
We now show that the theory $\Sigma=T\cup\mathrm{Diag}(\AA)\cup\{\varphi(\vec{d})\}$ is consistent. Assuming that it is inconsistent, there are $\psi_0(\vec{d}),\dots,\psi_n(\vec{d})\in\mathrm{Diag}(\AA)$ such that $T\models \bigwedge_{i\leq m}\psi_i(\vec{d}) \rightarrow \neg\varphi(\vec{d})$ and hence $T\models \varphi(\vec{d})\rightarrow \bigvee_{i\leq m}\neg\psi_i(\vec{d})$. So $\bigvee_{i\leq m}\neg\psi_i(\vec{d})\in \Gamma(\vec{d})$. Since $\MM\models \Gamma(\vec{d})$ and $\Gamma(\vec{d})$ only contains quantifier-free formulas, we have $\AA\models \Gamma(\vec{d})$. So $\AA\models \bigvee_{i\leq m}\neg\psi_i(\vec{d})$ and thus there is some $i\leq m$ with $\AA\models \neg\psi_i(\vec{d})$. But this contradicts the fact that $\psi_i(\vec{d})\in \mathrm{Diag}(\AA)$. 
Since we have now shown that $\Sigma$ is consistent, let $\NN$ be a model of $\Sigma$. Then $\NN$ is a model of $T\cup\{\neg\psi(\vec{d})\}$ and we can hence assume that it contains $\AA$ as a substructure. Since $\MM$ is a model of $T\cup\{\psi(\vec{d})\}$ that contains $\AA$ as a substructure and $\vec{d}^\MM=\vec{d}^\NN\in \AA$, this contradicts our assumption (b). 
\end{proof} 

By the completeness theorem, there are finitely many sentences $\theta_0(\vec{d}),\dots,\theta_k(\vec{d})\in \Gamma(\vec{d})$ such that $\Gamma(\vec{d})\models \theta(\vec{d})$ for $\theta(\vec{d})=\bigwedge_{i\leq k}\theta_i(\vec{d})$. Then $T\models\theta(\vec{d})\leftrightarrow \varphi(\vec{d})$ and hence $T\models \forall \vec{x}(\theta(\vec{x})\leftrightarrow \varphi(\vec{x}))$. 
\end{proof} 

We now want to see that it is sufficient to show absoluteness for a very restricted class of existential formulas. 

\begin{definition} 
A formula $\psi$ is called \emph{simple existential} if it is of the form $\exists x \varphi$ for some quantifier-free formula $\varphi$. If $\varphi$ is moreover a conjunction of basic formulas (i.e. formulas of the form $\psi$ or $\neg\psi$, where $\psi$ is atomic), then $\psi$ is called \emph{primitive existential}. 
\end{definition} 

The next lemma shows that primitive existential formulas are sufficient.  

\begin{lemma} \label{quantifier elimination using primitive existential formulas}
An $\L$-theory $T$ has quantifier elimination if and only if every primitive existential formula is equivalent modulo $T$ to a quantifier-free formula. 
\end{lemma} 
\begin{proof} 
Suppose that this condition holds and $\exists x \varphi(x)$ is a simple existential formula. By writing $\varphi$ in disjunctive normal form $\bigvee_{i<n} \varphi_i$, $\exists x \varphi(x)$ is equivalent to the formula $\bigvee_{i<n}\exists x \varphi_i(x)$. By the assumption, this is equivalent modulo $T$ to a quantifier-free formula. 
Given an arbitrary $\L$-formula, we can write it in prenex normal form $Q_0 x_0\dots Q_n x_n\varphi(x_0,\dots,x_n)$, where $Q_i\in\{\forall,\exists\}$ for all $i\leq n$ and $\varphi$ is quantifier-free. We now proceed by induction $n$. If $Q_n=\exists$, then we find a quantifier-free formula $\psi$ that is equivalent modulo $T$ to $\exists x_n \varphi(x_0,\dot,x_n)$ and proceed with $Q_0 x_0\dots Q_{n-1} x_{n-1}\psi(x_0,\dots,x_{n-1})$. If $Q_n=\forall$, then we find a quantifier-free formula $\psi$ that is equivalent modulo $T$ to $\exists x_n \neg\varphi(x_0,\dot,x_n)$ and proceed with $Q_0 x_0\dots Q_{n-1} x_{n-1}\neg\psi(x_0,\dots,x_{n-1})$. 
\end{proof} 

Combined with Lemma \ref{criterion for quantifier elimination}, this is a useful test for quantifier elimination. Many theories $T$ have the additional property of having \emph{algebraically prime models} (see \cite[p. 78]{MR1924282}): For any substructure $\AA$ of a model of $T$, there is a model $\AA^*$ of $T$ that extends $\AA$ such that every embedding $f\colon \AA\rightarrow \NN$ into a model of $T$ extends to an embedding $g\colon \AA^*\rightarrow \NN$. For example, for a subring $\AA$ of a model of $T$, a theory that contains the field axioms, its quotient field $\AA^*$ has this property. It is easy to see that for such theories, it suffices to check condition \ref{condition for quantifier elimination} for $\NN=\AA$ (and only for primitive existential formulas by Lemma \ref{quantifier elimination using primitive existential formulas}). 

Quantifier elimination is often used as a tool to prove completeness. For instance, this follows immediately if the theory has a prime structure, which is defined as follows. 

\begin{definition} 
Suppose that $T$ is an $\L$-theory and $\MM$ is a model of $T$. 
\begin{enumerate-(a)} 
\item 
$\MM$ is a \emph{prime structure} for $T$ if it can be embedded into every model of $T$. 
\item 
$\MM$ is a \emph{prime model} for $T$ if it can be elementarily embedded into every model of $T$. 
\end{enumerate-(a)} 
\end{definition} 

If a theory $T$ has quantifier elimination and a prime structure $\MM$, then this is also a prime model by absoluteness of quantifier-free formulas. So clearly $T$ is complete. 
%In this case there is an embedding $f\colon \MM\rightarrow \NN$ into any model $\NN$ of $T$ that is necessarily elementary and in particular, $T$ is complete. 
Some examples of this are mentioned in the beginning of the next section. 






%%%%%%%%%%%%
\subsection{Some theories with quantifier elimination} \label{subsection: some theories with quantifier elimination} 

Using the criterion in the previous section, it is easy to show quantifier elimination for the following theories: 

\begin{example} 
The theory $DLO$ of dense linear orders without end points and the theory $T_{RG}$ of the random graph have quantifier elimination. Since the are $\aleph_0$-catergorical, they have prime models are therefore complete. 
\end{example} 

%\begin{example} 
%The theory $T_{RG}$ of the random graph has quantifier elimination and is complete. 
%\end{example} 

%We will see later that the theory $T_{RG}$ of the random graph has quantifier elimination. 

We now show quantifier elimination for vector spaces and algebraically closed fields. 

\begin{lemma} \label{quantifier elimination for infinite-dimensional vector spaces} 
The theory of infinite-dimensional vector spaces over a fixed field $K$ has quantifier elimination. 
\end{lemma} 
\begin{proof} 
Suppose that $V_0$ and $V_1$ are $K$-vector spaces of infinite dimension that both contain a $K$-vector space $V$ of infinite dimension. %We can assume that $\dim(A)=\omega$. 
Suppose that $\psi=\exists x \varphi(x,x_0,\dots,x_n)$ is a simple existential formula that holds in $V_0$ for some $a_0,\dots,a_n\in V$, witnessed by some $a\in V_0$. If $a\in V$ then $\psi$ holds in $V_1$, so suppose that $a\in V_0\setminus V$. 
Let $b\in V_1\setminus V$ and let $A$ and $B$ be subspaces of $V_0$ and $V_1$ of $K$-dimension $\omega$ that contain $a_0,\dots,a_n$ with $a\in A$ and $b\in B$. Since $\varphi$ is quantifier-free, $A\models \varphi(a,a_0,\dots,a_n)$. Since there is an isomorphism from $A$ to $B$ that maps each $a_i$ to $b_i$ and $a$ to $b$, we have $B\models \varphi(b,a_0,\dots,a_n)$. Hence $V\models \exists x \varphi(x,a_0,\dots,a_n)$. 
\end{proof} 

With a slight adjustment, this can also be shown for arbitrary vector spaces over a fixed infinite field. 

\begin{problem} 
Show that for infinite fields $K$, Lemma \ref{quantifier elimination for infinite-dimensional vector spaces} holds for the theory of arbitrary $K$-vector spaces. 
\end{problem} 

For both finite and infinite fields $K$, it is easy to see that the theory of infinite $K$-vector spaces has a prime model. Therefore it is complete. 

\begin{theorem} 
For any characteristic $p$, the theory $ACF_p$ has quantifier elimination.  
\end{theorem} 
\begin{proof} 
Suppose that $L$ and $M$ are algebraically closed fields of characteristic $p$ and $R$ is a substructure of both, i.e. a subring. Then the quotient fields of $R$ in $L$ and $M$ are isomorphic and hence we can assume that they are equal and denote this field by $K$. Moreover the proof of uniqueness of algebraic closures shows that there is an isomorphism between the algebraic closures of $K$ in $L$ and $M$ that is the identity on $K$. We can thus assume that there is an an algebraic closure $\bar{K}$ of $K$ that is contained in both $L$ and $M$. 

We now assume that some primitive existential formula $\exists x \varphi(x,x_0,\dots,x_n)$ holds in $L$ for some $a_0,\dots,a_n\in R$.
%$\varphi$ is a conjunction of basic formulas. 
Moreover assume that this is witnessed by some $a\in L$. If $a\in \bar{K}$, then $\varphi(a,a_0,\dots,a_n)$ holds in $M$, since $\varphi$ is quantifier-free and hence absolute. We can thus assume that $a\notin \bar{K}$. 
%We write $\vec{x}=(x_0,\dots,x_n)$ and $\vec{a}=(a_0,\dots,a_n)$. 
Suppose that $\varphi(x)=(\bigwedge_{i<k}f_i(x)=0)\wedge(\bigwedge_{j<l}g_j(x)\neq 0)$, where $f_i$ and $g_j$ are polynomials over $R$. Then $f_i(a)=0$ for all $i<k$. Since $a$ is not algebraic over $K$, each $f_i$ is the zero polynomial. 
Since $g_j\neq 0$ in $R[X]$ for all $j<l$, the polynomial $x\cdot \prod_{j<l} g_j(x)+1$ is not constant and hence has a root $b$ in $M$. Hence $g_j(b)\neq 0$ for all $j<l$ and thus $M\models \varphi(b,a_0,\dots,a_n)$. \footnote{The last step is different in the proof of \cite[Theorem 3.3.11]{MR2908005}. } 
\end{proof} 

Again, the theory $ACF_p$ has a prime model and we thus have another proof of its completeness. Moreover, quantifier elimination can be used to prove the following result. 

\begin{problem} 
Show that definable subsets of algebraically closed fields are finite or co-finite. 
\end{problem} 


We now give some examples for using quantifier elimination for $ACF_p$ to obtain elegant proofs of some results in field theory. 
%We now get two easy applications of quantifier elimination for $ACF_p$. 

\begin{theorem} (Hilbert's Nullstellensatz) 
Suppose that $K$ is an algebraically closed field and $f_0,\dots,f_n\in K[X]$ such that $I=(f_0,\dots,f_n)$ is proper ideal in $K[X_0,\dots,X_n]$ (i.e. $1\notin I$). Then there are $a_0,\dots,a_n\in K$ such that $f_i(a_0,\dots,a_n)=0$ for all $i\leq n$. 
\end{theorem} 
\begin{proof} 
By Zorn's Lemma applied to the set of proper ideals $J$ in $K[X]$ that contain $I$, there is a maximal ideal $J$ in $K[X_0,\dots,X_n]$ containing $I$. Since $J$ is a maximal ideal, $L=K[X_0,\dots,X_n]/J$ is a field. Moreover we can identify $K$ with a subfield of $L$ by identifying $a\in K$ with $a+I$. 
For each $i\leq k$ we have $f_i(X_0+J,\dots,X_n+J)=f_i(X_0,\dots,X_n)+J=J$, since $f_i(X_0,\dots,X_n)\in I$. Hence $\exists x_0,\dots,x_n \bigwedge_{i\leq n} f_i(x_0,\dots,x_n)=0$ holds in $L$ and thus also in its algebraic closure $\bar{L}$. By quantifier elimination for $ACF_p$, this is also true in $K\prec \bar{L}$ and thus $f_0,\dots,f_n$ have a common zero in $K$ as required. 
\end{proof} 

If $K$ is an algebraically closed field, then a subset $S$ of $K^n$ is called \emph{constructible} if it is a finite Boolean combination of zero sets of polynomials in $K[X_0,\dots,X_{n-1}]$ and their complements. By quantifier elimination, every definable subset of $K^n$ is constructible (and conversely). 

\begin{lemma} (Chevalley) 
If $K$ is an algebraically closed field, $S$ is a constructible subset of $K^m$ and $f\colon K^m\rightarrow K^n$ is a polynomial function, then $f(S)$ is constructible. 
\end{lemma} 
\begin{proof} 
Suppose that $f$ is given by the polynomials $g_0,\dots,g_{n-1}\in K[X_0,\dots,X_{m-1}]$. Then $(a_0,\dots,a_{n-1})\in f(S)\Leftrightarrow \exists x_0,\dots,x_{m-1}\in K \bigwedge_{i<n} g_i(x_0,\dots,x_{m-1})=a_i$. Thus $f(S)$ is definable and hence constructible. 
\end{proof} 

We now want to see that quantifier elimination for $ACF_0$ and $ACF_p$ implies that for these theories, the algebraic closure is equal to its model-theoretic version, which we now define. 
%We now define a model-theoretic version of algebraic closures. 

\begin{definition} 
Suppose that $\MM=(M,\dots)$ is an $\L$-structure, $\varphi(x)$ is an $\L$-formula and $A\subseteq M$. 
\begin{enumerate-(a)} 
\item 
Let $\varphi(\M)=\{x\in M\mid\MM\models \varphi(x)\}$. 
\item 
$\varphi$ is called \emph{algebraic over $\MM$} if $\varphi(\MM)$ is finite. 
\item 
An element $x\in M$ is called \emph{algebraic over $A$} if $\MM\models \varphi(x)$ for some algebraic $\L_A$-formula. 
\item 
The \emph{algebraic closure} $\acl(A)=\acl^\MM(A)$ of $A$ in $\MM$ is the set of all algebraic elements of $M$ over $A$. 
\item 
$A$ is \emph{algebraically closed} in $\MM$ if $\acl(A)=A$. 
\end{enumerate-(a)} 
\end{definition} 

If $A$ is a subset of an algebraically closed field $K$ of characteristic $p$, then $\acl_K$ is equal to the standard algebraic closure (from algebra) by quantifier elimination for $ACF_p$. To see this, suppose that $a\in\acl(A)$ and $\varphi(x)$ is a quantifier-free formula with parameters in $A$ and only finitely many solutions including $a$. By quantifier elimination, we can assume that it is quantifier-free and replace it by a logically equivalent formula $\bigvee_{i\in I} \bigwedge_{i\in J_i} \varphi_{i,j}(x)$ with basic formulas $\varphi_{i,j}$. Then $a$ satisfies the formula $\psi(x)=\bigwedge_{i\in J_i} \varphi_{i,j}(x)$ for some $i\in I$ -- the same is true when the basic inequalities $\varphi_{i,j}(x)$ are removed. This conjunction of polynomial equations can be rewritten as a single polynomial equation, showing that $a$ is in the algebraic closure in the usual sense (as defined in field theory). 

%We will use the algebraic closure in later proofs... 

Finally, we would like to mention some examples of theories without quantifier elimination. 

\begin{problem} 
Show that the theories of the structures $(\NNN, S)$ and $(\NNN, +)$ do not satisfy quantifier elimination, where $S$ is the successor function. 
\end{problem} 







\section{Various applications of types}

A \emph{type} is a maximal consistent (with respect to a theory) set of formulas with a fixed finite number of (fixed) free variables. So it generalizes the notion of a theory by allowing free variables. This notion is important for many results in model theory, and we will see some applications in this section. In Section \ref{section: realizing and omitting types}, we prove the omitting types theorem and study the spaces $S_n(T)$ of $n$-types with respect to a fixed theory. We then use types to characterize the following properties of theories. 

\begin{itemize} 
\item 
$\aleph_0$-categoricity (Theorem \ref{section: ryll-nardzewski}) 
\item 
The existence of countable saturated models (Theorem \ref{characterization of small theories}) 
\item 
The existence of prime models (Theorem \ref{characterization of theories with prime models}) 
\end{itemize} 

We will also use it to prove Vaught's \emph{never two theorem} (Theorem \ref{Vaught's never two theorem}), which states that a countable complete theory cannot have exactly two countable models up to isomorphism. 

In the last sections, we will use types to study algebraic closure in saturated structures, a connection between homogeneity and quantifier elimination, and facts about saturated structures. 



\subsection{Realizing and omitting types} \label{section: realizing and omitting types} 

We will see that every type can be realized, but only certain types can be omitted. 
We first fix the following notation: if $\MM=(M,\dots)$ is an $\L$-structure and $A\subseteq M$, let $\MM_A$ denote the $\L_A$-structure induced by $\MM$. 

\begin{definition} 
Suppose that $T$ is a theory in a language $\L$, $\MM=(M,\dots)$ an $\L$-structure, $A$ a subset of $M$ and $a_0,\dots,a_{n-1}\in M$. 
\begin{enumerate-(a)} 
\item 
An \emph{$n$-type} (with respect to $T$) is a maximal set $t(x_0,\dots,x_{n-1})$ of formulas $\varphi(x_0,\dots,x_{n-1})$ that is consistent with $T$, for some fixed distinct variables $x_0,\dots,x_{n-1}$. 
\item
$S_n(T)$ denotes the set of $n$-types; if $T$ is complete then $S_0(T)=\{T\}$. 
\item \label{types over a set}
The type $$\tp(a_0,\dots,a_{n-1}/A)=\tp^\MM(a_0,\dots,a_{n-1}/A)$$ of $a_0,\dots,a_{n-1}$ over $A$ is defined as the set $\{\varphi(x_0,\dots,x_{n-1})\in L_A\mid \MM_A\models \varphi(a_0,\dots,a_{n-1})\}$. 
We will also write $$\tp(a_0,\dots,a_{n-1})=\tp^\MM(a_0,\dots,a_{n-1})=\tp^\MM(a_0,\dots,a_{n-1}/\emptyset).$$ 
%Moreover let $\tp(a_0,\dots,a_{n-1})=\tp^\MM(a_0,\dots,a_{n-1})=\tp^\MM(a_0,\dots,a_{n-1}/\emptyset)$ denote the $n$-type of a tuple $(a_0,\dots,a_{n-1})$ over the empty set. 
\item 
An $n$-type in $\L_A$ is \emph{realized} in $\MM$ if it is equal to $\tp(a_0,\dots,a_{n-1})$ for some tuple $(a_0,\dots,a_{n-1})$, otherwise it is {omitted} in $\MM$. 
\item \label{definition of types over a set} 
Let $S_n(A)=S_n^\MM(A)=S_n(\Th_{\L_A}(\MM))$ denote the set of $n$-types over $A$. 
%as in \ref{types over a set}. 
\end{enumerate-(a)} 
\end{definition} 

A consistent set of formulas in free variables $x_0,\dots,x_{n-1}$ is often called a \emph{partial type}. 
Note that the types in \ref{definition of types over a set} do not depend on the structure $\MM$ but only on its theory and moreover, they are not necessarily realized in $\MM$. 

Here are two examples where the realization or non-realization of a type tells us something interesting about the model. 

\begin{example} 
If $\NN=(\NN,0,+,\cdot,<)$ is the structure of natural numbers, then $$\Sigma(x)=\{n<x\mid n\in \NNN\}$$ is consistent, so it can be extended to an $\L_\NNN$-type that is not realized in $\NN$. However, in every non-standard model of $\Th(\NN)$ some $1$-type extending $\Sigma(x)$ is realized. 
\end{example} 

\begin{example} 
Consider the set $\Sigma(x)=\{p(x)\neq 0\mid p\in \ZZ[X]\text{ is non-constant}\}$ in the language $\L_R$ of rings. This is realized in the field extensions of $\QQ$ with transcendence degree at least $1$. In particular it is not realized in the field $\bar{\QQ}$ of algebraic complex numbers, but is realized in $\CCC$. 
\end{example} 

We first show that all types can be simultaneously realized in a sufficiently large model. 

\begin{lemma} \label{realizing types} 
Every $\L$-structure $\MM=(M,\dots)$ has an elementary extension $\NN=(N,\dots)$ in which all types in $\L_M$ over $\MM$ (with respect to $\Th(\MM)$) are realized. 
%If moreover $\MM$ has size $\kappa$, where $\kappa$ is an infinite cardinal, then $\NN$ can be chosen to have size $\kappa$.\footnote{I forgot to mention in the lecture that we can choose $|\NN|=\kappa$; this is used below. }  
\end{lemma} 
\begin{proof} 
Let $S=\{t_i(x_0^i,\dots,x_{n_i}^i)\mid i\in I\}$ be obtained by changing the variables in each such type so that they are pairwise different. 
%be a set of types in which all such types occur with pairwise distinct variables. 
% $x_k^i\neq x_l^j$ for $i\neq j$, $k\leq n_i$ and $l\leq n_j$. 
Then for any $\varphi_0(x_0^i,\dots,x_{n_i}^i),\dots,\varphi_k(x_0^i,\dots,x_{n_i}^i)\in t_i(x_0^i,\dots,x_{n_i}^i)$, this type also contains the sentence $\exists y_0^i,\dots,y_{n_i}^i \bigwedge_{j\leq k}\varphi_j(y_0^i,\dots,y_{n_i}^i)$ and hence there is a tuple in $\MM$ that satisfies $\varphi_j$ for all $j\leq k$. Hence the theory 
$$\Th_{\L_M}(\MM)\cup \bigcup_{i\in I} t_i(c_0^i/x_0^i,\dots,c_{n_i}^i/x_{n_i}^i)$$ 
is finitely satisfiable and hence satisfiable by some structure $\NN$. Each type $t_i(x_0^i,\dots,x_{n_i}^i)$ is realized by $c_0^\NN,\dots,c_{n_i}^\NN$. Moreover $\NN$ is an elementary extension of $\MM$ since it is a model of $\Th_{\L_M}(\MM)$. 
\end{proof} 

The following condition on a type implies that it is realized in every model. We will phrase this more generally for consistent sets of formulas in $n$ fixed free variables, which are called \emph{partial types}. 

\begin{definition} 
Suppose that $T$ is an $\L$-theory and $\Sigma(\vec{x})$ is a set of $\L$-formulas with at most the free variables $\vec{x}=(x_0,\dots,x_{n-1})$. A formula $\varphi(\vec{x})$ \emph{isolates} $\Sigma(\vec{x})$ if 
\begin{enumerate-(a)} 
\item 
$\varphi(\vec{x})$ is consistent with $T$ and 
\item 
$T\models \forall \vec{x} (\varphi(\vec{x})\rightarrow \sigma(\vec{x}))$ for all $\sigma(\vec{x})$ in $\Sigma(\vec{x})$. 
\end{enumerate-(a)} 
Then $\Sigma(\vec{x})$ is called \emph{isolated} with respect to $T$. 
% a type and there is such a formula $\varphi(x)$, it is called an \emph{isolated type}. 
\end{definition} 

Note that for any type $\Sigma(x)$ that is isolated by a formula $\varphi(x)$ with respect to a complete theory $T$, the sentence $\exists x \varphi(x)$ is in $T$ and therefore $\Sigma(x)$ is satisfied in every model of $T$. The following converse to this statement shows that non-isolated types with respect to countable theories can always be omitted. 

\begin{theorem} (Omitting types theorem) \label{omitting types}
Suppose that $T$ is a countable consistent theory in a countable language $\L$ and $\Sigma_0(x_0,\dots,x_{n_0})$, $\Sigma_1(x_0,\dots,x_{n_1})$, \dots is a sequence of sets of formulas that are consistent with $T$. If no $\Sigma_i$ is isolated in $T$, then then $T$ has a model that omits all $\Sigma_i$. 
\end{theorem} 
\begin{proof} 
We first assume that there is only a single set $\Sigma=\Sigma_0$ and moreover $n_0=0$. We thus work with a set $\Sigma(x)$ of formulas in a free variable $x$ and we can assume that it is consistent with $T$. 

We choose a countable set $C$ of new constants (that are not in $\L$) and call the extended language $\L(C)$. We extend $T$ to a consistent complete theory $T^*$ with the following properties.
\begin{enumerate-(a)} 
\item 
$T^*$ is a Henkin theory with respect to the set $C$ of constants, i.e. for all $\L(C)$-formulas, $\psi(x)$ there is a constant $c\in C$ with $(\exists x \psi(x)\rightarrow \psi(c))\in T^*$. 
\item 
For all $c\in C$ there is some $\sigma(x)\in \Sigma(x)$ with $\neg\sigma(c)\in T^*$. 
\end{enumerate-(a)} 

We now construct an increasing sequence $T=T_0\subseteq T_1\subseteq \dots$ of theories such that $T_{n+1}\setminus T_n$ is finite for all $n\in\omega$. Let $\langle c_i\mid i\in\NNN\rangle$ be an enumeration of $C$ and $\langle \psi_i(x)\mid i\in\NNN\rangle$ an enumeration of all $\L(C)$-formulas. 

Now assume that $T_{2i}$ is already constructed and let $c\in C$ be a variable that does not occur in $T_{2i}\cup\{\psi_i(x)\}$. Then the theory $T_{2i+1}=T_{2i}\cup\{\exists x\ \psi_i(x)\rightarrow \psi_i(c)\}$ is consistent. Since we can form the conjunction of the finitely many new formulas, the theory $T_{2i+1}$ is equivalent to $T\cup\{\delta(c_i,\vec{c})\}$ for some $\L$-formula $\delta(x,\bar{y})$ and a tuple $\vec{c}$ in $C$ that does not contain $c_i$. Since the formula $\exists \vec{y} \delta(x,\vec{y})$ does not isolate $\Sigma(x)$, there is some $\sigma(x)\in \Sigma(x)$ such that the formula $(\exists \vec{y} \delta(x,\vec{y}))\wedge\neg \sigma(x)$ is consistent with $T$. Hence the theory $T_{2_i+2}=T_{2i+1}\cup\{\neg\sigma(c_i)\}$ is consistent. 

Let $T^*$ be a completion of the theory $\bigcup_{n\in\NNN} T_n$ and $\MM$ a model of $T^*$. By Tarski's test (Lemma \ref{Tarskis test}), the set $\{c^\MM\mid c\in C\}$ is the domain of an elementary substructure of $\MM$. Moreover $\Sigma(x)$ is omitted in $\MM$ by the second property above. 

The proof for the case of a set of formulas $\Sigma_0(x_0,\dots,x_{n_0})$ in $n_0$ variables is virtually the same as the previous proof. Finally, the proof for the general case in which we have sets $\Sigma_i(x_0,\dots,x_{n_i})$ for all $i\in\NNN$ is very similar and only the order of the construction is changed. In each step, we consider one of the sets $\Sigma_i(x_0,\dots,x_{n_i})$ such that each is considered infinitely often, and do the same as above in each step.  
\end{proof} 


\begin{example} 
$S_n(T)$ is finite for the theory $T=DLO$ in the language $\L=\{<\}$ and all $n\in\NNN$. 

For each $\vec{x}=(x_0,\dots,x_{n-1})$, we consider all formulas of the form $\varphi(\vec{x})=\bigwedge_{i<j<n} \varphi_{i,j}(x_i,x_j)$, where $\varphi_{i,j}(x,y)$ is of the form $x<y$, $x=y$ or $x>y$. 

We first claim that every type $t(\vec{x})=t(x_0,\dots,x_{n-1})$ is generated by some formula of this form. To see this, note that each formula in $t(\vec{x})$ is equivalent to a quantifier-free formula by quantifier elimination for DLO. Thus it is equivalent to a formula $\bigvee_{i<m} \bigwedge_{ j<n_i} \psi_{i,j}(\vec{x})$ in disjunctive normal form, where each formula $\psi_{i,j}(\vec{x})$ is basic (the formulas $\top$ and $\bot$ are atomic by definition). Since there are only finitely many such formulas, their conjunction $\psi(\vec{x})$ isolates $t(\vec{x})$. We can assume that $\psi(\vec{x})$ is in disjunctive normal form. We fix a consistent conjunction $\chi(\vec{x})$ in $\psi(\vec{x})$ and write it equivalently (by re-ordering the components) as $\bigwedge_{i<j<n,\ k<l_{i,j}} \chi_{i,j}^k(x_i,x_j)$, where each formula $\chi_{i,j}^k(x_i,x_j)$ is basic. 
%=\bigwedge_{ j<n_i} \psi_{i,j}(\vec{x})$ in $\psi(\vec{x})$. 
% that is consistent. 
Since $\chi(\vec{x})$ implies $\psi(\vec{x})$, it isolates $t(\vec{x})$ as well (and is hence equivalent to $\psi(\vec{x})$). 
We now replace each sub-formula $\bigwedge_{k<l_{i,j}}\chi_{i,j}^k(x_i,x_j)$ by a stronger formula $\varphi_{i,j}(x_i,x_j)$ of the form $x_i<x_j$, $x_i=x_j$ or $x_i>x_j$ in $t(\vec{x})$ and thus $\chi(\vec{x})$ is replaced by the formula $\varphi(\vec{x})=\bigwedge_{i<j<n} \varphi_{i,j}(x_i,x_j)$, which isolated $t(\vec{x})$. 

We now claim that each such formula $\varphi(\vec{x})$ isolates an $n$-type. To see this, suppose that $s(\vec{x})$ and $t(\vec{x})$ are distinct $n$-types that contain $\varphi(\vec{x})$. As we have shown, $s(\vec{x})$ and $t(\vec{x})$ are isolated by formulas $\psi(\vec{x})=\bigwedge_{i<j<n} \psi_{i,j}(x_i,x_j)$ and $\theta(\vec{x})=\bigwedge_{i<j<n} \theta_{i,j}(x_i,x_j)$ as above. But since $s(\vec{x})$ and $t(\vec{x})$ are distinct, these formulas are distinct. So one is different from and hence inconsistent with $\varphi(\vec{x})$, contradicting the assumption. 

It follows that there are exactly $n!$ many $n$-types. This argument (reducing types to these formulas) can also be shown by a back-and-forth construction. 
\end{example} 

\begin{example} \label{example: types in algebraically closed fields} 
$S_n(T)$ is countably infinite for the theory $T=ACF_p$ in the language $\L_R$ for any characteristic $p$ and all $n\geq1$. 

We will write $\FF_0=\QQ$ here to simplify the notation; so $\FF_p$ is the (unique) prime structure for $ACF_p$.  We first determine the $1$-types. The set $\{f(x)\neq 0\mid f\in \FF_p[X]\setminus \FF_p\}$ generates a $1$-type $t(x)$, 
which is called the \emph{trancendental $1$-type}, 
%$t(x)=\{f(x)\neq 0\mid f\in \FF_p[X]\setminus \FF_p\}$ is actually a type (i.e. maximal consistent), 
since for any field extensions $\FF_p(x)$ and $\FF_p(y)$ of $\FF_p$ with both $x$ and $y$ transcendental, there is an isomorphism $h\colon \FF_p(x)\rightarrow \FF_p(y)$ with $h(x)=y$ and hence $x$ and $y$ have the same type; it follows that this type is equal to $t(\vec{x})$.  

Any other type $t(x)$ is realized by an algebraic number $x$. Let $g(X)\in\FF_p[X]$ be the minimal polynomial of $x$, i.e. the unique polynomial of least degree with $g(x)=0$ and leading coefficient $1$. We claim that the formula $g(x)=0$ isolates $t(x)$. To see this, take any two fields $\FF_p(x)$ and $\FF_p(y)$ with $g(x)=g(y)=0$. Then $\FF_p(x)\cong \FF_p[X]/(g)\cong \FF_p(y)$ by an isomorphism that sends $x$ to $y$ (this calculation is done further above). Hence $x$ and $y$ have the same type. 

Now suppose that $t(\vec{x})=t(x_0,\dots,x_{n-1})$ is an $n$-type. For all $i< n$, the $1$-type of $x_i$ over $\FF_p(x_0,\dots,x_{i-1})$ induced by $t(\vec{x})$ determines $\FF_p(x_0,\dots,x_i)$ up to isomorphism. Except for the transcendental $1$-type, all $1$-types over $\FF_p(x_0,\dots,x_{i-1})$ are determined by minimal polynomials over $\FF_p(x_0,\dots,x_{i-1})$ as for $\FF_p$. Hence there are countably many such types. 
\end{example} 

The $n$-types in $ACF_p$ are completely described for all $n>1$ in \cite[Example 4.1.14]{MR1924282}. It is shown there that there is a bijective correspondence between $n$-types over a subset $A$ of the field $K$ and prime ideals in $L[X_0,\dots,X_{n-1}]$, where $L$ is the subfield of $K$ generated by $A$. 

\begin{problem} 
Determine $S_n(T)$ for the following theories and all $n\geq1$. 
\begin{enumerate-(a)} 
\item 
The theory $T_{RG}$ of the random graph. 
\item 
The theory of $\QQ$-vector spaces. 
\end{enumerate-(a)} 
\end{problem} 

The next observation states that every type with respect to a complete theory $T$ is finitely satisfiable in every model of $T$. 

\begin{observation} 
If $\MM$ is a model of a complete $\L$-theory $T$, then a set of $\L$-formulas $\Sigma(\vec{x})$ in the free variables $\vec{x}=(x_0,\dots,x_{n-1})$ is consistent with $T$ (i.e. it is satisfied in a model of $T$ with an assignment of the variables in $\vec{x}$) if and only if it is finitely satisfiable in $\MM$. 
\end{observation}  
\begin{proof} 
We first assume that $\Sigma(\vec{x})$ is consistent with $T$. To show that $\Sigma(\vec{x})$ is finitely satisfiable in $M$, let $\varphi_0(\vec{x})$,\dots, $\varphi_{n-1}(\vec{x})$ by formulas in $t(\vec{x})$. Since $T$ is complete, we have that $\exists x_0,\dots,x_{n-1} \bigwedge_{i<n}\varphi(\vec{x})\in T$; since $\MM$ is a model of $T$ this holds in $\MM$. 

Conversely, assume that $t(\vec{x})$ is finitely satisfiable in $\MM$. Let $\L'=\L\cup\{c_0,\dots,c_{n-1}\}$, where $\vec{c}=(c_0,\dots,c_{n-1})$ are new constants, and $T'=T\cup\{\varphi(\vec{c})\mid \varphi(\vec{x})\in t(\vec{x})\}$. Since $t(\vec{x})$ is finitely satisfiable in $\MM$, for every finite subset $T_0$ of $T'$ there is a choice of values for $\vec{c}$ that defines an expansion of $\MM$ to an $\L'$-structure that is a model of $T_0$. By the compactness theorem $T$ has a model $\MM'$ and in this model, $t(\vec{x})$ is realized by $\vec{c}^{\MM'}=(c_0^{\MM'},\dots,c_{n-1}^{\MM'})$. 
\end{proof} 

Moreover, it is immediate that types are preserved by elementary embeddings. 

\begin{observation} \label{elementary embeddings preserve types} 
Suppose that $\MM=(M,\dots)$ and $\NN=(N,\dots)$ are $\L$-structures and $f\colon \MM\rightarrow \NN$ is an elementary embedding. Then for all $a_0,\dots,a_{n-1}\in M$, we have $\tp^\MM(a_0,\dots,a_{n-1})=\tp^\NN(f(a_0),\dots,f(a_{n-1}))$. 
\end{observation}

If a theory is categorical in some infinite cardinal, then it can be easily seen as follows that all types are realized in every model. 

\begin{observation} 
If $\kappa$ is an infinite cardinal, $\L$ a language of size at most $\kappa$ and $T$ a $\kappa$-categorical $\L$-theory, then any model $\MM$ of size $\kappa$ realizes all types. 
\end{observation} 
\begin{proof} 
Assume that some type $t(\vec{x})=t(x_0,\dots,x_{n-1})$ is not realized in $\MM$. By Lemma \ref{realizing types}, there is an elementary extension $\NN$ of $\MM$ of size $\kappa$ that realizes $t(\vec{x})$. but $\MM$ and $\NN$ are isomorphic by $\kappa$-categoricity and hence have to realize the same types by Lemma \ref{elementary embeddings preserve types}. 
\end{proof} 

To study the types of an arbitrary complete $\L$-theory $T$, we now introduce a topology on $S_n(T)$ for fixed $n\geq 1$; with this topology $S_n(T)$ is called the \emph{space of $n$-types}. The basic open sets in $S_n(T)$ are defined as 
$$[\varphi(\vec{x})]=\{t(\vec{x})\in S_n(T)\mid \varphi(\vec{x})\in t(\vec{x})\}$$ 
for any $\L$-formula $\varphi(\vec{x})$ and $\vec{x}=(x_0,\dots,x_{n-1})$. This collection forms a base for a topology since it is closed under finite intersections by the following lemma. 

\begin{lemma} 
Fix some $n\geq 1$. 
\begin{enumerate-(a)} 
\item 
$[\varphi]\subseteq [\psi]$ $\Leftrightarrow$ $T\models \varphi \rightarrow \psi$ 
\item 
$[\varphi]= [\psi]$ $\Leftrightarrow$ $T\models \varphi \leftrightarrow \psi$ 
\item 
$[\bot]=\emptyset$ 
\item
 $[\top]=S_n(T)$ 
\item 
$[\varphi]\cap[\psi]= [\varphi\wedge \psi]$ 
\item 
$[\varphi]\cup[\psi]= [\varphi\vee \psi]$ 
\item 
$S_n(T)\setminus [\varphi]= [\neg \varphi]$ 
\end{enumerate-(a)} 
\end{lemma} 
\begin{proof} 
These facts can all be checked very easily. For example, to show that $[\varphi]\cup[\psi]= [\varphi\vee \psi]$, assume that $t(\vec{s})$ is an $n$-type in $[\varphi(\vec{x})]\cup[\psi(\vec{x})]$; we can assume that $t(\vec{x})\in[\varphi(\vec{x})]$, so $\varphi(\vec{x})\in t(\vec{x})$ and hence $\varphi(\vec{x})\vee\psi(\vec{x})\in t(\vec{x})$ since $t(\vec{x})$ is complete. If on the other hand $t(\vec{x})\in[\varphi(\vec{x})\vee\psi(\vec{x})]$, then $\varphi(\vec{x})\in t(\vec{x})$ or $\psi(\vec{x})\in t(\vec{x})$ since $t(\vec{x})$ is complete. 
\end{proof} 

%Note that we could also define $S_0(T)=\{T\}$ when $T$ is complete. 

\begin{lemma} 
For any complete theory $T$ and any $n\in\NNN$, the type space $S_n(T)$ is a zero-dimensional compact Hausdorff space. 
\end{lemma} 
\begin{proof} 
A space is zero-dimensional (by definition) if it has a base that consists of sets that are both open and closed. This holds for the base given above, since $S_n(T)\setminus [\varphi]=[\neg\varphi]$ is open and hence $[\varphi]$ is closed. To see that $S_n(T)$ is Hausdorff, let $s(\vec{x})$ and $t(\vec{x})$ be distinct elements of $S_n(T)$. Then there is a formula $\varphi(\vec{x})$ with $\varphi(\vec{x})\in s(\vec{x})$ and$\neg\varphi(\vec{x})\in t(\vec{s})$. Now the basic open sets $[\varphi]$ and $[\neg\varphi]$ separate $s(\vec{x})$ and $t(\vec{x})$. It remains to show that $S_n(T)$ is compact. To see this, suppose that $\{ [\varphi_n(\vec{x})]\mid n\in\NNN\}$ covers $S_n(T)$. If no finite subset covers $S_n(T)$, the the set $\{\neg\varphi_n(\vec{x}) \mid n\in\NNN\}$ is finitely satisfiable and hence it can be extended to a type $t(\vec{x})$. But this contradicts the fact that $t(\vec{x})\in [\varphi_n(\vec{x})]$ for some $n\in\NNN$. 
\end{proof} 

If the language $\L$ is countable, then there are only countably many formulas and hence $S_n(T)$ has a countable base. In this case, $S_n(T)$ can be embedded into the \emph{Cantor space} $2^\NNN$ of infinite binary sequences. It is equipped with the metric $d$ defined by $d(x,y)=2^{-n}$ for all $x\neq y$, where $n$ is least with $x(n)\neq y(n)$. 

\begin{problem}\footnote{This additional fact will not be used later. } 
Show that for any complete theory $T$ in a countable language, the space $S_n(T)$ is homeomorphic to a subset of the Cantor space. 
\end{problem} 

This can either be done by using an enumeration $\langle \varphi_n(x_0,\dots,x_{n-1})\mid n\in\NNN\rangle$ of all formulas with free variables in $x_0,\dots,x_{n-1}$ to define a metric on $S_n(T)$, or using the fact that any zero-dimensional compact Hausdorff space with a countable base is homeomorphic to a subset of the Cantor space. 

In some of the following proofs, we will need the notion of an embeddings between subsets of structures that is elementary with respect to the whole structures. 

\begin{definition} 
Suppose that $A_0$ and $B_0$ are subsets of $\L$-structures $\AA=(A,\dots)$ and $\BB=(B,\dots)$. A function $f\colon A_0\rightarrow B_0$ is called \emph{elementary} if 
%it preserves the truth of formulas, i.e. 
for all formulas $\varphi(\vec{x})=\varphi(x_0,\dots,x_{n-1})$ and $\vec{a}=(a_0,\dots,a_{n-1})\in A^n$, we have $\AA\models \varphi(\vec{a}) \Longleftrightarrow \BB\models \varphi(f(\vec{a}))$. 
\end{definition} 

We will also use the fact that elementary embeddings induce continuous maps between the type spaces over subsets of structures. 

\begin{lemma} \label{induced maps on type spaces} 
Suppose that $A_0$ and $B_0$ are subsets of $\L$-structures $\AA$ and $\BB$, $f\colon A_0\rightarrow B_0$ is an elementary embedding and $n\geq 1$. Then the function $S_n(f)\colon S_n^\BB(B_0)\rightarrow S_n^\AA(A_0)$ defined by 
$$S_n(f)(t(\vec{x}))=\{\varphi(\vec{x},\vec{a})\mid \vec{a}\in A_0^{<\omega},\ \varphi(\vec{x},f(\vec{a}))\in t(\vec{x})\}$$ 
for $\vec{x}=(x_0,\dots,x_{n-1})$ is well-defined, continuous and surjective. 
\end{lemma} 
\begin{proof} 
Since $f$ is elementary, the set $S_n(f)(t(\vec{x}))$ is finitely satisfiable and hence satisfiable for each $n\geq1$. Since $t(\vec{x})$ is complete, the same holds for $S_n(f)(t(\vec{x}))$ and thus $S_n(f)$ is well-defined. It is moreover continuous since the preimage of $[\varphi(\vec{x},\vec{a})]$ is the set $[\varphi(\vec{x},f(\vec{a})]$. To see that it is surjective, it is sufficient to show that the set $\Phi=\{\varphi(\vec{x},f(\vec{a}))\mid \varphi(\vec{x},\vec{a})\in t(\vec{x})\}$ is finitely satisfiable for every $t(\vec{x})\in S_n^\AA(A_0)$. So let $\varphi_i(\vec{x},\vec{a})\in t(\vec{x})$ for $i<k$. Since $\AA\models \exists \vec{x} \bigwedge_{i<k}\varphi_i(\vec{x},\vec{a})$ and $f$ is elementary, we have $\BB\models \exists \vec{x} \bigwedge_{i<k}\varphi_i(\vec{x},f(\vec{a}))$. Hence $\bigwedge_{i<k}\varphi_i(\vec{x},f(\vec{a}))$ is satisfiable and thus $\Phi$ is finitely satisfiable. 
\end{proof} 

Note that the map $S_n(f)$ is the identity on $S_n(T)$ if $A_0$ and $B_0$ are empty. We will use the following special cases of the previous lemma. 
\begin{itemize} 
\item 
If  $\AA=\BB$, $A_0\subseteq B_0$ and $f=\id_{A_0}$, then we write $t(\vec{x}){\upharpoonright}A_0=S_n(f)(t(\vec{x}))$ and call this the \emph{restriction} of $t(\vec{x})$ to $A_0$.
\item 
If $f\colon A_0\rightarrow B_0$ is an isomorphism, then we write $f=S_n(f)^{-1}\colon S_n(A_0)\rightarrow S_n(B_0)$ for the isomorphism induced by $f$. 
\end{itemize} 






\subsection{$\aleph_0$-categorical theories, small theories and Vaught's never two theorem} \label{section: ryll-nardzewski} 

We will characterize properties of complete theories by the number of types and properties of isolated types. 

Note that the isolated types are exactly the isolated points in the type space: if a type $t(\vec{x})$ is isolated by a formula $\varphi(\vec{x})$, then it is the unique element of the basic open set $[\varphi(\vec{x})]$. 
% and hence $t(\vec{x})$ is an isolated point in the type space $S_n(T)$. 
If on the other hand $t(\vec{x})$ is an isolated point in $S_n(T)$ and thus $\{t(\vec{x})\}$ is open, let $\varphi(\vec{x})$ be a formula with $\{t(\vec{x})\}=[\varphi(\vec{x})]$. If $\varphi(\vec{x})$ does not isolate $t(\vec{z})$, then there is a formula $\psi(\vec{x})\in t(\vec{x})$ such that $T\cup\{\varphi(\vec{x}),\neg\psi(\vec{x})\}$ is consistent, contradicting the assumption. 

%does not follow from $T\cup\{\varphi(\vec{x})\}$, so there is a type $u(\vec{x})$ that contains both $\varphi(\vec{x})$ and $\neg\psi(\vec{x})$, but by our assumption $t(\vec{x})$ is the unique $n$-type that contains $\varphi(\vec{x})$. 

\begin{lemma} \label{characterization of finite type spaces} 
The following conditions are equivalent for a complete theory $T$. 
\begin{enumerate-(a)} 
\item 
$S_n(T)$ is finite for all $n\geq 1$. 
\item 
Every $n$-type is isolated. 
\item 
There are only finitely many formulas of the form $\varphi(\vec{x})=\varphi(x_0,\dots,x_{n-1})$ up to equivalence in $T$, where $\varphi(\vec{x})$ and $\psi(\vec{x})$ are equivalent in $T$ if $T\models \forall \vec{x}\ \varphi(\vec{x})\leftrightarrow \psi(\vec{x})$. 
\end{enumerate-(a)} 
\end{lemma} 
\begin{proof} 
If $S_n(T)$ is finite, then every $n$-type is isolated since the space is Hausdorff. 
If conversely every $n$-type is isolated, then $S_n(T)$ consists only of isolated points and hence it is finite by compactness. 

We now assume again that $S_n(T)$ is finite and let $t_0(\vec{x})$, \dots, $t_k(\vec{x})$ enumerate $S_n(T)$, where $\vec{x}=(x_0,\dots,x_{n-1})$. If $\psi(\vec{x})$ and $\theta(\vec{x})$ are not equivalent in $T$, then there is an $n$-type $t(\vec{x})$ that contains $\psi(\vec{x})$ and not $\theta(\vec{x})$ or conversely. Hence any two formulas are equivalent if and only if they are contained in the same types, and thus there are only finitely many equivalence classes. 
If on the other hand there are only $k$ many formulas with $n$ free variables up to equivalence in $T$, then there are at most $2^k$ many $n$-types. 
\end{proof} 

\begin{definition} 
Suppose that $\kappa$ is an infinite cardinal and $T$ is an $\L$-theory. A model $\MM=(M,\dots)$ of $T$ is called \emph{$\kappa$-saturated} if for every subset $A$ of $M$ of size strictly less than $\kappa$, every $1$-type in $S_1^\MM(A)$ is realized in $\MM$. 
% for all $n\geq 1$. 
Moreover $\MM=(M,\dots)$ is called \emph{saturated} if it is $|M|$-saturated. 
\end{definition} 

\begin{lemma} \label{unique saturated model}
Suppose that $T$ is a complete theory and $\kappa$ is an infinite cardinal. Then there is at most one $\kappa$-saturated model of size $\kappa$ up to isomorphism. 
\end{lemma} 
\begin{proof} 
Suppose that $\MM=(M,\dots)$ and $\NN=(N,\dots)$ are $\kappa$-saturated models of $T$ of size $\kappa$. Suppose that $\langle a_i\mid i<\kappa\rangle$ and $\langle b_i\mid i<\kappa\rangle$ enumerate $M$ and $N$. We construct an increasing sequence $\langle f_j\mid j\leq\kappa\rangle$ of elementary functions $f_j\colon A_j\rightarrow B_j$ between finite subsets of $\MM$ and $\NN$ (note that for if $\vec{c}$ is an enumeration of $A_j$, this means that $\tp^\MM(\vec{c})=\tp^\NN(f(\vec{c})$)). Let $f_0=A_0=B_0=\emptyset$ and let $f_j=\bigcup_{i<j} f_i$, $A_j=\bigcup_{i<j} A_i$ and $B_j=\bigcup_{i<j}B_i$ for limits $j\leq\kappa$. Now suppose that $f_j\colon A_j\rightarrow B_j$ is already constructed. To extend the domain of $f_j$ to $A_j\cup\{a\}$, let $t(x)=\tp^\MM(a_j/A_j)$. Since $\NN$ is $\kappa$-saturated and $|B_j|<\kappa$, there is some $b\in N$ with $\tp^\MM(b/B_j)=f(t(x))$. Then the extension $f_j'\colon A_j\cup\{a_j\}\rightarrow  B_j\cup\{b\}$ of $f_j$ that is defined by $f_j'(a_j)=b$ is elementary. To extend the range of $f_j'$ to $B_j\cup\{b_j,b\}$, let $t(x)=\tp^\MM(b_j/B_j\cup\{b\})$. Since $\MM$ is $\kappa$-saturated and $|A_j\cup\{a_j\}|<\kappa$, there is some $a\in N$ with $\tp^\MM(a/A_j\cup\{a\})=f_j^{-1}(t(x))$. Then the extension $f_{j+1}\colon A_j\cup\{a_j,a\}\rightarrow  B_j\cup\{b,b_j\}$ of $f_j'$ that is defined by $f_{j+1}(a)=b_j$ is elementary. Finally let $A_{j+1}=A_j\cup\{a_j,a\}$ and $B_{j+1}=B_j\cup\{b,b_j\}$. Then $f_\kappa$ is an isomorphism between $\MM$ and $\NN$. 
\end{proof} 

\begin{theorem} [Engeler, Ryll-Nardzewski, Svenonius] \label{Ryll-Nardzewski}

The following conditions are equivalent for a complete theory $T$ with infinite models. 
\begin{enumerate-(a)} 
\item 
$S_n(T)$ is finite for all $n\geq 1$. 
\item 
Every countable model of $T$ is $\omega$-saturated. 
\item 
$T$ is $\aleph_0$-categorical. 
\end{enumerate-(a)} 
\end{theorem} 
\begin{proof} 
We first assume that $S_n(T)$ is finite for all $n\geq 1$. By Lemma \ref{characterization of finite type spaces}, there are $k$ many $\L$-formulas of the form $\varphi(x_0,\dots,x_n)$ up to equivalence modulo $T$. To see that any countable model $\MM=(M,\dots)$ of $T$ is $\omega$-saturated, suppose that $A=\{a_0,\dots,a_{n-1}\}$ is a subset of $M$. Then there are at most $k$ many $\L_A$-formulas of the form $\psi(x)$ up to equivalence modulo $\Th_{\L_A}(\MM)$. It follows from Lemma \ref{characterization of finite type spaces} that every $1$-type $t(x)$ with respect to $\Th_{\L_A}(\MM)$ is isolated by an $\L_A$-formula $\psi(x)$. Then $\exists x \psi(x)$ follows from $\Th_{\L_A}(\MM)$ and hence $t(x)$ is realized in $\MM$.  

If every countable model of $T$ is $\aleph_0$-saturated, then $T$ is $\aleph_0$-categorical by Lemma \ref{unique saturated model}. 

Now suppose that $T$ is $\aleph_0$-categorical. 
%If there is an infinite sequence $\langle \varphi_i(x)\mid i\in\NNN\rangle$ of pairwise inequivalent (modulo $T$) formulas of the form $\varphi(x_0,\dots,x_{n-1})$ for some $n\in\NNN$, then there is some non-isolated type $t(x_0,\dots,x_{n-1})$ by Lemma \ref{characterization of finite type spaces}. 
If $S_n(T)$ is infinite, then is must contain a non-isolated type by compactness of $S_n(T)$. 
By the omitting types theorem (Theorem \ref{omitting types}), there is a countable model $\MM$ of $T$ in which $t(x)$ is not realized. But there is also a countable model $\NN$ in which $t(x)$ is realized, so $T$ cannot be $\aleph_0$-categorical. 
\end{proof} 

\begin{definition} 
\begin{enumerate-(a)} 
\item 
If $\MM$ is an $\L$-structure, let $\Aut(\MM)$ denote the group of automorphisms of $\MM$. 
\item 
If $A$ is any set, a subgroup $G$ of the group $\Sym(A)$ of permutations of $A$ is called \emph{oligomorphic} of for every $n\in\NNN$, there are only finitely many orbits of the action of $G$ on $A^n$. 
\end{enumerate-(a)} 
\end{definition} 

\begin{theorem} \label{oligomorphic groups and categoricity}
The following conditions are equivalent for any countable complete theory $T$ with infinite models. 
\begin{enumerate-(a)} 
\item \label{oligomorphic1}
$T$ is $\aleph_0$-categorical. 
\item \label{oligomorphic2}
If $A$ is any countable model of $T$, then $\Aut(A)$ is oligomorphic. 
\item \label{oligomorphic3}
$T$ has a countable model $A$ such that $\Aut(A)$ is oligomorphic. 
\item \label{oligomorphic4}
Some countable model $A$ of $T$ realizes only finitely many $n$-types for each $n\in\NNN$. 
\end{enumerate-(a)} 
\end{theorem} 
\begin{proof} 
First assume that \ref{oligomorphic1} holds. Then every countable model $A$ of $T$ is $\omega$-saturated by the Theorem of Engeler, Ryll-Nardzewski and Svenonius \ref{Ryll-Nardzewski}. 
Now the proof of Lemma \ref{unique saturated model} shows that for any two tuples $\vec{a}=(a_0,\dots,a_{n-1})$ and $\vec{b}=(b_0,\dots,b_{n-1})$ of the same type, there is an automorphism $h\colon A\rightarrow A$ with $h(a_i)=b_i$ for all $i<n$. Since there are only finitely many $n$-types by Theorem \ref{Ryll-Nardzewski}, $\Aut(A)$ isoligomorphic. 

The implication from \ref{oligomorphic2} to \ref{oligomorphic3} is clear. The implication from \ref{oligomorphic3} to \ref{oligomorphic4} is also clear because automorphisms preserve types. 

Now assume that \ref{oligomorphic4} holds and $t_0(\vec{x})$, \dots, $t_k(\vec{x})$ are all $n$-types realized in $\MM$, where $\vec{x}=(x_0,\dots,x_{n-1})$. 
%Then for each $i\leq k$, there is a formula $\varphi_i(\vec{x})\in t_i(\vec{x})\setminus \bigcup_{j\leq k,\ j\neq i} t_j(\vec{x})$ (let $\psi_j(\vec{x})\in t_i(\vec{x})\setminus t_j(\vec{x})$ and let $\varphi_i(\vec{x})$ be their conjunction). 
We claim that there are no more $n$-types (with respect to $T$). This implies \ref{oligomorphic1} by the Theorem of Engeler, Ryll-Nardzewski and Svenonius \ref{Ryll-Nardzewski}. 
Towards a contradiction, suppose that $t(\vec{x})$ is an $n$-type that is not realized in $\MM$. Then there is a formula $\varphi(\vec{x})\in t(\vec{x})\setminus \bigcup_{i\leq k} t_i(\vec{x})$ (let $\psi_i(\vec{x})\in t(\vec{x})\setminus t_i(\vec{x})$ and $\varphi(\vec{x})=\bigwedge_{i\leq k}\psi_i(\vec{x})$). Then $\exists \vec{x}\ \varphi(\vec{x})$ is in $t(\vec{x})$ and hence in $T$. So $\varphi(\vec{x})$ is satisfiable in $\MM$, but this contradicts the fact that it is not in any $t_i(\vec{x})$. 
\end{proof} 

\begin{definition} 
A theory $T$ is \emph{small} if $S_n(T)$ is countable for all $n\geq 1$. 
\end{definition} 

For example $S_n(ACF_p)$ is countably infinite for all $n\geq 1$ and all primes $p$ or $p=0$. The following equivalence implies that $ACF_p$ has a countable $\omega$-saturated model. This is the algebraically closed field with transcendence degree $\omega$. 

\begin{theorem} \label{characterization of small theories}
A countable complete theory $T$ with infinite models is small if and only if it has a countable $\omega$-saturated model. 
\end{theorem} 
\begin{proof} 
If all types are realized in a countable model, then there are only countably many types. 
Conversely, assume that $S_n(T)$ is countable for all $n\geq 1$. We construct an \emph{elementary chain} $\langle \MM_i\mid i\in\NNN\rangle$ where $\MM_i=(M_i,\dots)$, i.e. $\MM_i\prec\MM_j$ for all $i<j$. It follows from Tarski's Test (Lemma \ref{Tarskis test}) that the union of these models is an elementary extension of all of them (this is called \emph{Tarski's chain lemma}). 
Let $\MM_0$ be any countable model of $T$. If $\MM_i$ is already constructed, there is a countable elementary extension $\MM_i'$ of $\MM_i$ such that all types with respect to $\Th_{M_i}(\MM_i)$ are realized in $\MM_i'$. Since there are only countably many types over finite subsets of $\MM_i$, let $\MM_{i+1}\prec \MM_i'$ be an elementary substructure containing $\MM_i$ such that all $n$-types over finite subsets of $\MM_i$ (i.e. types in $S_n(\Th_{\L_A}(\MM_i))$ for finite subsets $A$ of $M$) are realized in $\MM_{i+1}$ for all $n\geq1$ by the L\"owenheim-Skolem theorem. 
Finally, the union $\MM$ of the structures $\MM_i$ for $i\in\NN$ is $\omega$-saturated. 
\end{proof} 

We have now given equivalent formulations of the statements that every countable model of a given theory is saturated and of the existence of a countable saturated model. Recall that one of our main questions is the number of models of size $\kappa$ of a given complete theory for all infinite cardinals $\kappa$. The next result shows that it is impossible to have exactly two countable models up to isomorphism. 

\begin{theorem} [Vaught] \label{Vaught's never two theorem} 
A countable complete theory $T$ cannot have exactly two countable models up to isomorphism. 
\end{theorem} 
\begin{proof} 
If $T$ is not small, then $S_n(T)$ is uncountable for some $n\geq 1$. Since every type is realized in some countable model of $T$, there are uncountably many non-isomorphic countable models of $T$. We can thus assume that $T$ is small and not $\aleph_0$-categorical. Then there is a non-isolated type $t(\vec{x})$ by Lemma \ref{characterization of finite type spaces} and the theorem of Engeler, Ryll-Nardzewski and Svenonius \ref{Ryll-Nardzewski}. By the omitting types theorem, $t(\vec{x})$ is omitted in some countable model $\MM$ of $T$. Moreover $T$ has a countable $\omega$-saturated model $\NN$ of $T$ in which $t(\vec{x})$ is realized by some tuple $\vec{a}=(a_0,\dots,a_{k-1})$. Let $A=\{a_0,\dots,a_{k-1}\}$. Since $T$ is not $\aleph_0$-categorical, $S_n(T)$ is infinite for some $n\geq 1$. Moreover we have that each type in $S_n(T)$ extends to a type in $S_n(\Th_{\L_A}(\NN))$ by the argument in the proof of Theorem \ref{realizing types}. Hence $\Th_{\L_A}(\NN)$ is not $\aleph_0$-categorical and hence there is a countable model $\KK$ of $\Th_{\L_A}(\NN)$ that is not $\omega$-saturated. Moreover the models $\MM$, $\NN$ and $\KK$ are different. 
\end{proof} 

However, it is not too hard to find countable complete theories with exactly $n$ countable models up to isomorphism -- see \cite[Exercise 4.3.5]{MR2908005}. 

\begin{problem} 
Show that for every $n>2$, there is a countable complete theory with exactly $n$ countable models up to isomorphism. 
\end{problem} 

What about uncountable models? We will hopefully get to this later... 




\subsection{Prime models} 

How can we determine if a given theory $T$ has a prime model, i.e. one that can be elementarily embedded into every other model of $T$? A prime model is by definition the smallest possible model of a theory. For example, the algebraic closures of $\QQ$ and $\FF_p$ are prime models by quantifier elimination, while $\QQ$ and $\FF_p$ are prime structures. 
% for these theories. 

We first want to derive properties of prime models -- the following is a necessary condition. We will then consider criteria how to determine if a prime model exists. An example of a countable complete theory with quantifier elimination, but no prime model can be found in \cite[p. 60]{MR2908005}. 

\begin{definition} 
A model $\MM=(M,\dots)$ of a theory $T$ is \emph{atomic} if every type $\tp^\MM(\vec{a})$ realized in $\MM$ is isolated (in $S_n(T)$). 
\end{definition} 

\begin{theorem} \label{characterization of prime models} 
Suppose that $T$ is a theory in a countable language $\L$ with infinite models. Then a model $\MM=(M,\dots)$ of $T$ is prime if and only if it is countable and atomic. 
\end{theorem} 
\begin{proof} 
Any prime model of $T$ is countable, since $T$ has countable models by the L\"owenheim-Skolem theorem. Since all non-isolated types can be omitted by the omitting types theorem, only isolated types can be realized in prime models. 

Now assume that $\MM=(M,\dots)$ is a countable atomic model of $T$ and $\NN=(N,\dots)$ is any countable model of $T$. We construct a sequence $\langle f_n\mid n\in\NNN\rangle$ of elementary functions $f_n\colon A_n\rightarrow B_n$ between finite subsets of $M$ and $N$. The map $f_0$ with empty domain is elementary with respect to $\MM$ and $\NN$, since both are models of $T$. 
It is sufficient to show that for every elementary function $f\colon A\rightarrow B$ between finite subsets $A$ and $B$ of $M$ and $N$ and every $a\in M$, there is some $b\in N$ such that the extension of $f$ that maps $a$ to $b$ is elementary. 
To this end, suppose that $A=\{a_0,\dots,a_{k-1}\}$ and $\vec{a}=(a_0,\dots, a_{k-1})$. 
%Let $t(x)=\tp^\MM(a/A)$. 
Since $\MM$ is atomic, there is a formula $\varphi(x,\vec{x})$ that isolates $\tp^\MM(a,\vec{a})$. Then $\varphi(x,\vec{a})$ isolates the type $t(x)=\tp^\MM(a/A)$. Since $f\colon A\rightarrow B$ is elementary, the formula $\varphi(x,f(a_0),\dots,f(a_{k-1}))$ isolates $f(t(x))$. Hence $\exists x\ \varphi(x,f(a_0),\dots,f(a_{k-1}))$ holds in $\NN$, witnessed by some $b\in N$ with type $f(t(x))$. Thus the extension of $f$ that sends $a$ to $b$ is elementary. 
%Let $A_{n+1}=A_n\cup\{a\}$, $B_{n+1}=B_n\cup\{b\}$ and let $f_{n+1}\colon A_{n+1}\rightarrow B_{n+1}$ extend $f_n$ with $f_{n+1}(a)=b$. 
%Now suppose that $f_n\colon A_n\rightarrow B_n$ is constructed, $A_n=\{a_0,\dots,a_{k-1}\}$ and $a\in M$. Let $t(x)=\tp^\MM(a/A)$. Let $\varphi(x,x_0,\dots,x_{k-1})$ be a formula that isolates $\tp^\MM(a,a_0,\dots,a_{k-1})$. Then $\varphi(x,a_0,\dots,a_{k-1})$ isolates $t(x)=\tp^\MM(a/A)$. Since $f_n\colon A_n\rightarrow B_n$ is elementary, $\varphi(x,f_n(a_0),\dots,f_n(a_{n-1}))$ isolates $f(t(x))$. Hence $\exists x\ \varphi(x,f_n(a_0),\dots,f_n(a_{n-1}))$ holds in $\NN$, witnessed by some $b\in N$. Let $A_{n+1}=A_n\cup\{a\}$, $B_{n+1}=B_n\cup\{b\}$ and let $f_{n+1}\colon A_{n+1}\rightarrow B_{n+1}$ extend $f_n$ with $f_{n+1}(a)=b$. 
\end{proof} 

If prime models exists, then they are unique. This can be easily shown by a variation of the previous proof.  

\begin{lemma} 
All prime models of a theory $T$ in a countable language with infinite models are isomorphic. 
\end{lemma} 
\begin{proof} 
As in the proof of Theorem \ref{characterization of prime models}, we can construct a sequence of elementary maps between finite subsets of two prime models, but extend them on both sides in each step to get an isomorphism. 
\end{proof} 

We now want to characterize the existence of prime models by a property of the type spaces. To state this criterion, recall that a subset $X$ of a topological space $S$ is \emph{dense} if and only if $X\cap U\neq \emptyset$ for every nonempty open subset $U$ of $S$. 

\begin{lemma} \label{characterization of theories with prime models}
A theory $T$ in a countable language with infinite models has a prime model if and only if for every $n\geq1$, the isolated $n$-types are dense in $S_n(T)$. 
\end{lemma} 
\begin{proof} 
Suppose that $\MM=(M,\dots)$ is a prime model of $T$; then $\MM$ is infinite. Moreover, suppose that $\varphi(\vec{x})=\varphi(x_0,\dots,x_{n-1})$ is an $\L$-formula that is consistent with $T$. Then the sentence $\exists \vec{x}\varphi(\vec{x})$ is in $T$ and hence holds in $\MM$, witnessed by some $\vec{a}=(a_0,\dots,a_{n-1})\in M^n$. Then $tp^\MM(\vec{a})\in [\varphi(\vec{x})]$ and this type is isolated by Theorem \ref{characterization of prime models}. So the isolated $n$-types are dense. 

Now suppose that for each $n\geq1$, the isolated $n$-types are dense in $S_n(T)$. It is sufficient to find a countable model $\MM$ of $T$ in which only isolated types are realized. By the omitting types theorem \ref{omitting types}, it suffices to show that for each $n\geq1$, the set $\Sigma_n(\vec{x}_n)=\{\neg \varphi(\vec{x}_n)\mid \varphi(\vec{x}_n)\text{ isolates an $n$-type}\}$ is either inconsistent, or consistent but not isolated (then all $\Sigma_n(\vec{x}_n)$ can be simultaneously omitted in a countable model of $T$). Towards a contradiction, suppose that for some $n\geq1$, $\psi(\vec{x}_n)$ is a formula that is consistent with $T$ and isolates $\Sigma_n(\vec{x}_n)$. Then $T\models \psi(\vec{x}_n)\rightarrow \neg\varphi(\vec{x}_n)$ for all formulas $\varphi(\vec{x}_n)$ that isolate a type. Hence no isolated $n$-type is in $[\psi(\vec{x}_n)]$ for any $n\geq1$. But this contradicts the assumption that the isolated $n$-types are dense. 
\end{proof} 

Here is another sufficient condition for the existence of prime models. Its proof idea will show up again later... 

\begin{lemma} 
If $T$ is a complete theory and $S_n(T)$ is countable, then the isolated $n$-types are dense. In particular, any small theory $T$ in a countable language with infinite models has a prime model. 
\end{lemma} 
\begin{proof} 
If the isolated $n$-types are not dense, then there is a formula $\varphi(\vec{x})$ such that $[\varphi(\vec{x})]$ does not contain any isolated type. Using this fact, we can inductively construct a 'tree' $\langle \varphi_s(\vec{x})\mid s \in 2^{<\omega}\rangle$ of formulas consistent with $T$ such that $\varphi_0(\vec{x})=\varphi(\vec{x})$ and for each $s\in 2^{<\omega}$, there is a formula $\psi(\vec{x})$ with $\varphi_{s^\smallfrown 0}(\vec{x})=\varphi_s(\vec{x})\wedge\psi(\vec{x})$ and $\varphi_{s^\smallfrown 1}=\varphi_s(\vec{x})\wedge\neg \psi(\vec{x})$. Such a formula $\psi(\vec{x})$ exists because $\varphi_s(\vec{x})$ does not isolate a type (because $[\varphi(\vec{x})]$ does not contain any isolated types). We now obtain $2^\omega$ distinct $n$-types, since for every $z\in 2^\omega$ the set $T\cup\{\varphi_{z{\upharpoonright}k}(\vec{x})\mid k\in\NNN\}$ is finitely satisfiable and can hence be extended to a type $t_z(\vec{x})$. The second claim now follows from Lemma \ref{characterization of theories with prime models}. 
\end{proof} 





\subsection{Algebraic closure in saturated structures} 

%The motivation for this section is to see how some of the previously studied properties and notions are related -- in particular $\aleph_0$-categoricity, homogeneity, quantifier elimination, amalgamation and algebraic closure. We study some implications without determining the complete relationships between these notions. 

The following modified notion of homogeneity is useful because it also makes sense for structures that are not homogeneous. For instance, the linear order $(\QQ\cap [0,1),<)$ is $\omega$-homogeneous but not homogeneous. 

\begin{definition} 
A structure $\MM=(M,\dots)$ is \emph{$\kappa$-homogeneous} if for every elementary map $f$ defined on a subset $A$ of $M$ with $|A|<\kappa$ and for any $a \in M$, there is some $b\in M$ such that $g = f\cup \{(a,b)\}$ is elementary. 
\end{definition} 

Note that $g = f\cup \{(a,b)\}$ is elementary if and only if $\tp^\MM(b/B)=f(\tp^\MM(a/A))$ -- simply check that equality of types means that the truth of formulas is preserved (we could say that \emph{$b$ realizes the type $f(\tp^\MM(a/A))$ over $B$} -- this makes sense as it is a type in the language $\L_B$). 

An interesting fact about this property is that it holds for all $\kappa$-saturated structures. 

\begin{lemma} \label{saturated structures are homogeneous} 
Every $\kappa$-saturated structure is $\kappa$-homogeneous. 
\end{lemma} 
\begin{proof} 
Suppose that $f\colon A\rightarrow B$ is an elementary map between subsets of a $\kappa$-saturated structure $\MM=(M,\dots)$ of size strictly less than $\kappa$ and $a\in M$. Since $\MM$ is $\kappa$-saturated, then $f(\tp^\MM(a/A))$ is an $n$-type over $B$ by the definition of the map $f=S_n(f)^{-1}\colon S_n^\MM(A)\rightarrow S_n^\MM(B)$ in Lemma \ref{induced maps on type spaces}. By $\kappa$-saturation, it is realized by some $b\in M$ and then $g = f\cup \{(a,b)\}$ is elementary. 
\end{proof} 

The next problem shows that this is not a necessary condition. 

\begin{problem} \label{prime models are omega-homogeneous} 
Show that every prime model is $\omega$-homogeneous. 
\end{problem} 

If a structure is $\omega$-homogeneous, then we have the following test to determine whether it is $\aleph_0$-saturated. 

\begin{problem} 
Show that any $\omega$-homogeneous model $\MM$ of a complete theory $T$ that realizes every $n$-type with respect to $T$ for all $n\geq1$ is $\aleph_0$-saturated. 
\end{problem} 

%We now move to algebraic closures in saturated structures. Algebraic closure is also important later for instance in the proof of Morley's theorem. Here we need it for the promised result about quantifier elimination. 

In the next definition, we have a version of the algebraic closure in a structure $\MM$ that is defined only from the action of $\Aut(\MM)$. 

\begin{definition} 
Suppose that $A$ is a subset of a model $\MM=(M,\dots)$ of a theory $T$. 
\begin{enumerate-(a)} 
\item 
A \emph{conjugate} of $x\in M$ over $A$ is an element $g(x)\in M$, where $g$ is an automorphism of $\MM$ that fixes $A$ pointwise. 
\item 
The ${}^*$-algebraic closure $\acl^*(A)=\acl_{\MM}^*(A)$ is defined as the set of elements $x\in M$ with only finitely many conjugates over $A$. \footnote{This is a non-standard notation -- the algebraic closure is usually only considered in saturated structures and the two notions are equal there. }
\end{enumerate-(a)} 
\end{definition} 

Since automorphisms of $\MM$ that fix $A$ pointwise preserve the truth of formulas with parameters in $A$, we have $\acl(A)\subseteq\acl^*(A)$. We will show that these two notions are the same in saturated structures. This will use the next lemma. 

%\begin{lemma} 
%Suppose that $\kappa$ is an infinite cardinal and $\MM$ is a $\kappa$-saturated model of size $\kappa$. If a subset $X$ of $M^n$ is invariant under all automorphism of $\MM$ that fix a finite set $A$ pointwise, then $X$ is definable from parameters in $A$. 
%\end{lemma} 
%\begin{proof} 
%Let $A=\{a_0,\dots,a_{k-1}\}$. 
%We claim that for any $x,y\in M$ with $\tp^\MM(x/A)=\tp^\MM(y/A)$, there is an automorphism of $\MM$ that fixes $A$ pointwise and maps $x$ to $y$. This was shown in the proof that every $\kappa$-saturated model of size $\kappa$ is homogeneous. WHERE? 
%\end{proof} 

\begin{lemma} \label{orbits and types in saturated structures} 
If $\kappa$ is an infinite cardinal and $\MM$ is a saturated model of size $\kappa$, then two $n$-tuples are in the same $\Aut(\MM)$-orbit if and only if they have the same type. 
\end{lemma} 
\begin{proof} 
Since automorphisms preserve types, two tuples in the same $\Aut(\MM)$-orbit have the same type. Conversely, if two tuples $\vec{a}=(a_0,\dots,a_{n-1})$ and $\vec{b}=(b_0,\dots,b_{n-1})$ have the same type, then the function $f$ sending $a_i$ to $b_i$ can be extended to an automorphism. Since $\MM$ has size $\kappa$ and is $\kappa$-homogeneous by Lemma \ref{saturated structures are homogeneous}, it is easy to construct such an isomorphism as a union of partial isomorphisms. 
\end{proof} 

We first note an interesting consequence for definable subsets of $\aleph_0$-categorical structures. 

\begin{lemma} 
Given an $\aleph_0$-categorical structure $\MM=(M,\dots)$, the definable subsets of $M^n$ are precisely the $\Aut(\MM)$-invariant subsets of $M^n$, i.e. unions of $\Aut(\MM)$-orbits on $M^n$. 
\end{lemma} 
\begin{proof} 
It is clear that a definable subset of $M^n$ is $\Aut(\MM)$-invariant. For the converse, note that $\MM$ is $\omega$-saturated by the theorem of Engeler, Ryll-Nardzewski and Svenonius (Theorem \ref{Ryll-Nardzewski}). Since all types are isolated, Lemma \ref{orbits and types in saturated structures} implies that each $\Aut(\MM)$-orbit is definable. Since there are only finitely many orbits by Theorem \ref{oligomorphic groups and categoricity}, any union of orbits is also definable. 
\end{proof} 

\begin{theorem} \label{characterization of algebraic closure} 
If $\MM=(M,\dots)$ is a saturated structure of size $\kappa$ and $A$ is a subset of $M$ with $|A|<\kappa$, the following conditions are equivalent. 
\begin{enumerate-(a)} 
\item \label{characterization of algebraic closure 1} 
$b\in \acl(A)$ 
\item \label{characterization of algebraic closure 2} 
$b\in \acl^*(A)$ 
\item \label{characterization of algebraic closure 3} 
$\tp^\MM(b/A)$ has only finitely many realizations in $\MM$. 
\end{enumerate-(a)} 
\end{theorem} 
\begin{proof} 
\ref{characterization of algebraic closure 1} implies \ref{characterization of algebraic closure 2}, since automorphisms preserve the truth of formulas. Moreover, the implication from \ref{characterization of algebraic closure 2} to \ref{characterization of algebraic closure 3} holds since any two elements with the same type are in the same $\Aut_A(\MM)$-orbit, where $\Aut_A(\MM)$ denotes the group of $g\in \Aut(\MM)$ fixing $A$ pointwise, by Lemma \ref{orbits and types in saturated structures} applied to the action of $\Aut_A(\MM)$ on $M\setminus A$. 
We now assume that \ref{characterization of algebraic closure 3} holds. Suppose that there are exactly $n$ realizations of $\tp^{\MM}(b/A)$ in $\MM$ and let $\vec{x}=(x_0,\dots,x_n)$ (the point is that there are $n+1$ many free variables). By our assumption, the set $$\Sigma(\vec{x})=\{\varphi(x_i)\mid \varphi(x)\in\tp^\MM(b/A),\ i\leq n\}\cup\{\bigwedge_{i<j\leq n}x_i\neq x_j\}$$ 
is not realized in $\MM$. Since $\MM$ is $\kappa$-saturated, some finite subset of $\Sigma(\vec{x})$ is not satisfied in $\MM$. So there are formulas $\varphi_0(x), \dots, \varphi_k(x)\in \tp^\M(b/A)$ with $\MM\models (\bigwedge_{i\leq n}\bigwedge_{j\leq k}\varphi_j(x_i))\rightarrow \bigvee_{i<j\leq n} x_i=x_j$. Hence the formula $\bigwedge_{j\leq k}\varphi_j(x)$ defines a set of size at most $n$ that contains $b$. 
\end{proof} 

%\begin{problem} 
%Suppose that $\MM=(M,\dots)$ is $\kappa$-saturated. Show that for every subset $A$ of $M$ of size strictly less than $\kappa$ and every $n\geq1$, every $n$-type in $S_n^\MM(A)$ is realized in $\MM$. 
%\end{problem} 

\begin{question} 
Is Theorem \ref{characterization of algebraic closure} also true for sets $A$ of size $\kappa$?  
\end{question} 

We claim that in any $\aleph_0$-categorical theory, there are only finitely many $n$-types over any finite set for each $n\geq 1$. This follows immediately from the theorem of Engeler, Ryll-Nardzewski and Svenonius (Theorem \ref{Ryll-Nardzewski}, since $\omega$-saturation is equivalent for a structure $\M=(M,\dots)$ as an $\L$-structure or an $\L_A$-structure, if $A$ is a finite subset of $M$ (note that this can fail for infinite subsets). Therefore, we have the following useful result about algebraic closures. 

\begin{corollary} \label{finite algebraic closures} 
In $\aleph_0$-categorical structures, algebraic closures of finite sets are finite. 
\end{corollary} 

This holds since there are only finitely many $1$-types over a finite set $A$, and each $1$-type has finitely many realizations in $\acl(A)$. 

In particular, any $\aleph_0$-categorical structure is \emph{locally finite}, i.e. any finitely generated substructure is finite. This holds because the elements of the substructure $\langle A\rangle^\MM$ of such a structure $\MM=(M,\dots)$ generated by a finite subset $A$ of $M$ are exactly the elements $t^\MM(\vec{a})$ for some $\vec{a}\in A^{<\omega}$. But $t^\MM(\vec{a})\in\acl(A)$, since $\{t^\MM(\vec{a})\}$ is definable from $\vec{a}$. 

%Moreover, we will study interesting cases where the algebraic closure is \emph{trivial}, i.e. $\acl(A)=A$ for every finite set $A$. 
%Studying the algebraic closure abstractly is also important for the following topics, for instance for the proof of Morley's theorem. 

We now study a strengthening of the amalgamation property, which states that structures $B$ and $C$ with a common substructure $A$ can be amalgamated in such a way that their intersection is $A$. 

\begin{definition} 
A class $\KK$ of $\L$-structures has the \emph{strong amalgamation property} if for all embeddings $f_0\colon \AA\rightarrow \BB$ and $g_0\colon \AA\rightarrow \CC$ between structures in $\KK$, there are embeddings $g_0\colon \BB\rightarrow \DD$ and $g_1\colon \CC\rightarrow \DD$ into some $\DD\in\KK$ with $g_0\circ f_0=g_1\circ f_1$ and $\ran{g_0}\cap \ran{g_1}=\ran{g_0\circ f_0}$. 
%=\ran{g_1\circ f_1}$. 
\end{definition} 

We say that algebraic closure in a structure $\MM=(M,\dots)$ is \emph{trivial} if $\acl(A)=A$ for every subset $A$ of $M$ -- since every element of $\acl(A)$ is is in $\acl(A')$ for a finite subset $A'$ of $A$, it is sufficient to show this only for finite sets. To show that the strong amalgamation property is equivalent to trivial algebraic closure in the Fraisse limit, we need the following lemma. 

\begin{lemma} [Neumann] \label{Neumann's lemma} 
Suppose that a group $G$ acting on a set $X$ has only infinite orbits, and let $A$ and $B$ be finite subsets of $X$. There there is some $g\in G$ with $g A \cap B=\emptyset$. 
\end{lemma} 
\begin{proof} 
We prove this by induction on $|A|$. It is clear for $|A|=0$. We now assume that the statement of the lemma fails for a nonempty set $A$, but holds for all sets $A'$ with $|A'|<|A|$. 

\begin{claim} 
For any subset $C$ of $X$ with $|C|\leq |A|$, only finitely many translates of $A$ contain $C$. 
\end{claim} 
\begin{proof} 
This is proved by induction on $|A|-|C|$. It is clearly true for sets $C$ with $|A|-|C|=0$. Now assume that $|C|<|A|$ and the claim holds for all $C'$ with $|C'|>|C|$. By the induction hypothesis for the lemma, we can assume (by translating $C$ by some $g\in G$) that that $B\cap C=\emptyset$. Moreover, by the induction hypothesis for the claim, for each $b\in B$ only finitely many translates of $A$ contain $C\cup\{b\}$. So only finitely many translates of $A$ contain $C$ and meet $B$. Since we are assuming that the lemma fails for $A$, every translate of $A$ meets $B$. So the claim holds for $C$. 
\end{proof} 
By letting $C=\emptyset$, the claim shows that $A$ has only finitely many translates. But this contradicts the assumption that $G$ has only infinite orbits. 
\end{proof} 

In the next proof, we will work with the following subgroups of automorphism groups: We define $\Aut_A(\MM)$ for any structure $\MM=(M,\dots)$ and any subset $A$ of $M$ as the group of all $g\in \Aut(\MM)$ that fixes $A$ pointwise, i.e. the \emph{pointwise stabilizer of $A$}. 



\begin{lemma} 
Suppose that $\L$ is a countable language and $\MM$ is a homogeneous $\omega$-categorical $\L$-structure. Then the algebraic closure in $\MM$ is trivial if and only if the skeleton of $\MM$ has the strong amalgamation property. 
\end{lemma} 
\begin{proof} 
First suppose that the skeleton $\KK$ of $\MM$ has the strong amalgamation property. Let $A$ be a finite subset of $M$ and $b\in M\setminus A$. Since we have to show that $b$ is in an infinite $\Aut_A(M)$-orbit, it is sufficient to show that for every $n\in\NNN$ there are at least $n$ elements in this orbit. Let $B$ be the substructure $A\cup\{b\}$ of $M$. We can apply strong amalgamation $(n-1)$ times to obtain a substructure $C$ of $\MM$ that consists of $n$ distinct copies of $B$ amalgamated over $A$. So $C=A\cup\{b_0,\dots,b_{n-1}\}$. By homogeneity we can assume that $b_0=b$. For any $i<j<n$, the isomorphism $f\colon A\cup\{b_i\}\rightarrow A\cup\{b_j\}$ fixing $A$ pointwise and mapping $b_i$ to $b_j$ extends to an automorphism of $\MM$ by homogeneity. Hence $b_i$ and $b_j$ are in the same $\Aut_A(\MM)$-orbit. 

Conversely, suppose that the algebraic closure is trivial in $\MM$. Take embeddings $f_0\colon A\rightarrow B$ and $f_1\colon A\rightarrow C$; we can assume that $f_0$ is the identity. 
%Let $C'=\ran{f_1}$. 
By homogeneity, there is an automorphism $h$ of $\MM$ extending $f_1$. Then $h^{-1}(C)\supseteq A$. We now apply Neumann's lemma (Lemma \ref{Neumann's lemma}) to $\Aut_A(\MM)$ acting on $M\setminus A$. Then $(k\circ h^{-1})(C)\cap B=A$ for some $k\in \Aut_A(\MM)$. Now $D=(k\circ h^{-1})(C)\cup B$ is a strong amalgamation of $f_0$ and $f_1$ via the inclusion $g_0\colon B\rightarrow D$ and the embedding $g_1\colon C\rightarrow D$ with $g_1=(k\circ h^{-1}){\upharpoonright}C$. 
\end{proof} 

\begin{problem} 
Find concrete finitely generated fields witnessing failures of strong amalgamation for any characteristic. 
\end{problem} 





\subsection{Quantifier elimination and homogeneity} 

We now connect the purely syntactical condition of quantifier elimination with homogeneity, a property of structures. 
%This explains why quantifier elimination is useful to prove homogeneity and conversely. 
For the following statements, recall that a relational language is one without constant and function symbols. 
\begin{itemize} 
\item 
If the language is finite and relational, then quantifier elimination is equivalent to homogeneity of countable models (Theorem \ref{quantifier elimination for relational structures}). Moreover, this implies $\aleph_0$-categoricity (Lemma \ref{quantifier elimination and categoricity for relational structures}). 
\item 
If the theory is $\aleph_0$-categorical, then the same equivalence 
%of quantifier elimination and homogeneity of countable models 
holds without assumptions on the language (Theorem \ref{characterization of quantifier elimination for aleph0-categorical structures}). 
\end{itemize} 

\begin{lemma} \label{quantifier elimination and categoricity for relational structures} 
If $T$ is a complete theory in a finite relational language with infinite models and quantifier elimination, then $T$ is $\aleph_0$-categorical. 
\end{lemma} 
\begin{proof} 
There are only finitely many quantifier-free formulas in the free variables $x_0,\dots,x_{n-1}$ up to equivalence modulo $T$ (consider formulas in disjunctive normal form). Hence $S_n(T)$ is finite for all $n\geq 1$. Thus the claim follows from the theorem of Engeler, Ryll-Nardzewski and Svenonius (Theorem \ref{Ryll-Nardzewski}). 
\end{proof} 


\begin{theorem} \label{quantifier elimination for relational structures}
If $T$ is a complete theory in a finite relational language and $\MM$ is an infinite model of $T$, then the following conditions are equivalent. 
\begin{enumerate-(a)} 
\item \label{quantifier elimination for relational structures 1}
$T$ has quantifier elimination. 
\item \label{quantifier elimination for relational structures 2}
The domain of any isomorphism between finite substructures of $\MM$ can be extended by one further element (if $\MM$ is countable, this implies that there is an extension to an automorphism of $\MM$ and hence it is equivalent to homogeneity). 
\item \label{quantifier elimination for relational structures 3}
Any isomorphism between finite substructures of $\MM$ is elementary. 
\end{enumerate-(a)} 
\end{theorem} 
\begin{proof} 
Assume \ref{quantifier elimination for relational structures 1}. 
%that $T$ has quantifier elimination.
Then any isomorphism between finite substructures of $\MM$ is elementary. Since $T$ is $\aleph_0$-categorical by Lemma \ref{quantifier elimination and categoricity for relational structures}, any countable model of $T$ is $\omega$-saturated by the Theorem of Engeler, Ryll-Nardzewski and Svenonius \ref{Ryll-Nardzewski}. Since any model of $T$ has a countable elementary submodel, it follows that every model of $T$ is $\omega$-saturated. Hence any model of $T$ is also $\omega$-homogeneous by Lemma \ref{saturated structures are homogeneous} and condition \ref{quantifier elimination for relational structures 2} follows. 

Now assume \ref{quantifier elimination for relational structures 2}. Clearly any isomorphism $f\colon A\rightarrow B$ between finite substructures of $\MM$ preserves the truth of quantifier-free formulas. We prove \ref{quantifier elimination for relational structures 3} by induction on formulas and will only do the existential case, since the other cases are easier. To this end, assume that any isomorphism $f\colon A\rightarrow B$ between finite substructures of $\MM$ preserves the truth of a formula $\varphi(x,\vec{y})$ and that $\exists x\ \varphi(x,\vec{a})$ holds in $\MM$ for a tuple $\vec{a}=(a_0,\dots,a_{n-1})$. Pick a witness $a\in M$ and apply \ref{quantifier elimination for relational structures 2} to obtain some $b\in N$ such that $g=f\cup\{(a,b)\}$ is an isomorphism. Then $\exists x\ \varphi(x,f(\vec{a}))$ holds in $\MM$ by the induction hypothesis for $g$. 

We finally assume \ref{quantifier elimination for relational structures 3}. Let 
$$\tp_{\mathrm{at}}^\MM(\vec{a})=\{ \theta(\vec{x})\mid \MM\models \theta(\vec{a})\text{ and $\theta$ is basic} \}$$ 
denote the \emph{atomic type} of any $\vec{a}=(a_0,\dots,a_{n-1})\in M^n$. Moreover, let $\Sigma_0(\vec{x})$, \dots, $\Sigma_{k-1}(\vec{x})$ list the atomic types $\tp_{\mathrm{at}}^\MM(\vec{a})$ of all $\vec{a}=(a_0,\dots,a_{n-1})\in M^n$ with $\MM\models \varphi(\vec{a})$. Let further $\theta_i(\vec{x})=\bigwedge_{\psi(\vec{x})\in \Sigma_i(\vec{x})}\psi(\vec{x})$ for all $i<k$. By \ref{quantifier elimination for relational structures 3}, $\forall \vec{x}(\varphi(\vec{x})\leftrightarrow \bigvee_{i<k}\theta_i(\vec{x}))$ holds in $\MM$. Since the sets $\Sigma_i(\vec{x})$ depend only on $T$ (and not on $\MM$), this holds in any model of $T$ and thus \ref{quantifier elimination for relational structures 1} follows. 
\end{proof} 

For example, this can be applied to the Fraisse limit of any amalgamation class in a finite relational language. Recall that this is homogeneous, so condition \ref{quantifier elimination for relational structures 2} holds and therefore its theory has quantifier elimination: 

\begin{corollary} 
If $\MM$ is the Fraisse limit of an amalgamation class of finite $\L$-structures in a finite relational language $\L$, then $\Th(\MM)$ has quantifier elimination. 
\end{corollary} 
%\begin{proof} 
%We already showed that $\MM$ is homogeneous, so condition \ref{quantifier elimination for relational structures 2} holds. 
%\end{proof} 

We obtain the following characterization of quantifier elimination for $\aleph_0$-categorical structures. 

\begin{lemma} \label{characterization of quantifier elimination for aleph0-categorical structures} 
The theory $T$ of an $\aleph_0$-categorical $\L$-structure $\MM$ has quantifier elimination if and only if $\MM$ is homogeneous. 
\end{lemma} 
\begin{proof} 
Assume that $\MM$ is $\aleph_0$-categorical and $f\colon A\rightarrow B$ is an isomorphism between finitely generated substructures of $\MM$. By Corollary \ref{finite algebraic closures}, $A$ and $B$ are finite. Let $\vec{a}=(a_0,\dots,a_{n-1})$ and $\vec{b}=(b_0,\dots,b_{n-1})$ enumerate $A$ and $B$. Then $\vec{a}$ and $\vec{b}$ satisfy the same quantifier-free formulas in $\MM$. Thus $\tp^\MM(\vec{a})=\tp^\M(\vec{b})$ by quantifier elimination. Finally, there is an isomorphism of $\MM$ extending $f$ by Lemma \ref{characterization of algebraic closure}. 

For the reverse direction, we claim each type in $T$ is determined by its quantifier-free part. By Problem 23, this shows that $T$ has quantifier elimination. To prove the claim, suppose that $\vec{a}=(a_0,\dots,a_{n-1})$ and $\vec{b}=(b_0,\dots,b_{n-1})$ are tuples with the same atomic types as defined in the proof of Theorem \ref{quantifier elimination for relational structures}. Then there is an isomorphism $f$ between the substructures generated by them (this is always the case -- it does not use the assumption that the theory is $\aleph_0$-categorical). Since $\MM$ is homogeneous, there is an automorphism of $\MM$ that extends $f$. Hence $\vec{a}$ and $\vec{b}$ have the same type. 
%(?:) If a?, a?? are tuples in M with the same quantifier free type, then a?  ? a?? extends to an isomorphism f : A ? A? between the substructures generated by a?,a ??. By homogeneity, there is an automorphism of M extending f and so a?,a?? have the same type over ? in M. So quantifier-free types determine types (over ?) in M; as all types are principal it follows that every formula is equivalent to a quantifier free formula (in M).
\end{proof} 

%Recall that we showed that the class of countably infinite homogeneous structures is equal to the class of Fraisse limits (of amalgamation classes). 




\subsection{More on saturated structures} 

Saturated models do not necessarily exist for all complete theories. But with additional set-theoretic assumptions they do: for any infinite cardinal $\kappa$ with $\kappa^{<\kappa}=\kappa$ one can easily construct a saturated model as the union of a chain of models of length $\kappa$, where each model is strictly smaller than $\kappa$ and all types over subsets are realized on the way. Moreover, assuming a version of $\mathsf{ZFC}$ with global choice such as G\"odel-Bernays class theory $\mathsf{GBC}$, one can construct a fully saturated model of proper class size in the same way -- this is called a \emph{monster model} (see \cite[Theorem 6.1.7]{MR2908005}). 

However, for many interesting theories saturated models always exists, for instance for $ACF_p$. 

\begin{lemma} 
An algebraically closed field $K$ is saturated if and only if its transcendence degree is infinite. 
\end{lemma} 
\begin{proof} 
Suppose that $A$ is a finite subset of $K$ and $R$ is the subring generated by $A$. Let $t(x)$ be the $1$-type generated by the set $\{f(x)\neq0\mid 0\neq f\in R[X]\}$, which says that $x$ is transcendental over $R$. If $K$ is $\omega$-saturated, then $t(x)$ is realized in $K$ and hence $A$ cannot be a transcendence base for $K$, so $K$ has infinite transcendence degree. 

Conversely, suppose that $K$ has infinite transcendence degree and $A$ is a subset of $K$ of size strictly less than $|K|$. Moreover, suppose that $t(x)$ is a $1$-type over $A$. Let further $R$ be the subring of $K$ generated by $A$ and $L$ its quotient field in $K$. If $t(x)$ is the trancendental type over $L$, i.e. $(f(x)=0)\notin t(x)$ for every nonzero polynomial $f\in L[X]$, then it is realized in $K$. Otherwise let $M$ be an elementary extension of $K$ in which $t(x)$ is realized by some $a\in M$ and let $f\in L[X]$ be the minimal polynomial of $a$. As we have argued in Example \ref{example: types in algebraically closed fields}, the minimal polynomial $f$ completely determines the type of $a$ over $L$. Additionally, the type is absolute between fields containing $L(a)$ by quantifier elimination. Since $K$ is algebraically closed, $f(x)$ has a zero in $K$ and this realizes $t(x)$. 
\end{proof} 

We showed that $\acl(A)=\acl^*(A)$ for subsets $A$ strictly smaller than $\kappa$ of any saturated structure of size $\kappa$. The next problem shows that saturation is not a necessary condition. 

\begin{problem} 
Show that $\acl_K(A)=\acl_K^*(A)$ for arbitrary subsets $A$ of any algebraically closed field $K$. 
\end{problem} 

While a prime model is the least model of a theory with respect to elementary embeddability, any saturated model is maximal for models of at most its size. If $\lambda$ is an infinite cardinal, a model $\MM$ of $T$ is called \emph{$\lambda$-universal} if every model of $T$ of size strictly smaller than $\lambda$ is elementarily embeddable into $\MM$.  

\begin{lemma} 
If $T$ is a complete theory and $\kappa$ is an infinite cardinal, then every $\kappa$-saturated model $\NN=(N,\dots)$ of $T$ is $\kappa^+$-universal. 
\end{lemma} 
\begin{proof} 
Suppose that $\MM=(M,\dots)$ is a model of $T$ of size at most $\kappa$. Moreover, fix an enumeration $\langle a_i\mid i<\kappa\rangle$ of $N$ and let $A_j=\{a_i\mid i<j\}$ for all $j\leq \kappa$. We  now construct a sequence $\langle f_j\mid j\leq \kappa\rangle$ of elementary embeddings $f_j\colon A_j\rightarrow B_j$ from $\MM$ to $\NN$, where $B_j$ is a subset of $N$. Let $f_0$ be the empty function and let $f_j=\bigcup_{i<j}f_i$ for limits $j\leq\kappa$. If $f_j$ is already defined, by $\kappa$-homogeneity of $\NN$ there is some $b\in N$ such that $f_{j+1}=f_j\cup\{(a_j,b)\}$ is elementary. 
\end{proof} 

The next result proves the converse implication. 

\begin{lemma} 
If $T$ is a complete theory of size at most $\kappa$, where $\kappa$ is an infinite cardinal, then every $\kappa$-homogeneous $\kappa^+$-universal model $\NN=(N,\dots)$ of $T$ is $\kappa$-saturated. 
\end{lemma} 
\begin{proof} 
Suppose that $A$ is a subset of $N$ with $|A|<\kappa$ and $t(x)\in S_1^\NN(A)$. By the L\"owenheim-Skolem theorem, there is an infinite elementary substructure $\MM=(M,\dots)$ of $\NN$ containing $A$ with $|M|\leq\kappa$. Since $t(x)\in S_1^\NN(A)$, there is an elementary extension $\MM'=(M',\dots)$ of $M$ of the same size as $M$ in which $t(x)$ is realized by some $a\in M'$. Since $\NN$ is $\kappa^+$-universal, there is an elementary embedding $f\colon \MM'\rightarrow \NN$. Then $g=f^{-1}\colon f(A)\rightarrow A$ is elementary with respect to $\NN$ and $\MM'$. Since $\NN$ is $\kappa^+$-homogeneous, there is some $b\in N$ such that $g\cup\{(f(a),b)\}$ is elementary. Then $b$ realizes $t(x)$ in $\NN$. 
\end{proof} 





\bibliographystyle{alpha}
\bibliography{references}



















 
 












  
\end{document}
