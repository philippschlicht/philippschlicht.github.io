\documentclass[a4paper,oneside,11pt]{article}

\usepackage{amsmath,amsthm,amsfonts,amssymb}
\usepackage[latin2]{inputenc}
\usepackage[english]{babel}
%\usepackage[bottom=3.0cm,top=3.0cm,left=3.0cm,right=3.0cm]{geometry}
\usepackage[a4paper, margin=2.8cm]{geometry}

\renewcommand\refname{\large{References}}
\newcommand\comment[1]{}

\begin{document}
  \thispagestyle{empty}

\begin{center}
\large{\bf 
Perfect set games and colorings on generalized Baire spaces
}
\end{center}
\begin{center} Dorottya Szir\'aki 
\end{center}

The notion of perfectness can be generalized
for the $\kappa$-Baire space
in a number of different ways (when $\kappa=\kappa^{<\kappa}>\omega$). 
We 
discuss
the connections between 
these different generalizations and between 
the games underlying some of their definitions, 
as well as 
the corresponding generalizations of 
scatteredness,
density in itself 
and the Cantor-Bendixson hierarchy.
For example, we show that V\"a\"an\"anen's generalized Cantor-Bendixson theorem is equivalent to the $\kappa$-perfect set property, and is therefore equiconsistent with the existence of an inaccessible cardinal above~$\kappa$. 
If time permits, we will mention 
analogues of the above results
for variants of these
perfect set games associated to open colorings.
As an application, we present 
a Cantor-Bendixson theorem
for independent subsets of ${}^\kappa\kappa$ with respect to~$\mathbf{\Pi}^0_2(\kappa)$ colorings.
\end{document}
