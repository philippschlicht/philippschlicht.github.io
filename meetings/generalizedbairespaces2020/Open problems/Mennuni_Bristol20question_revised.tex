\documentclass[a4paper, 11pt]{article}
\usepackage[T1]{fontenc} %WARNING make sure you have the package "cm-super" installed, or this line will produce ugly bitmap fonts. If you cannot install cm-super, probably it is better to comment this line
\usepackage[utf8]{inputenc}
\usepackage[british]{babel}
\usepackage{amsfonts}
\usepackage{amsmath} 
\usepackage{amsthm}  
\usepackage{amssymb}
\usepackage{mathtools}

\renewcommand{\phi}{\varphi}

\DeclarePairedDelimiter{\set}{\{}{\}}
\DeclarePairedDelimiter{\abs}{\lvert}{\rvert}

\theoremstyle{definition}
\newtheorem{thm}{Theorem}
\newtheorem*{question}{Question}
\newtheorem*{disclaimer}{Disclaimer}
\title{A question for the Fifth Workshop on Generalised Baire Spaces}
\author{Rosario Mennuni}
\date{}
  \begin{document}
  \maketitle
\begin{disclaimer}
The answer to this question might be easy or already known.
\end{disclaimer}


The most basic version of the Omitting Type Theorem states:
\begin{thm}
  Let $T$ be a  consistent first-order theory in a countable language, $x=(x_0,\ldots,x_{n-1})$ an $n$-tuple of variables, and $\pi(x)$ a partial $n$-type over $\emptyset$ such that for no formula $\phi(x)$ consistent with $T$ we have $T\cup \set{\phi(x)}\vdash \pi(x)$. Then there is a countable  $M\models T$ such that no element of $M^n$ realises $\pi(x)$.
\end{thm}
A better version states (see~\cite[Theorem~10.3]{poizat}):
\begin{thm}\label{thm:ott_meagre}
  Let $T$ be a consistent first-order theory in a countable language, and for every $n\in \omega$ let $A_n$ be a meagre subset of the space $S_n(T)$ of $n$-types over $\emptyset$. Then there is a countable $M\models T$ such that, for every $n\in \omega$ and $p(x)\in A_n$, no element of $M^n$ realises $p(x)$.
\end{thm}
The basic version can be generalised to what is called the \emph{$\kappa$-omitting types theorem} (see~\cite[Theorem~2.2.19]{changkeisler}).
\begin{thm}\label{thm:kappaott}
  Let $T$ be a consistent first-order theory in a language of size $\kappa$, $x=(x_0,\ldots, x_{n-1})$ an $n$-tuple of variables, and $\pi(x)$ a  partial $n$-type over $\emptyset$ such that for no set of formulas $\Phi(x)$ of size $<\kappa$  consistent with $T$ we have $T\cup \Phi(x)\vdash \pi(x)$. Then there is $M\models T$ with $\abs M\le \kappa$ and such that no element of $M^n$ realises $\pi(x)$.
\end{thm}


\begin{question}
Let $L$ be a language of size $\kappa$, and let $T$ be a consistent $L$-theory.  Equip each $S_n(T)$ with the topology  generated by declaring partial types of size $<\kappa$ to induce open sets, and  replace ``meagre'' with ``$\kappa$-meagre''. For which cardinals $\kappa$ does Theorem~\ref{thm:kappaott} generalise in the fashion of Theorem~\ref{thm:ott_meagre}?
\end{question}
\begin{thebibliography}{0}
  \bibitem{changkeisler}
C.C.~\textsc{Chang}, H.J.~\textsc{Keisler}.
\newblock \emph{Model Theory},
\newblock Third edition, Studies in Logic and the Foundations of Mathematics 73. North-Holland Publishing Co., Amsterdam, (1990). 
\bibitem{poizat}
B.~\textsc{Poizat}.
\newblock \emph{A Course in Model Theory}.
\newblock Universitext. Springer (2000).
\end{thebibliography}
\end{document}


